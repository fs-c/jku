\documentclass{article}
\usepackage[utf8]{inputenc}
\usepackage[ngerman]{babel}

% Convenience improvements
\usepackage{csquotes}
\usepackage[inline]{enumitem}
\usepackage{amsmath}
\usepackage{amssymb}
\usepackage{mathtools}
\usepackage{tabularx}

\usepackage{multicol}

% Proper tables and centering for overfull ones
\usepackage{booktabs}
\usepackage{adjustbox}

\usepackage[top=1cm, bottom=1cm, left=1cm, right=1cm, ]{geometry}

\usepackage{parskip}

% Drawings  
\usepackage{tikz}

% Units
\usepackage{siunitx}
\sisetup{locale=DE,round-mode=figures,round-precision=4,round-pad=false}

\begin{document}
\begin{multicols}{2}

    \section{Identification}

    \paragraph{Barcodes} usually object class (EAN, UPC) or object instance (EPC, basis RFID, successor).

    \paragraph{RFID} application to RFID reader to(data, clock, energy) RFID tag (transponder). transmission reader/tag contactless (inductor, microwave-antenna). EEPROM (common, high-energy writes)/SRAM (fast, needs constant energy). state machine vs microprocessor. operating freq (LF-30-300kHz, HF-3-30MHz, UHF) defines range.

    \paragraph{NFC} passive (initiator provides carrier field, target answers by modulating), active (initiator and target commite by alternately generating fields, full duplex), 13.56MHz, 4cm, 424kbit/s.

    \section{Authentication}

    Verify the identity: posession (key, card), knowledge (secret: password, non-secret: user id), biometric (physiological: fingerprint, iris, face, retina, dna, veins, behavioral: walking, patterns).

    Authorization: access control, permission, rights, privileges.

    Authentication: validating [and figuring out] the claimed identity. (1:1, actual is compared against claimed identity, 1:n, identity is found out)

    \paragraph{Biometric} Identification criteria: universality, uniqueness, permanence, collectability, performance, acceptability, circumvention.

    FRR (False Rejection Rate, user who should be granted access is not), FAR (False Acceptance Rate, inverse), EER (Equal Error Rate)

    Fingerprints: minutia based (maps points where ridges start/stop/branch), image/pattern based (aligns and overlays images)

    Facial: geometry of the face, or relative distances between features, can use standard video camera, no physical contact, affected by lightning/glasses/expression/age

    Ears: very effective, geometry of the ear

    Iris: texture of colored part of the eye, very stable, affected by diseases, not retinal scanning (measures blood vessel patterns at back of eye, considered intrusive, takes 10-15s)

    Hand geometry: US nuclear power, DoD, timekeeping, easy, fast

    \section{Sensors}

    Seonsors collect data from environment, actuators control machines (may influence environment).

    Physical sensors measure physical quantities (size, weight, accel, field strength, temp, brightness) through induction, hall effect (gen of voltage through mag field), light-electric effects, piezoelectric effect (charge accumulated by mechanical stress)

    Position/Location/Orientation sensors (GPS, optical, magnetic, ultraonic)

    \paragraph{Light} photoresistors (resistance change i.r.t. light, affected by temp), photodiodes, phototransistors (potentially certain wavelength band)

    \paragraph{Temperature} contact methods (slow, but can predict from r.o.c., thermistors), non-contact methods (infrared optical, fast). NTC thermistors (material res dec with temp, metal-oxide), PTC thermistors (inverse, basically all metals), both nonlinear.

    \paragraph{Acoustic} pressure sensors (mic). resistive mic (not in use), condenser mic (parralel-plate capacitor, distance between plates changes with pressure, is converted to V), piezoelectric mic (piezoel. crystals, very high freq, ultrasonic), dynamic mic (sound moves diaphragm, loudspeaker in reverse)

    \paragraph{Acceleration} resisitive, capacitive, fiber optic, servo/force balance, vibrating quartz, piezoelectric (acceleration = force on some mass in the sensor, piezoelectric material converts this force to voltage)

    measured accel is combiation of gravity + actual movement, not easy to filter out gravity (one way is to use bandpass filter)

    \paragraph{Angle} gyroscope, fast spinning wheel inside two gimbals, keeps axis stable due to inertia. when combined with accelerometer: inertial measurement unit (IMO, 6 degree of freedom, XYZ, roll pitch yaw)

    \paragraph{Magnetometer} compass, measures magnetic field, can be used to determine orientation, can be affected by other magnetic fields (e.g. from motors). scalar m (strength), vector m (direction)

    \paragraph{Localization} two methods: tracking (sensor network tracks all positions) and positioning (moving object determines its own position)

    \paragraph{Radiolocation} usually for indoor positioning. received signal strength (no additional infra, most receivers can already estimate incoming signal strength), time of flight ($t_d = ((t_3 - t_0) - (t_2 = t_1)) / 2$, $d = ct_d$), time of arrival is only one packet with send time, receiver can calculate distance (synchronized clocks!)

    With ToA: Lateration is calc of current position of A with at least 2 other points B and C (2 equations). Adding third point (3 equations) allows elimination of unknown $t_a$.

    TDoA: constructs hyperbola with foci at receivers, intersection of hyperbolas gives position.

    BoundingBox method is similar to trilateration, using squares instead of circles. Less precision, reduced processing power (no floatingp math).

    Angle of Arrival, two techs: directional antenna is sweeping (rotating), receiver-array (no moving parts, more error, time diff in receiving signal at each single receiver)

    WiFi Fingerprinting: fingerprint is set of all APs in range with respective signal strength, generate fp for all locations of interest (grid?), compare current fp with stored fps

    Bluetooth 5.1: Angle of Arrival, low cost high precision.

    \section{Actuators}

    Enable a microproc to affect phys environment.

\end{multicols}
\end{document}
