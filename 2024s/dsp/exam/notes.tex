\documentclass{article}
\usepackage[utf8]{inputenc}
\usepackage[ngerman]{babel}

\usepackage{multicol}

% Convenience improvements
\usepackage{csquotes}
\usepackage{enumitem}
\setlist[enumerate,1]{label={\alph*)}}
\usepackage{amsmath}
\usepackage{amssymb}
\usepackage{mathtools}
\usepackage{tabularx}

% Proper tables and centering for overfull ones
\usepackage{booktabs}
\usepackage{adjustbox}

% Change page/text dimensions, the package defaults work fine
\usepackage{geometry}

\usepackage{parskip}

% Adjust header and footer
\usepackage{fancyhdr}

\newcommand{\Deltaop}{\, \Delta\, }
\newcommand{\xor}{\, \oplus\, }
\newcommand{\id}{\text{id}}
\newcommand{\proj}{\text{proj}}

\begin{document}

\begin{multicols}{2}

    \section*{Sampling}

    \paragraph{Sampling Theorem} $f_s > 2f_{max}$. A signal is unambiguously represented by its samples if the sampling frequency is larger than twice the maximum frequency of the signal.

    The spectra repeat at multiples of the sampling frequency. If the sampling frequency is too low, the spectra overlap and aliasing occurs.

    \paragraph{Anti-Aliasing Filter} is useful if signal is not sufficiently band-limited (would violate sampling theorem). It is a low-pass filter that removes high-frequency components from the signal.

    A sharp cut-off filter is not ideal, as it would introduce ringing artifacts.

    \paragraph{Reconstruction Filter} Analogous to anti-aliasing filter, but used when converting sampled signal back to analog signal.

    \paragraph{Normalized Angular Frequency} $\Omega = 2\pi \frac{f_0}{f_s}$, in this case the signal is $2\pi$-periodic, the baseband $-\frac{f_s}{2} < f < \frac{f_s}{2}$ corresponds to $-\pi < \Omega < \pi$.

    \section*{Signal Properties}

    \paragraph{Even/Odd} $x(t)$ is even if $x(t) = x(-t)$ (symmetric around vertical), odd if $x(t) = -x(-t)$ (symmetric around origin).

    \paragraph{Periodic} $x(t)$ is periodic with period $T$ if $x(t) = x(t + T)$ for all $t$. (The smallest such $T$ is the fundamental period $T_0$.)

    \paragraph{Energy} $W = \int_{-\infty}^{\infty} |x(t)|^2 dt$. If $W$ is finite the signal is called analog-time energy signal.

    \paragraph{Power} $P = \lim_{T \to \infty} \frac{1}{T} \int_{-T/2}^{T/2} |x(t)|^2 dt$. If $P$ is finite the signal is called analog-time power signal. \textbf{Parseval's Theorem} states that the power of a periodic signal equals the sum of the powers of its harmonic components (in the Fourier Series).

    \section*{Fourier Series}

    \begin{align*}
        x(t) = \frac{a_0}{2} \sum_{k = 1}^{\infty} \left[ a_k \cos(2\pi k f_0 t) + b_k \sin(2\pi k f_0 t) \right]
    \end{align*}

    \section*{Discrete LTI Systems}

    \paragraph{Linearity} $\alpha T(x_1[n]) = T(\alpha x_1[n])$ and $T(x_1[n] + x_2[n]) = T(x_1[n]) + T(x_2[n])$

    \paragraph{Time-Invariance} $T(x[n - n_0]) = y[n - n_0]$

    \paragraph{Input-Output Behavior} A unit inpulse $\delta[n]$ causes the impulse response $h[n]$ at the output. In general we have
    \begin{align*}
        y[n] = x[n] \star h[n] = \sum_{i = -\infty}^{\infty} x[i] h[n - i]
    \end{align*}
    which is a discrete convolution, and
    \begin{align*}
        Y(f) = X(f) \star H(f) = \int_{-\infty}^{\infty} X(\tau) H(f - \tau) d\tau
    \end{align*}
    which is a continuous convolution.

    In particular $x[n] \star \delta[n - n_0] = x[n - n_0]$.

    \paragraph{Causality} A system is causal if $h[n] = 0$ for $n < 0$. (In other words, output only depends on past and present input.)

    \paragraph{Stability} A system is stable if $h[n]$ is bounded.

    \paragraph{FIR/IIR} FIR systems have an impulse response of finite length, IIR systems do not.

    \paragraph{Frequency Response} See DTFT, with $x = h$.

    \section*{DTFT}

    Results in the spectrum of a discrete-time signal, it is $2\pi$-periodic and continuous.
    \begin{align*}
        X(e^{j\Omega}) = \sum_{n = -\infty}^{\infty} x[n] e^{-j\Omega n}
    \end{align*}

    \paragraph{Resolution} is the distance between frequency points, $\Delta f = \frac{f_s}{N}$. Can be improved by zero-padding.

    \paragraph{z-Transform} $X(z) = \sum_{n = -\infty}^{\infty} x[n] z^{-n}$

    \paragraph{Leakage} the signal length is finite, this can be viewed as a convolution with a rectangular window. As a result, parts of the spectral energy are distributed over a wide frequency range. Reduced by windowing, eg with Hamming, Hanning, Blackman window.

    \section*{Digital Filters}

    \paragraph{Convolution} is the mathematical basis (since they are just LTI systems).

    \paragraph{Symmetric FIR Filters} have a (mirror or point) symmetric impulse response. Benefits: linear phase response, constant group delay. Disadvantages: high order, high computational complexity compared to IIR. Example: linear-phase low-pass filter.

    \paragraph{Window Method} Set the desired frequencies of an ideal rectangular filter and get the corresponding filter impulse response (infinitely long, not realizable). Window the impulse response and delay it (causality) to get a realizable filter.
\end{multicols}

\end{document}
