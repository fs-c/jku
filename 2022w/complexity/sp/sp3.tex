\documentclass{article}
\usepackage[utf8]{inputenc}
\usepackage[ngerman]{babel}

% Convenience improvements
\usepackage{csquotes}
\usepackage{enumitem}
\setlist[enumerate,1]{label={\alph*)}}
\usepackage{amsmath}
\usepackage{amssymb}
\usepackage{mathtools}
\usepackage{tabularx}

% Proper tables and centering for overfull ones
\usepackage{booktabs}
\usepackage{adjustbox}

% Change page/text dimensions, the package defaults work fine
\usepackage{geometry}

\usepackage{parskip}

% Drawings
\usepackage{tikz}
\usepackage{pgfplots}
\usetikzlibrary{automata,positioning} 

% Adjust header and footer
\usepackage{fancyhdr}
\pagestyle{fancy}
\fancyhead[L]{\textbf{Computational Complexity} --- Special Topic 3}
\fancyhead[R]{Laurenz Weixlbaumer (11804751)}
\fancyfoot[C]{}
\fancyfoot[R]{\thepage}
% Stop fancyhdr complaints
\setlength{\headheight}{12.5pt}

\usepackage{hyperref}

\newcommand{\Deltaop}{\, \Delta\, }
\newcommand{\xor}{\, \oplus\, }
\newcommand{\id}{\text{id}}

\begin{document}

\section*{Time Hierarchy Theorem}

Informally, the theorem states that given more time, a Turing machine can solve more problems. Formally, if $f(n)$ is a time-honest function then
\begin{align*}
    \mathsf{DTIME}(f(n)) \subsetneq \mathsf{DTIME}\left(f(n)^2 \right),
\end{align*}
with the understanding that this can be restated in more general terms (as in the lecture notes).

As suggested, I will prove a weaker version, showing that $\mathsf{DTIME}(f(n))$ is smaller than (thus a strict subset of) $\mathsf{DTIME}(f(2n + 1)^3)$. Following a hint from the lecture notes, let $A$ be the language that contains all TMs which accept an input after at most $f(|x|)$ steps.

\end{document}
