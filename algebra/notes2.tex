\documentclass{article}
\usepackage[utf8]{inputenc}
\usepackage[ngerman]{babel}

% Convenience improvements
\usepackage{csquotes}
\usepackage{enumitem}
\setlist[enumerate,1]{label={\alph*)}}
\usepackage{amsmath}
\usepackage{amssymb}
\usepackage{mathtools}
\usepackage{tabularx}

% Proper tables and centering for overfull ones
\usepackage{booktabs}
\usepackage{adjustbox}

% Change page/text dimensions, the package defaults work fine
\usepackage{geometry}

\usepackage{parskip}

% Drawings
\usepackage{tikz}

% Adjust header and footer
\usepackage{fancyhdr}
\pagestyle{fancy}
\fancyhead[L]{Algebra --- \textbf{Notizen 2}}
\fancyhead[R]{Laurenz Weixlbaumer (11804751)}
\fancyfoot[C]{}
\fancyfoot[R]{\thepage}
% Stop fancyhdr complaints
\setlength{\headheight}{12.5pt}

\newcommand{\Deltaop}{\, \Delta\, }
\newcommand{\xor}{\, \oplus\, }

\begin{document}

\paragraph{Beispiel Diophantische Gleichung}

\begin{align*}
    x' \cdot a + y' \cdot b &= g \\
    123x + 57y &= 6
\end{align*}

Lösbar wenn 6 teilbar durch $\text{ggt}(a, b)$. Eine Lösung ist vorletzte Zeile des EEA.

Homogene Gleichung: Eine Seite konstant.

\begin{align*}
    ua + vb &= 0 \\
    123u + 57v = 0
\end{align*}

Eine Lösung ist letzte Zeile des EEA. Kann nun beliebig skaliert werden nachdem eine Seite null. Kann aus demselben Grund beliebig oft auf die inhomogene Gleichung addiert werden.

\paragraph{Polynome}



\end{document}
