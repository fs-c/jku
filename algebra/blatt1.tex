\documentclass{article}
\usepackage[utf8]{inputenc}
\usepackage[ngerman]{babel}

% Convenience improvements
\usepackage{csquotes}
\usepackage{enumitem}
\setlist[enumerate,1]{label={\alph*)}}
\usepackage{amsmath}
\usepackage{amssymb}
\usepackage{mathtools}
\usepackage{tabularx}

% Proper tables and centering for overfull ones
\usepackage{booktabs}
\usepackage{adjustbox}

% Change page/text dimensions, the package defaults work fine
\usepackage{geometry}

\usepackage{parskip}

% Drawings
\usepackage{tikz}

% Adjust header and footer
\usepackage{fancyhdr}
\pagestyle{fancy}
\fancyhead[L]{Algebra --- \textbf{Übungsblatt 1}}
\fancyhead[R]{Laurenz Weixlbaumer (11804751)}
\fancyfoot[C]{}
\fancyfoot[R]{\thepage}
% Stop fancyhdr complaints
\setlength{\headheight}{12.5pt}

\newcommand{\Deltaop}{\, \Delta \,}

\begin{document}

% \paragraph{Aufgabe 1}

% \begin{enumerate}
%     \item \begin{center}
%         \begin{tabular}{c | c c c c c}
%             \toprule
%             \bottomrule
%         \end{tabular}
%     \end{center}
% \end{enumerate}

\paragraph{Aufgabe 2}

(Observation: $\Delta$ ist äquivalent zu $\oplus$.)

Zu zeigen ist, dass für $(\mathcal{P}(X), \Delta)$ das Assoziativgesetz gilt (Halbgruppe), dass ein neutrales Element exisitiert (Monoid), dass für jedes Element ein Inverses exisitiert (Gruppe) und, dass das Kommutativgesetz gilt (abelsch).

\begin{enumerate}
    \item Zu zeigen ist für beliebige $x, y \in M$, dass $x \Deltaop (y \Deltaop z) = (x \Deltaop y) \Deltaop z$.

    \item Zu zeigen ist, dass $\exists\, e \in M : \forall\, x \in M : e \Deltaop x = x \Deltaop e = x$.
\end{enumerate}

\end{document}
