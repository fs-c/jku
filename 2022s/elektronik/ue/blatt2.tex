\documentclass{article}
\usepackage[utf8]{inputenc}
\usepackage[ngerman]{babel}

% Convenience improvements
\usepackage{csquotes}
\usepackage{enumitem}
\setlist[enumerate,1]{label={\alph*)}}
\usepackage{amsmath}
\usepackage{amssymb}
\usepackage{mathtools}
\usepackage{tabularx}

% Proper tables and centering for overfull ones
\usepackage{booktabs}
\usepackage{adjustbox}

% Change page/text dimensions, the package defaults work fine
\usepackage{geometry}

\usepackage{parskip}

\usepackage{siunitx}
\sisetup{round-mode=places, round-precision=4, locale=DE}

% Drawings
\usepackage{tikz}
\usepackage{circuitikz}

% Adjust header and footer
\usepackage{fancyhdr}
\pagestyle{fancy}
\fancyhead[L]{Elektronik --- \textbf{Blatt 2}}
\fancyhead[R]{Laurenz Weixlbaumer (11804751)}
\fancyfoot[C]{}
\fancyfoot[R]{\thepage}
% Stop fancyhdr complaints
\setlength{\headheight}{12.5pt}

\begin{document}

\paragraph{Aufgabe 3}

\begin{enumerate}
    \item Es handelt sich um eine Serienschaltung von Widerst\"anden. Somit gilt
    \begin{align*}
        R_{ges} = R_i + R_a = 100\si{\ohm} + 50\si{\kohm} = \num{50.1}\si{\kohm}.
    \end{align*}
    Weiters gilt aus dem gleichen Grund, dass $I$ konstant bleibt. Somit kann $I$ durch
    \begin{align*}
        I = \frac{U_q}{R_{ges}} = \frac{8\si{\volt}}{50.1\si{\kohm}} = \num{0.1596806387225549}\si{\milli\ampere}
    \end{align*}
    berechnet werden. Damit k\"onnen die Spannungen
    \begin{align*}
        U_i &= R_i I = 100\si{\ohm} \cdot \num{0.1596806387225549}\si{\milli\ampere} = \num{15.96806387225549}\si{\milli\volt} \\
        U_k &= R_a I = 50\si{\kilo\ohm} \cdot \num{0.1596806387225549}\si{\milli\ampere} = \num{7.984031936127744}\si{\volt}
    \end{align*}
    berechnet werden.

    \item Es gilt
    \begin{align*}
        R_{ges} = \frac{U_q}{I} = \frac{7\si{\volt}}{100\si{\milli\ampere}} = 70\si{\ohm}
    \end{align*}
    und weiters
    \begin{align*}
        R_{ges} &= R_i + R_a \\
        R_a &= R_{ges} - R_i = 70\si{\ohm} - 8\si{\ohm} = 62\si{\ohm}.
    \end{align*}

    \item Wir haben
    \begin{align*}
        I_1 &= \num{2}\si{\ampere} & I_2 &= \num{3.2}\si{\ampere} \\
        U_{k_1} &= 12\si{\volt} & U_{k_2} &= 6\si{\volt}
    \end{align*}
    und k\"onnen somit
    \begin{align*}
        R_{a_1} &= \frac{U_{k_1}}{I_1} & R_{a_2} &= \frac{U_{k_2}}{I_2} \\
        R_{a_1} &= \frac{12\si{\volt}}{\num{2}\si{\ampere}} = 6\si{\ohm} & R_{a_2} &= \frac{6\si{\volt}}{\num{3.2}\si{\ampere}} = \num{1.875}\si{\ohm}
    \end{align*}
    berechnen. Es gilt $U_q = R_{ges} \cdot I$ beziehungsweise
    \begin{align*}
        U_q &= (R_{a_1} + R_i)I_1 & U_q &= (R_{a_2} + R_i)I_2 \\
        U_q &= (6\si{\ohm}+ R_i) \cdot \num{2}\si{\ampere} & U_q &= (\num{1.875}\si{\ohm}+ R_i) \cdot \num{3.2}\si{\ampere} \\
        U_q &= 12\si{\volt} + 2\si{\ampere} \cdot R_i & U_q &= 6\si{\volt} + \num{3.2}\si{\ampere} \cdot R_i
    \end{align*}
    Gleichsetzen der Gleichungen zur Ermittlung von $R_i$.
    \begin{align*}
        12\si{\volt} + 2\si{\ampere} \cdot R_i &= 6\si{\volt} + \num{3.2}\si{\ampere} \cdot R_i \\
        6\si{\volt} + 2\si{\ampere} \cdot R_i &= \num{3.2}\si{\ampere} \cdot R_i \\
        6\si{\volt} &= \num{1.2}\si{\ampere} \cdot R_i \\
        5\si{\ohm} &= R_i
    \end{align*}
    Einsetzen von $R_i$ zur Ermittlung von $U_q$.
    \begin{align*}
        U_q &= 12\si{\volt} + 2\si{\ampere} \cdot R_i & U_q &= 6\si{\volt} + \num{3.2}\si{\ampere} \cdot R_i \\
        U_q &= 12\si{\volt} + 10\si{\volt} & U_q &= 6\si{\volt} + 16\si{\volt} \\
        U_q &= 22\si{\volt} & U_q &= 22\si{\volt}
    \end{align*}
    Der maximale Strom $I_k$ kann durch
    \begin{align*}
        U_q &= (R_a + R_i)I_k \\
        22\si{\volt} &= (0\si{\ohm} + 5\si{\ohm})I_k \\
        I_k &= \frac{22\si{\volt}}{5\si{\ohm}} = \num{4.4}\si{\ampere}
    \end{align*}
    angegeben werden.
\end{enumerate}

\paragraph{Aufgabe 4} Es gilt
\begin{align*}
    R_{q} &= (R + R) \vert\vert R = \frac{(R + R)R}{R + R + R} = \frac{2R^2}{3R} = \frac{2R}{3} & R_{h} &= 2R_q & \frac{1}{R_{ges}} &= \frac{1}{R_h} + \frac{1}{R_h} + \frac{1}{R}
\end{align*}

Und somit nach Einsetzen und Aufl\"osen $R_{ges} = \frac{2R}{5}$.

\begin{center}
    \begin{circuitikz}[european, /tikz/circuitikz/bipoles/length=1cm, scale=.75]
        \draw (0,0) to[R=$R$, *-*] (3,0) to[R=$R$, *-*] (6,0);
        \draw (0,-3) to[R=$R$, *-*] (6,-3);
        \draw (0,-6) to[R=$R$, *-*] (3,-6) to[R=$R$, *-*] (6,-6);

        \draw (0,0) to[R=$R$, *-*] (0,-3) node[label=left:$E$]{} to[R=$R$, *-*] (0,-6);
        \draw (6,0) to[R=$R$, *-*] (6,-3) node[label=right:$F$]{} to[R=$R$, *-*] (6,-6);

        \draw (0,-3) to[R=$R$, *-*] (3,0) to[R=$R$, *-*] (6,-3);
        \draw (0,-3) to[R=$R$, *-*] (3,-6) to[R=$R$, *-*] (6,-3);
    \end{circuitikz}\quad
    \begin{circuitikz}[european, /tikz/circuitikz/bipoles/length=1cm, scale=.75]
        \draw (0,-3) node[label=left:$E$]{} to[R=$R$, *-*] (6,-3) node[label=right:$F$]{};

        \draw (0,-3) to[R=$R_q$, *-*] (3,0) to[R=$R_q$, *-*] (6,-3);
        \draw (0,-3) to[R=$R_q$, *-*] (3,-6) to[R=$R_q$, *-*] (6,-3);
    \end{circuitikz}\quad
    \begin{circuitikz}[european, /tikz/circuitikz/bipoles/length=1cm, scale=.75]
        \draw (0,-3) node[label=left:$E$]{} to[R=$R$, *-*] (6,-3) node[label=right:$F$]{};
        
        \draw (0,-3) to (0,-1.5) to[R=$R_h$] (6,-1.5) to (6,-3);
        \draw (0,-3) to (0,-4.5) to[R=$R_h$] (6,-4.5) to (6,-3);  
    \end{circuitikz}\quad
    \begin{circuitikz}[european, /tikz/circuitikz/bipoles/length=1cm, scale=.75]
        \draw (0,0) node[label=left:$E$]{} to[R=$R_{ges}$, *-*] (6,0) node[label=right:$F$]{};
    \end{circuitikz}
\end{center}

\paragraph{Aufgabe 5}

Unter der Annahme, dass die Spannung $U_2$ an $R_2$ anliegt, gilt gemäß der Spannung\-steilerformel, dass
\begin{align*}
    U_2 &= U_q \frac{R_2}{R_1 + R_2}
\end{align*}
und weiters
\begin{align*}
    U_4 = U_2 \frac{R_4}{R_3 + R_4}.
\end{align*}
Nach Einsetzen ergibt sich nun
\begin{align*}
    f(R_1, R_2, R_3, R_4, U_q) = U_q \cdot \frac{R_2}{R_1 + R_2} \cdot \frac{R_4}{R_3 + R_4}.
\end{align*}

\end{document}
