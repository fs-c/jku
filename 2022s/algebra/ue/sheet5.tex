\documentclass{article}
\usepackage[utf8]{inputenc}
\usepackage[ngerman]{babel}

% Convenience improvements
\usepackage{csquotes}
\usepackage{enumitem}
\setlist[enumerate,1]{label={\alph*)}}
\usepackage{amsmath}
\usepackage{amssymb}
\usepackage{mathtools}
\usepackage{tabularx}

% Because good typography is cool
\usepackage{microtype}

% Proper tables and centering for overfull ones
\usepackage{booktabs}
\usepackage{adjustbox}

% Change page/text dimensions, the package defaults work fine
\usepackage{geometry}

\usepackage{parskip}

% Drawings
\usepackage{tikz}
\usepackage{pgfplots}
\pgfplotsset{
    compat=1.17,
    default/.style={
        grid=both,
        grid style=dashed,
        scaled y ticks = false,
    },
}

% Adjust header and footer
\usepackage{fancyhdr}
\pagestyle{fancy}
\fancyhead[L]{Algebra --- \textbf{Exercise Sheet 5}}
\fancyhead[R]{Laurenz Weixlbaumer (11804751)}
\fancyfoot[C]{}
\fancyfoot[R]{\thepage}
% Stop fancyhdr complaints
\setlength{\headheight}{12.5pt}

\newcommand{\Deltaop}{\, \Delta\, }
\newcommand{\xor}{\, \oplus\, }
\newcommand{\id}{\text{id}}

\begin{document}

\paragraph{Exercise 1}

\begin{enumerate}
    \item The equations are equivalent to $y = 4x - 3$ and $y = \frac{-x + 1}{2}$ respectively and can thus be trivially drawn. They intersect at $x = \frac{7}{9}$.
    
    \item As determined previously we have
    \begin{align*}
        4x - 3 &= \frac{-x + 1}{2} \\
        8x - 6 &= -x + 1 \\
        9x &= 7 \\
        x &= \frac{7}{9}
    \end{align*}
    which is expected.
\end{enumerate}

\paragraph{Exercise 2}

\begin{enumerate}
    \item \phantom{}
    \begin{center}
        \begin{tabular}{c c c c c | c}
            I & \boxed{1} & 1 & 3 & 1 & 0 \\
            II & 0 & $-1$ & $-1$ & $-1$ & 1 \\
            III & 3 & 1 & 5 & 3 & 0 \\
            IV & 1 & 5 & 11 & 8 & 0 \\
        \end{tabular}
        \quad
        \begin{tabular}{c c c c c | c}
            I & \boxed{1} & 1 & 3 & 1 & 0 \\
            II & 0 & \boxed{-1} & $-1$ & $-1$ & 1 \\
            III - 3I & 0 & $-2$ & $-4$ & 0 & 0 \\
            IV - I & 0 & 4 & 8 & 7 & 0 \\
        \end{tabular}

        \begin{tabular}{c c c c c | c}
            I & \boxed{1} & 1 & 3 & 1 & 0 \\
            II & 0 & \boxed{-1} & $-1$ & $-1$ & 1 \\
            III - 2II & 0 & 0 & \boxed{-2} & 2 & $-2$ \\
            IV + 4II & 0 & 0 & 4 & 3 & 4 \\
        \end{tabular}
        \qquad
        \begin{tabular}{c c c c c | c}
            I & \boxed{1} & 1 & 3 & 1 & 0 \\
            II & 0 & \boxed{-1} & $-1$ & $-1$ & 1 \\
            III & 0 & 0 & \boxed{-2} & 2 & $-2$ \\
            IV + 2III & 0 & 0 & 0 & 7 & 0 \\
        \end{tabular}
    
        % \begin{tabular}{c c c c c | c}
        %     I & \boxed{1} & 1 & 3 & 1 & 0 \\
        %     II & 0 & \boxed{-1} & $-1$ & $-1$ & 1 \\
        %     III - 2II & 0 & 0 & \boxed{-2} & 2 & $-2$ \\
        %     IV + 4II & 0 & 0 & 6 & 3 & 4 \\
        % \end{tabular}
        % \qquad
        % \begin{tabular}{c c c c c | c}
        %     I & \boxed{1} & 1 & 3 & 1 & 0 \\
        %     II & 0 & \boxed{-1} & $-1$ & $-1$ & 1 \\
        %     III & 0 & 0 & \boxed{-2} & 2 & $-2$ \\
        %     IV + 3III & 0 & 0 & 0 & \boxed{11} & $-2$ \\
        % \end{tabular}
    \end{center}
    Somit 
\end{enumerate}

\end{document}
