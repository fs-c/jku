\documentclass{article}
\usepackage[utf8]{inputenc}
\usepackage[ngerman]{babel}

% Convenience improvements
\usepackage{csquotes}
\usepackage{enumitem}
\setlist[enumerate,1]{label={\alph*)}}
\usepackage{amsmath}
\usepackage{amssymb}
\usepackage{mathtools}
\usepackage{tabularx}

% Proper tables and centering for overfull ones
\usepackage{booktabs}
\usepackage{adjustbox}

% Change page/text dimensions, the package defaults work fine
\usepackage{geometry}

\usepackage{parskip}

% Drawings
\usepackage{tikz}
\usepackage{pgfplots}

% Adjust header and footer
\usepackage{fancyhdr}
\pagestyle{fancy}
\fancyhead[L]{Algebra --- \textbf{Exercise Sheet 7}}
\fancyhead[R]{Laurenz Weixlbaumer (11804751)}
\fancyfoot[C]{}
\fancyfoot[R]{\thepage}
% Stop fancyhdr complaints
\setlength{\headheight}{12.5pt}

\newcommand{\Deltaop}{\, \Delta\, }
\newcommand{\xor}{\, \oplus\, }
\newcommand{\id}{\text{id}}

\begin{document}

\paragraph{Exercise 1}

\begin{enumerate}
    \item The given subset is linearly dependent.
    \begin{center}
        \begin{tabular}{c c c | c}
            1 & $-1$ & 0 & 0 \\
            1 & 2 & 6 & 0 \\
            3 & 1 & 8 & 0 \\
        \end{tabular}
        \quad
        \begin{tabular}{c c c | c}
            \boxed{1} & $-1$ & 0 & 0 \\
            0 & 3 & 6 & 0 \\
            0 & 4 & 8 & 0 \\
        \end{tabular}
        \quad
        \begin{tabular}{c c c | c}
            1 & $-1$ & 0 & 0 \\
            0 & \boxed{3} & 6 & 0 \\
            0 & 0 & 0 & 0 \\
        \end{tabular}
    \end{center}

    \item The given subset is linearly dependent.
    \begin{center}
        \begin{tabular}{c c c | c}
            1 & 1 & 1 & 0 \\
            $t$ & $t-1$ & $t+1$ & 0 \\
            $t^2$ & $(t-1)^2$ & $(t+1)^2$ & 0 \\
        \end{tabular}
        \quad
        \begin{tabular}{c c c | c}
            \boxed{1} & 1 & 1 & 0 \\
            0 & $-1$ & $1$ & 0 \\
            0 & $-2t+1$ & $2t+1$ & 0 \\
        \end{tabular}
        \quad
        \begin{tabular}{c c c | c}
            1 & 1 & 1 & 0 \\
            0 & \boxed{$-1$} & $1$ & 0 \\
            0 & 0 & 0 & 0 \\
        \end{tabular}
    \end{center}
\end{enumerate}

\paragraph{Exercise 2}

\begin{enumerate}
    \item First we show that
    \begin{equation*}
        \forall\, \vec{u}, \vec{v} \in U \cap W : \vec{u} + \vec{v} \in U \cap W.
    \end{equation*}
    Since $U$ is a subspace, for all elements $\vec{u_1}, \vec{u_2} \in U$ we know $\vec{u_1} + \vec{u_2} \in U$. The same holds for the vectors in $W$. We know that the vectors $\vec{u}$ and $\vec{v}$ are in $U$ and $W$. Thus
    \begin{equation*}
        \vec{u} + \vec{v} \in U \land \vec{u} + \vec{v} \in W \quad\Longrightarrow\quad \vec{u} + \vec{v} \in U \cap W.
    \end{equation*}

    Second we show that
    \begin{equation*}
        \forall\, \vec{v} \in U \cap W : \forall \lambda \in V : \lambda \cdot \vec{v} \in U \cap W
    \end{equation*}
    We again know for all $\vec{u} \in U$ and $\lambda \in V$ that we have $\lambda \cdot \vec{u} \in U$ and analogously for $W$. Thus, since $\vec{v}$ is in both $U$ and $W$ the same reasoning as above applies.

    \item First we show that
    \begin{equation*}
        \forall\, \vec{v_1}, \vec{v_2} \in U + W : \vec{v_1} + \vec{v_2} \in U + W.    
    \end{equation*}
    Since
    \begin{align*}
        \vec{v_1} &= \vec{u_1} + \vec{w_1} \quad (\vec{u_1} \in U, \vec{w_1} \in W) \\
        \vec{v_2} &= \vec{u_2} + \vec{w_2} \quad (\vec{u_2} \in U, \vec{w_2} \in W)
    \end{align*}
    by definition of $U + W$ we can restate the condition to
    \begin{align*}
        \vec{v_1} + \vec{v_2} &\in U + W \\
        \vec{u_1} + \vec{w_1} + \vec{u_2} + \vec{w_2} &\in U + W \\
        \vec{u_1} + \vec{u_2} + \vec{w_1} + \vec{w_2} &\in U + W
    \end{align*}
    which is true since $\vec{u_1} + \vec{u_2} \in U$ and $\vec{w_1} + \vec{w_2} \in W$.

    Second we show that
    \begin{equation*}
        \forall\, \vec{v} \in U + W : \forall\, \lambda \in V : \lambda \cdot \vec{v} \in U + W
    \end{equation*}
    Since $\vec{v} = \vec{u} + \vec{w}$ (for some vectors $\vec{u}$ and $\vec{w}$ in $U$ and $W$ respectively) we can restate the above condition to
    \begin{align*}
         \lambda(\vec{u} + \vec{w}) &\in U + W \\
         \lambda \cdot \vec{u} + \lambda \cdot \vec{w} &\in U + W
    \end{align*}
    which is true since $\lambda \cdot \vec{u} \in U$ and $\lambda \cdot \vec{w} \in W$. (We can decompose the multiplication with $\lambda$ because $V$ is a vector space.)
\end{enumerate}

\paragraph{Exercise 3}

We show that $W_1 \cup W_2$ is a subspace if and only if $W_1 \subseteq W_2$ or $W_2 \subseteq W_1$.

First we show the \enquote{if}: Assume $W_1 \subseteq W_2$, then $W_1 \cup W_2 = W_2$ which is a subspace. The reasoning for $W_2 \subseteq W_1$ is very similar.

Second we show the \enquote{and only}: Let $w_1 \in W_1$ such that $w_1 \notin W_2$ and $w_2 \in W_2$ with $w_2 \notin W_1$. We claim $w_1 + w_2 \notin W_1 \cap W_2$ --- thus that $W_1 \cup W_2$ is not a subspace if neither is a subset of the other.

Proof by contradiction: Assume that $w_1 + w_2 \in W_1$. Since $-w_1$ is in $W_1$ (it is a subspace) we can construct the statement
\begin{align*}
    (w_1 + w_2) - w_1 &\in W_1 \\
    w_2 &\in W_1
\end{align*}
which is a contradiction. Same goes for the assumption that $w_1 + w_2 \in W_2$.

\paragraph{Exercise 5}

\begin{center}
    \begin{tikzpicture}
        \begin{axis}[domain=0:3*pi, axis lines=center, width=\linewidth, height=4cm, legend]
            \addplot[smooth, thick, dashed]{sin(deg(x))};
            \addlegendentry{$\sin(x)$};
            \addplot[smooth, dashed]{cos(deg(x))};
            \addlegendentry{$\cos(x)$};
            \addplot[smooth, thick]{sin(deg(2*x))};
            \addlegendentry{$\sin(2x)$};
            \addplot[smooth, very thick, dotted]{cos(deg(2*x))};
            \addlegendentry{$\cos(2x)$};
            \addplot[smooth, dotted]{sin(deg(x))*cos(deg(x))};
            \addlegendentry{$\sin(x)\cos(x)$};
        \end{axis}
    \end{tikzpicture}
\end{center}

There is no way to add the functions together to get $f(x) = 0$.

\paragraph{Exercise 7}

The base is
\begin{equation*}
    \left\{\begin{pmatrix*}
        1 \\ 0 \\ 0
    \end{pmatrix*}\begin{pmatrix*}
        0 \\ 1 \\ 0
    \end{pmatrix*}\begin{pmatrix*}
        0 \\ 0 \\ 1
    \end{pmatrix*}\right\}
\end{equation*}
and the dimension of the linear hull is 3.

\end{document}
