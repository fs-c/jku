\documentclass{article}
\usepackage[utf8]{inputenc}
\usepackage[ngerman]{babel}

% Convenience improvements
\usepackage{csquotes}
\usepackage{enumitem}
\setlist[enumerate,1]{label={\alph*)}}
\usepackage{amsmath}
\usepackage{amssymb}
\usepackage{mathtools}
\usepackage{tabularx}

% Proper tables and centering for overfull ones
\usepackage{booktabs}
\usepackage{adjustbox}

% Change page/text dimensions, the package defaults work fine
\usepackage{geometry}

\usepackage{parskip}

% Drawings
\usepackage{tikz}
\usepackage{pgfplots}

% Adjust header and footer
\usepackage{fancyhdr}
\pagestyle{fancy}
\fancyhead[L]{Algebra --- \textbf{Exercise Sheet 6}}
\fancyhead[R]{Laurenz Weixlbaumer (11804751)}
\fancyfoot[C]{}
\fancyfoot[R]{\thepage}
% Stop fancyhdr complaints
\setlength{\headheight}{12.5pt}

\newcommand{\Deltaop}{\, \Delta\, }
\newcommand{\xor}{\, \oplus\, }
\newcommand{\id}{\text{id}}
\newcommand{\proj}{\text{proj}}

\begin{document}

\paragraph{Exercise 1}

\begin{enumerate}
    \item We first determine a base of $U$ by noting that $x = -y - z$ solves $x + y + z = 0$ for arbitrary $y$ and $z$ leading to
    \begin{align*}
        \begin{pmatrix}
            x \\ y \\ z
        \end{pmatrix}
        =
        \begin{pmatrix}
            -y - z \\ y \\ z
        \end{pmatrix}
        =
        y
        \begin{pmatrix}
            -1 \\ 1 \\ 0
        \end{pmatrix}
        +
        z
        \begin{pmatrix}
            -1 \\ 0 \\ 1
        \end{pmatrix}
    \end{align*}
    which shows that $(-1, 1, 0)$ and $(-1,0,1)$ form a basis for $U$ (since any element of $U$ can be written as a linear combination of them).

    Consider
    \begin{align*}
        \dim(R^3) &= \dim(U) + \dim(R^3/U) \\
        3 &= 2 + \dim(R^3/U) \\
        \dim(R^3/U) &= 1
    \end{align*}
    thus we are looking for one more element in $R^3$ such that it forms a basis of $R^3$ alongside our existing vectors.

    %TODO

    \item We first determine a base of $U$. By solving the linear system $x + y + z = 0$ and $x + 2y + 3z = 0$.
    \begin{center}
        \begin{tabular}{c c c | c}
            1 & 1 & 1 & 0 \\
            1 & 2 & 3 & 0
        \end{tabular}\quad
        \begin{tabular}{c c c | c}
            1 & 1 & 1 & 0 \\
            0 & 1 & 2 & 0
        \end{tabular}
    \end{center}
    Thus
    \begin{align*}
        y + 2z &= 0 & x + y + z = 0 \\
        y &= -2z    & x - 2z + z = 0 \\
        && x = z
    \end{align*}
    leading to
    \begin{equation*}
        \begin{pmatrix}
            x \\ x \\ -2x
        \end{pmatrix}
        =
        x
        \begin{pmatrix}
            1 \\ 1 \\ -2
        \end{pmatrix}
    \end{equation*}
    which makes $(1, 1, -2)$ a basis of $U$.

    % TODO
\end{enumerate}

\paragraph{Exercise 2}

\begin{enumerate}
    \item To show that the given vectors form an orthogonal system it is necessary to show that they are pairwise orthogonal. (That for any vectors $\vec{v}$ and $\vec{u}$ we have $\vec{v} \cdot \vec{u} = 0$) This is the case here.
    
    \item To show that the given vectors form a basis we show that they are linearly independent (skipped, $x_1, \ldots, x_4 = 0$). Since we have four independent vectors of a four-dimensional vector space they form a basis.
    
    \item To determine the coordinates of of $\vec{a}$ in relation to the given basis we solve
    \begin{center}
        \begin{tabular}{c c c c | c}
            0 & -10 & 26 & 9 & 6 \\
            0 & -5 & 47 & -12 & 37 \\
            1 & 6 & 66 & 4 & 78 \\
            2 & -3 & -33 & -2 & -39 \\
        \end{tabular}
    \end{center}
    and get $\lambda_1 = 0$, $\lambda_2 = 2$, $\lambda_3 = 1$, $\lambda_4 = 0$.
\end{enumerate}

\paragraph{Exercise 3}

\begin{enumerate}
    \item \begin{enumerate}[label=(\roman*)]
        \item \begin{align*}
            x_1^2 + 2x_2^2 &\geq 0 \\
            x_1^2 + 2x_2^2 = 0 &\Leftrightarrow \vec{x} = \vec{0}
        \end{align*}

        \item \begin{align*}
            x_1y_1 + 2x_2y_2 & = y_1x_1 + 2y_2x_2
        \end{align*}

        \item \begin{align*}
            (\lambda x_1 + \phi y_1)z_1 + 2(\lambda x_2 + \phi y_2)z_2 &= \lambda(x_1z_1 + 2x_2z_2) + \phi(y_1z_1 + 2y_2z_2) \\
        \lambda x_1 z_1 + \phi y_1 z_1 + 2(\lambda x_2 z_2 + \phi y_2z_2) &= \lambda(x_1z_1 + 2x_2z_2) + \phi(y_1z_1 + 2y_2z_2) \\
        \lambda x_1 z_1 + \phi y_1 z_1 + 2\lambda x_2 z_2 + 2\phi y_z z_2 &= \lambda(x_1z_1 + 2x_2z_2) + \phi(y_1z_1 + 2y_2z_2)
        \end{align*}
    \end{enumerate}

    \item \begin{enumerate}[label=(\roman*)]
        \item Same as regular scalar product. (No, contradiction.)
        
        \item \begin{align*}
            x_1y_2 + x_2y_1 = y_1x_2 + y_2x_1
        \end{align*}

        \item \begin{align*}
            (\lambda x_1 + \phi y_1)z_2 + (\lambda x_2 + \phi y_2)z_1 &= \lambda(x_1z_2 + x_2z_1) + \phi(y_1z_2 + y_2z_1) \\
            \lambda x_1 z_2 + \phi y_1 z_2 + \lambda x_2 z_1 + \phi y_2 z_1 &= \lambda(x_1z_2 + x_2z_1) + \phi(y_1z_2 + y_2z_1)
        \end{align*}
    \end{enumerate}

    \item Not a scalar product.
    \begin{enumerate}[label=(\roman*)]
        \item Same as regular scalar product.
        
        \item \begin{align*}
            x_1y_1 + x_2y_1 &= y_1x_1 + y_2x_1 \\
            x_2y_1 &= y_2x_1 
        \end{align*}
        Consider $\vec{x} = (1, 2)$ and $\vec{y} = (3, 4)$, we now have $6 = 4$.
    \end{enumerate}

    \item Not a scalar product.
    \begin{enumerate}[label=(\roman*)]
        \item \begin{align*}
            x_1 + x_2 + x_1 + x_2 \geq 0
        \end{align*}
        Consider $\vec{x} = (-1, 0)$ we now have $-1 + 0 - 1 + 0 \geq 0$.
    \end{enumerate}

    \item Not a scalar product, definition requires $V \times V \rightarrow \mathbb{R}$ but $R_3 \neq R_2$.
    
    \item Can be restated to
    \begin{align*}
        \begin{pmatrix}
            \begin{pmatrix}
                x_1 \\ x_2 \\ x_3
            \end{pmatrix}, \begin{pmatrix}
                y_1 \\ y_2 \\ y_3
            \end{pmatrix}
        \end{pmatrix}
        \mapsto
        \begin{pmatrix}
            \langle (x_1,x_2),(y_1,y_2) \rangle \\
            \langle (x_1,x_3),(y_1,y_3) \rangle \\
        \end{pmatrix}
    \end{align*}
    and is thus a scalar product. (No, contradiction.)
\end{enumerate}

\paragraph{Exercise 4}

\begin{enumerate}
    % TODO
    \item To show that a set of vectors form an orthogonal basis we show that they are a basis (they are linearly independent, done?) and that they are pairwise orthogonal (trivial).
    
    \item Since the system
    \begin{center}
        \begin{tabular}{c c c | c}
            1 & 1 & 1 & 5 \\
            1 & -1 & 0 & 1 \\
            1 & 1& -1 & 6 \\
            1 & -1 & 0 & 3 \\
        \end{tabular}
    \end{center}
    does not have a solution, the given vector is not in $U$. (Cannot be constructed from linear combination of vectors in $U$.)

    % TODO
\end{enumerate}

\paragraph{Exercise 5}

\begin{description}
    \item[$i = 1$] \begin{equation*}
        w_1 = u
    \end{equation*}

    \item[$i = 2$] \begin{align*}
        w_2 = v - (\proj_{w_1}(v))
    \end{align*}
    with
    \begin{align*}
        \proj_{w_1}(v) = \frac{v \cdot w_1}{w_1 \cdot w_1} \cdot w_1 &= \begin{pmatrix}
            2 \\ 2 \\ 2
        \end{pmatrix}
    \end{align*}
    thus
    \begin{align*}
        w_2 = \begin{pmatrix}
            -1 \\ 0 \\ 1
        \end{pmatrix}.
    \end{align*}

    \item[$i = 3$] \begin{align*}
        w_3 = w - (\proj_{w_1}(w) + \proj_{w_2}(w))
    \end{align*}
    with
    \begin{align*}
        \proj_{w_1}(w) &= \begin{pmatrix}
            \frac{1}{3} \\ \frac{1}{3} \\ \frac{1}{3}
        \end{pmatrix}\\
        \proj_{w_2}(w) &= \begin{pmatrix}
            \frac{1}{2} \\ 0 \\ -\frac{1}{2}
        \end{pmatrix}
    \end{align*}
    thus
    \begin{align*}
        w_3 = \begin{pmatrix}
            \frac{1}{6} \\ -\frac{1}{3} \\ \frac{1}{6}
        \end{pmatrix}.
    \end{align*}
\end{description}

\paragraph{Exercise 6}

\begin{enumerate}[label=(\roman*)]
    % TODO
    \item \begin{align*}
        x_1^2 + x_1x_2 + x_2x_1 + 2x_2^2 - x_1x_3 - x_3x_1 + 3x_3^2 &\geq 0 \\
        x_1^2 + 2x_1x_2 + 2x_2^2 - 2x_1x_3 + 3x_3^2 &\geq 0 \\
        x_1^2 + 2x_2^2 + 3x_3^2 &\geq -2x_1x_2 + 2x_1x_3 
    \end{align*}

    \item \phantom{i}

    \makebox[\textwidth]{\parbox{1.3\textwidth}{%
    \begin{align*}
        x_1y_1+x_1y_2+x_2y_1+2x_2y_2-x_1y_3-x_3y_1+3x_3y_3 &= y_1x_1+y_1x_2+y_2x-1+2y_2x_2-y_1x_3-y_3x_1+3y_3x_3 \\
        0=0 \\
    \end{align*}
    }}

    \item \phantom{i}

    \makebox[\textwidth]{\parbox{1.3\textwidth}{%
    \begin{align*}
        & (\lambda x_1 + \phi y_1)z_1 + (\lambda x_1 + \phi y_1)z_2 + (\lambda x_2 + \phi y_2)z_1 + 2(\lambda x_2 + \phi y_2)z_2 - (\lambda x_1 + \phi y_1)z_3 - (\lambda x_3 + \phi y_3)z_1 + 3(\lambda x_3 + \phi y_3)z_3 \\
        &= \lambda (x_1z_1+x_1z_2+x_2z_1+2x_2z_2-x_1z_3-x_3z_1+3x_3z_3) + \phi ( y_1z_1+y_1z_2+y_2z_1+2y_2z_2-y_1z_3-y_3z_1+3y_3z_3)
    \end{align*}
    }}
\end{enumerate}

\paragraph{Exercise 7}

The angle can be calculated with
\begin{align*}
    \cos(\alpha) &= \frac{\langle u, v \rangle}{||u|| \cdot ||v||} = \frac{1}{\sqrt{\left(\sqrt{2}+1\right)^2-2 \sqrt{2}+1} \cdot 1} \\
    \alpha &= 60\deg
\end{align*}

\paragraph{Exercise 8}

% TODO

Gram-Schmidt auf Standardbasis von R3 anwenden. 

\end{document}
