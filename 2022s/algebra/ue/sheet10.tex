\documentclass{article}
\usepackage[utf8]{inputenc}
\usepackage[ngerman]{babel}

% Convenience improvements
\usepackage{csquotes}
\usepackage{enumitem}
\setlist[enumerate,1]{label={\alph*)}}
\usepackage{amsmath}
\usepackage{amssymb}
\usepackage{mathtools}
\usepackage{tabularx}

% Proper tables and centering for overfull ones
\usepackage{booktabs}
\usepackage{adjustbox}

% Change page/text dimensions, the package defaults work fine
\usepackage{geometry}

\usepackage{parskip}

% Drawings
\usepackage{tikz}
\usepackage{pgfplots}

% Adjust header and footer
\usepackage{fancyhdr}
\pagestyle{fancy}
\fancyhead[L]{Algebra --- \textbf{Exercise Sheet 10}}
\fancyhead[R]{Laurenz Weixlbaumer (11804751)}
\fancyfoot[C]{}
\fancyfoot[R]{\thepage}
% Stop fancyhdr complaints
\setlength{\headheight}{12.5pt}

\newcommand{\Deltaop}{\, \Delta\, }
\newcommand{\xor}{\, \oplus\, }
\newcommand{\id}{\text{id}}
\newcommand{\proj}{\text{proj}}

\begin{document}

\paragraph{Exercise 1}

\begin{enumerate}
    \item \begin{align*}
        \begin{pmatrix}
            0 & 1 & 0 \\
            1 & 0 & 0 \\
            0 & 0 & 1 \\
        \end{pmatrix} \cdot \begin{pmatrix}
            a_{11} & a_{12} & a_{13} \\
            a_{21} & a_{22} & a_{23} \\
            a_{31} & a_{32} & a_{33} \\
        \end{pmatrix} &= \begin{pmatrix}
            a_{21} & a_{22} & a_{23} \\
            a_{11} & a_{12} & a_{13} \\
            a_{31} & a_{32} & a_{33} \\
        \end{pmatrix}
    \end{align*}

    \item \begin{align*}
        \begin{pmatrix}
            a_{11} & a_{12} & a_{13} \\
            a_{21} & a_{22} & a_{23} \\
            a_{31} & a_{32} & a_{33} \\
        \end{pmatrix} \cdot \begin{pmatrix}
            0 & 1 & 0 \\
            1 & 0 & 0 \\
            0 & 0 & 1 \\
        \end{pmatrix} &= \begin{pmatrix}
            a_{12} & a_{11} & a_{13} \\
            a_{22} & a_{21} & a_{23} \\
            a_{32} & a_{31} & a_{33} \\
        \end{pmatrix}
    \end{align*}

    \item \begin{align*}
        \begin{pmatrix}
            1 & 0 & 0 \\
            0 & 1 & 0 \\
            0 & 0 & 2 \\
        \end{pmatrix} \cdot \begin{pmatrix}
            a_{11} & a_{12} & a_{13} \\
            a_{21} & a_{22} & a_{23} \\
            a_{31} & a_{32} & a_{33} \\
        \end{pmatrix} &= \begin{pmatrix}
            a_{11} & a_{12} & a_{13} \\
            a_{21} & a_{22} & a_{23} \\
            2a_{31} & 2a_{32} & 2a_{33} \\
        \end{pmatrix}
    \end{align*}

    \item \begin{align*}
        \begin{pmatrix}
            a_{11} & a_{12} & a_{13} \\
            a_{21} & a_{22} & a_{23} \\
            a_{31} & a_{32} & a_{33} \\
        \end{pmatrix} \cdot \begin{pmatrix}
            1 & 0 & 0 \\
            0 & 3 & 0 \\
            0 & 0 & 1 \\
        \end{pmatrix} &= \begin{pmatrix}
            a_{11} & 3a_{12} & a_{13} \\
            a_{21} & 3a_{22} & a_{23} \\
            a_{31} & 3a_{32} & a_{33} \\
        \end{pmatrix}
    \end{align*}

    \item \begin{align*}
        \begin{pmatrix}
            1 & 0 & 0 \\
            0 & 1 & 0 \\
            0 & 2 & 1 \\
        \end{pmatrix} \cdot \begin{pmatrix}
            a_{11} & a_{12} & a_{13} \\
            a_{21} & a_{22} & a_{23} \\
            a_{31} & a_{32} & a_{33} \\
        \end{pmatrix} &= \begin{pmatrix}
            a_{11} & a_{12} & a_{13} \\
            a_{21} & a_{22} & a_{23} \\
            2a_{21} + a_{31} & 2a_{212} + a_{32} & 2a_{23} + a_{33} \\
        \end{pmatrix}
    \end{align*}

    \item \begin{align*}
        \begin{pmatrix}
            a_{11} & a_{12} & a_{13} \\
            a_{21} & a_{22} & a_{23} \\
            a_{31} & a_{32} & a_{33} \\
        \end{pmatrix} \cdot \begin{pmatrix}
            1 & 0 & 0 \\
            0 & 1 & 0 \\
            0 & 2 & 1 \\
        \end{pmatrix} &= \begin{pmatrix}
            a_{11} & a_{12} + 2a_{13} & a_{13} \\
            a_{21} & a_{22} + 2a_{23} & a_{23} \\
            a_{31} & a_{32} + 2a_{33} & a_{33} \\
        \end{pmatrix}
    \end{align*}

    \item \begin{align*}
        \begin{pmatrix}
            0 & 1 & 0 \\
            0 & 0 & 1 \\
            1 & 0 & 0 \\
        \end{pmatrix} \cdot \begin{pmatrix}
            a_{11} & a_{12} & a_{13} \\
            a_{21} & a_{22} & a_{23} \\
            a_{31} & a_{32} & a_{33} \\
        \end{pmatrix} &= \begin{pmatrix}
            a_{21} & a_{22} & a_{23} \\
            a_{31} & a_{32} & a_{33} \\
            a_{11} & a_{12} & a_{13} \\
        \end{pmatrix}
    \end{align*}

    \item \begin{align*}
        \begin{pmatrix}
            a_{11} & a_{12} & a_{13} \\
            a_{21} & a_{22} & a_{23} \\
            a_{31} & a_{32} & a_{33} \\
        \end{pmatrix} \cdot \begin{pmatrix}
            0 & 1 & 0 \\
            0 & 0 & 1 \\
            1 & 0 & 0 \\
        \end{pmatrix} &= \begin{pmatrix}
            a_{13} & a_{11} & a_{12} \\
            a_{23} & a_{21} & a_{22} \\
            a_{33} & a_{31} & a_{32} \\
        \end{pmatrix}
    \end{align*}
\end{enumerate}

\paragraph{Exercise 2}

\begin{align*}
    R_{2,1}(-3) \cdot 
    \begin{pmatrix}
        -1 & 1 & -2 & -3 \\
        -3 & 0 & -6 & -12 \\
        0 & -2 & 2 & 3 \\
        0 & -6 & 4 & 11 \\
    \end{pmatrix}
    &=
    \begin{pmatrix}
        -1 & 1 & -2 & -3 \\
        0 & -3 & 0 & -3 \\
        0 & -2 & 2 & 3 \\
        0 & -6 & 4 & 11 \\
    \end{pmatrix} \\
    R_{4,2}(-2) \cdot R_{3,2}\left(-\frac{2}{3}\right) \cdot \begin{pmatrix}
        -1 & 1 & -2 & -3 \\
        0 & -3 & 0 & -3 \\
        0 & -2 & 2 & 3 \\
        0 & -6 & 4 & 11 \\
    \end{pmatrix} &= \begin{pmatrix}
        -1 & 1 & -2 & -3 \\
        0 & -3 & 0 & -3 \\
        0 & 0 & 2 & 5 \\
        0 & 0 & 4 & 17 \\
    \end{pmatrix} \\
    R_{4,3}(-2) \cdot \begin{pmatrix}
        -1 & 1 & -2 & -3 \\
        0 & -3 & 0 & -3 \\
        0 & 0 & 2 & 5 \\
        0 & 0 & 4 & 17 \\
    \end{pmatrix}
    &=
    \begin{pmatrix}
        -1 & 1 & -2 & -3 \\
        0 & -3 & 0 & -3 \\
        0 & 0 & 2 & 5 \\
        0 & 0 & 0 & 7 \\
    \end{pmatrix}
\end{align*}

\begin{align*}
    R_{4,3}(-2) \cdot R_{4,2}(-2) \cdot R_{3,2}\left(-\frac{2}{3}\right) \cdot R_{2,1}(-3) \cdot \begin{pmatrix}
        -1 & 1 & -2 & -3 \\
        -3 & 0 & -6 & -12 \\
        0 & -2 & 2 & 3 \\
        0 & -6 & 4 & 11 \\
    \end{pmatrix} &= \begin{pmatrix}
        -1 & 1 & -2 & -3 \\
        0 & -3 & 0 & -3 \\
        0 & 0 & 2 & 5 \\
        0 & 0 & 0 & 7 \\
    \end{pmatrix}
\end{align*}

\begin{align*}
    \begin{pmatrix}
        -1 & 1 & -2 & -3 \\
        -3 & 0 & -6 & -12 \\
        0 & -2 & 2 & 3 \\
        0 & -6 & 4 & 11 \\
    \end{pmatrix}
    &=
    R_{4,3}(2) \cdot 
    R_{4,2}(2) \cdot 
    R_{3,2}\left(\frac{2}{3}\right) \cdot 
    R_{2,1}(3) \cdot
    \begin{pmatrix}
        -1 & 1 & -2 & -3 \\
        0 & -3 & 0 & -3 \\
        0 & 0 & 2 & 5 \\
        0 & 0 & 0 & 7 \\
    \end{pmatrix} \\
    \begin{pmatrix}
        -1 & 1 & -2 & -3 \\
        -3 & 0 & -6 & -12 \\
        0 & -2 & 2 & 3 \\
        0 & -6 & 4 & 11 \\
    \end{pmatrix}
    &=
    \begin{pmatrix}
        1 & 0 & 0 & 0 \\
        3 & 1 & 0 & 0 \\
        0 & \frac{2}{3} & 1 & 0 \\
        0 & 2 & 2 & 1 \\
    \end{pmatrix} \cdot \begin{pmatrix}
        -1 & 1 & -2 & -3 \\
        0 & -3 & 0 & -3 \\
        0 & 0 & 2 & 5 \\
        0 & 0 & 0 & 7 \\
    \end{pmatrix}
\end{align*}

\paragraph{Exercise 3}

\begin{enumerate}
    \item \begin{align*}
        R_{4,3}(-3) \cdot
        R_{3,1}(-1) \cdot 
        \begin{pmatrix}
            1 & 0 & 0 & 2 \\
            0 & -3 & 1 & 2 \\
            1 & 0 & -1 & 1 \\
            0 & 0 & -3 & 0 \\
        \end{pmatrix} = \begin{pmatrix}
            1 & 0 & 0 & 2 \\
            0 & -3 & 1 & 2 \\
            0 & 0 & -1 & -1 \\
            0 & 0 & 0 & 3 \\
        \end{pmatrix}
    \end{align*}

    \begin{align*}
        \begin{pmatrix}
            1 & 0 & 0 & 2 \\
            0 & -3 & 1 & 2 \\
            1 & 0 & -1 & 1 \\
            0 & 0 & -3 & 0 \\
        \end{pmatrix}
        &= 
        \begin{pmatrix}
            1 & 0 & 0 & 0 \\
            0 & 1 & 0 & 0 \\
            1 & 0 & 1 & 0 \\
            0 & 0 & 3 & 1 \\
        \end{pmatrix} \cdot 
        \begin{pmatrix}
            1 & 0 & 0 & 2 \\
            0 & -3 & 1 & 2 \\
            0 & 0 & -1 & -1 \\
            0 & 0 & 0 & 3 \\
        \end{pmatrix}
    \end{align*}

    \item \begin{align*}
        A\vec{x} = \vec{0} \quad\rightarrow\quad LU(\vec{x}) = \vec{0} \quad\rightarrow\quad L(U\vec{x}) = \vec{0}
    \end{align*}
    \begin{align*}
        \begin{pmatrix}
            1 & 0 & 0 & 0 \\
            0 & 1 & 0 & 0 \\
            1 & 0 & 1 & 0 \\
            0 & 0 & 3 & 1 \\
        \end{pmatrix} \cdot 
        \begin{pmatrix}
            1 & 0 & 0 & 2 \\
            0 & -3 & 1 & 2 \\
            0 & 0 & -1 & -1 \\
            0 & 0 & 0 & 3 \\
        \end{pmatrix} \cdot \begin{pmatrix}
            x_1 \\ x_2 \\ x_3 \\ x_4 \\
        \end{pmatrix} =
        \begin{pmatrix}
            0 \\ 0 \\ 0 \\ 0 \\ 
        \end{pmatrix}
    \end{align*}

    Let
    \begin{align*}
        \begin{pmatrix}
            1 & 0 & 0 & 2 \\
            0 & -3 & 1 & 2 \\
            0 & 0 & -1 & -1 \\
            0 & 0 & 0 & 3 \\
        \end{pmatrix} \cdot \begin{pmatrix}
            x_1 \\ x_2 \\ x_3 \\ x_4 \\
        \end{pmatrix}
        =
        \begin{pmatrix}
            y_1 \\ y_2 \\ y_3 \\ y_4 \\
        \end{pmatrix}
    \end{align*}
    and now the previous system becomes
    \begin{align*}
        \begin{pmatrix}
            1 & 0 & 0 & 0 \\
            0 & 1 & 0 & 0 \\
            1 & 0 & 1 & 0 \\
            0 & 0 & 3 & 1 \\
        \end{pmatrix} \cdot 
        \begin{pmatrix}
            y_1 \\ y_2 \\ y_3 \\ y_4 \\
        \end{pmatrix} =
        \begin{pmatrix}
            0 \\ 0 \\ 0 \\ 0 \\ 
        \end{pmatrix}
    \end{align*}
    with $\vec{y} = \vec{0}$. Now we can restate the intermediate system to
    \begin{align*}
        \begin{pmatrix}
            1 & 0 & 0 & 2 \\
            0 & -3 & 1 & 2 \\
            0 & 0 & -1 & -1 \\
            0 & 0 & 0 & 3 \\
        \end{pmatrix} \cdot \begin{pmatrix}
            x_1 \\ x_2 \\ x_3 \\ x_4 \\
        \end{pmatrix}
        =
        \begin{pmatrix}
            0 \\ 0 \\ 0 \\ 0 \\ 
        \end{pmatrix}
    \end{align*}
    with $\vec{x} = \vec{0}$.

    \item \begin{align*}
        \begin{pmatrix}
            1 & 0 & 0 & 0 \\
            0 & 1 & 0 & 0 \\
            1 & 0 & 1 & 0 \\
            0 & 0 & 3 & 1 \\
        \end{pmatrix} \cdot 
        \begin{pmatrix}
            1 & 0 & 0 & 2 \\
            0 & -3 & 1 & 2 \\
            0 & 0 & -1 & -1 \\
            0 & 0 & 0 & 3 \\
        \end{pmatrix} \cdot \begin{pmatrix}
            x_1 \\ x_2 \\ x_3 \\ x_4 \\
        \end{pmatrix} =
        \begin{pmatrix}
            5 \\ -4 \\ 6 \\ 3 \\ 
        \end{pmatrix}
    \end{align*}

    Let
    \begin{align*}
        \begin{pmatrix}
            1 & 0 & 0 & 2 \\
            0 & -3 & 1 & 2 \\
            0 & 0 & -1 & -1 \\
            0 & 0 & 0 & 3 \\
        \end{pmatrix} \cdot \begin{pmatrix}
            x_1 \\ x_2 \\ x_3 \\ x_4 \\
        \end{pmatrix} = \begin{pmatrix}
            y_1 \\ y_2 \\ y_3 \\ y_4 \\
        \end{pmatrix}
    \end{align*}
    and now the previous system becomes
    \begin{align*}
        \begin{pmatrix}
            1 & 0 & 0 & 0 \\
            0 & 1 & 0 & 0 \\
            1 & 0 & 1 & 0 \\
            0 & 0 & 3 & 1 \\
        \end{pmatrix} \cdot 
        \begin{pmatrix}
            y_1 \\ y_2 \\ y_3 \\ y_4 \\
        \end{pmatrix} =
        \begin{pmatrix}
            5 \\ -4 \\ 6 \\ 3 \\ 
        \end{pmatrix}
    \end{align*}
    with $y_1 = 5$, $y_2 = -4$, $y_3 = 1$, $y_4 = 0$. Now we can restate the intermediate system to
    \begin{align*}
        \begin{pmatrix}
            1 & 0 & 0 & 2 \\
            0 & -3 & 1 & 2 \\
            0 & 0 & -1 & -1 \\
            0 & 0 & 0 & 3 \\
        \end{pmatrix} \cdot \begin{pmatrix}
            x_1 \\ x_2 \\ x_3 \\ x_4 \\
        \end{pmatrix} = \begin{pmatrix}
            5 \\ -4 \\ 1 \\ 0 \\
        \end{pmatrix}
    \end{align*}
    with $x_4 = 0$, $x_3 = -1$, $x_2 = 1$ and $x_1 = 5$.
\end{enumerate}

\paragraph{Exercise 4}

The inverse of the given matrix is

\begin{center}
    \makebox[0pt]{
    \begin{tabular}{c c c c | c c c c}
        1 & 1 & 0 & 0 & 1 & 0 & 0 & 0 \\
        0 & 1 & 0 & 1 & 0 & 1 & 0 & 0 \\
        0 & 0 & 1 & 1 & 0 & 0 & 1 & 0 \\
        1 & 0 & 0 & 0 & 0 & 0 & 0 & 1 \\
    \end{tabular} \quad $\rightarrow$ \quad
    \begin{tabular}{c c c c | c c c c}
        1 & 1 & 0 & 0 & 1 & 0 & 0 & 0 \\
        0 & 1 & 0 & 1 & 0 & 1 & 0 & 0 \\
        0 & 0 & 1 & 1 & 0 & 0 & 1 & 0 \\
        0 & -1 & 0 & 0 & -1 & 0 & 0 & 1 \\
    \end{tabular} \quad $\rightarrow$ \quad
    \begin{tabular}{c c c c | c c c c}
        1 & 1 & 0 & 0 & 1 & 0 & 0 & 0 \\
        0 & 1 & 0 & 1 & 0 & 1 & 0 & 0 \\
        0 & 0 & 1 & 1 & 0 & 0 & 1 & 0 \\
        0 & 0 & 0 & 1 & -1 & 1 & 0 & 1 \\
    \end{tabular} \quad $\rightarrow$
    }

    \makebox[0pt]{
    \begin{tabular}{c c c c | c c c c}
        1 & 1 & 0 & 0 & 1  & 0 & 0 & 0 \\
        0 & 1 & 0 & 1 & 0  & 1 & 0 & 0 \\
        0 & 0 & 1 & 1 & 0  & 0 & 1 & 0 \\
        0 & 0 & 0 & 1 & -1 & 1 & 0 & 1 \\
    \end{tabular} \quad $\rightarrow$ \quad
    \begin{tabular}{c c c c | c c c c}
        1 & 1 & 0 & 0 & 1  & 0  & 0 &  0 \\
        0 & 1 & 0 & 1 & 0  & 1  & 0 &  0 \\
        0 & 0 & 1 & 0 & 1  & -1 & 1 & -1 \\
        0 & 0 & 0 & 1 & -1 & 1  & 0 &  1 \\
    \end{tabular} \quad $\rightarrow$ \quad
    \begin{tabular}{c c c c | c c c c}
        1 & 1 & 0 & 0 & 1  & 0  & 0 &  0 \\
        0 & 1 & 0 & 0 & 1  & 0  & 0 & -1 \\
        0 & 0 & 1 & 0 & 1  & -1 & 1 & -1 \\
        0 & 0 & 0 & 1 & -1 & 1  & 0 &  1 \\
    \end{tabular} \quad $\rightarrow$
    }
\end{center}
\begin{align*}
    \begin{pmatrix}
        0 & 0 & 0 & 1 \\
        1 & 0 & 0 & -1 \\
        1 & -1 & 1 & -1 \\
        -1 & 1 & 0 & 1 \\
    \end{pmatrix}.
\end{align*}

\begin{center}
    \makebox[0pt]{
    \begin{tabular}{c c c c | c c c c}
        0 & 0 & 0 & 1 &     1 & 0 & 0 & 0 \\
        1 & 0 & 0 & -1 &    0 & 1 & 0 & 0 \\
        1 & -1 & 1 & -1 &   0 & 0 & 1 & 0 \\
        -1 & 1 & 0 & 1 &    0 & 0 & 0 & 1 \\
    \end{tabular} \quad $\rightarrow$ \quad 
    \begin{tabular}{c c c c | c c c c}
        1 & 0 & 0 & 0 &     1 & \boxed{1} & 0 & 0 \\
        1 & 0 & 0 & -1 &    0 & 1 & 0 & 0 \\
        1 & -1 & 1 & -1 &   0 & 0 & 1 & 0 \\
        -1 & 1 & 0 & 1 &    0 & 0 & 0 & 1 \\
    \end{tabular} \quad $\rightarrow$ \quad 
    \begin{tabular}{c c c c | c c c c}
        1 & 0 & 0 & 0 &     1 & 1 & 0 & 0 \\
        0 & 1 & 0 & 0 &     0 & 1 & 0 & \boxed{1} \\
        1 & -1 & 1 & -1 &   0 & 0 & 1 & 0 \\
        -1 & 1 & 0 & 1 &    0 & 0 & 0 & 1 \\
    \end{tabular} \quad $\rightarrow$ 
    }

    \makebox[0pt]{
    \begin{tabular}{c c c c | c c c c}
        1 & 0 & 0 & 0 &     1 & 1 & 0 & 0 \\
        0 & 1 & 0 & 0 &     0 & 1 & 0 & 1 \\
        0 & 0 & 1 & 0 &     0 & 0 & 1 & \boxed{1} \\
        -1 & 1 & 0 & 1 &    0 & 0 & 0 & 1 \\
    \end{tabular} \quad $\rightarrow$ \quad 
    \begin{tabular}{c c c c | c c c c}
        1 & 0 & 0 & 0 &     1 & 1 & 0 & 0 \\
        0 & 1 & 0 & 0 &     0 & 1 & 0 & 1 \\
        0 & 0 & 1 & 0 &     0 & 0 & 1 & 1 \\
        0 & 0 & 0 & 1 &     \boxed{1} & \boxed{-1} & 0 & 1 \\
    \end{tabular} 
    }
\end{center}

\begin{align*}
    A^{-1} = R_{4,2}(-1) \cdot R_{4,1}(1) \cdot R_{3,4}(1) \cdot R_{2,4}(1) \cdot R_{1,2}(1)
\end{align*}

\paragraph{Exercise 5}

\begin{enumerate}
    \item The (two) columns are linearly independent thus the rank is 2.
    \item The zero matrix has rank zero.
    \item The columns are linearly dependent, the largest set of linearly independent columns has one element, thus the rank is 1.
    \item[d) -- g)] See c).
\end{enumerate}

\paragraph{Exercise 6}

\begin{enumerate}
    \item Using only elementary transformations,
    \begin{align*}
        P_1^{-1} = T_{1,3} \cdot T_{2,3}, \quad \text{thus} \quad
        \begin{pmatrix}
            0 & 0 & 1 \\
            1 & 0 & 0 \\
            0 & 1 & 0 \\
        \end{pmatrix}
    \end{align*}

    \item The given matrix is equivalent to the permutation
    \begin{align*}
        \begin{pmatrix}
            1 & 2 & 3 \\
            2 & 3 & 1 \\
        \end{pmatrix}, \quad \text{whose inverse is} \quad
        \begin{pmatrix}
            1 & 2 & 3 \\
            3 & 1 & 2 \\
        \end{pmatrix}, \quad \text{which is equvalent to} \quad
        \begin{pmatrix}
            0 & 0 & 1 \\
            1 & 0 & 0 \\
            0 & 1 & 0 \\
        \end{pmatrix}
    \end{align*}

    \item \phantom{} \begin{center}
        \begin{tabular}{c c c c c c | c c c c c c}
            1 & 0 & 0 & 0 & 0 & 0   & 0 & 0 & 1 & 0 & 0 & 0 \\
            0 & 1 & 0 & 0 & 0 & 0   & 1 & 0 & 0 & 0 & 0 & 0 \\
            0 & 0 & 1 & 0 & 0 & 0   & 0 & 0 & 0 & 0 & 0 & 1 \\
            0 & 0 & 0 & 1 & 0 & 0   & 0 & 1 & 0 & 0 & 0 & 0 \\
            0 & 0 & 0 & 0 & 1 & 0   & 0 & 0 & 0 & 0 & 1 & 0 \\
            0 & 0 & 0 & 0 & 0 & 1   & 0 & 0 & 0 & 1 & 0 & 0 \\
        \end{tabular}
    \end{center}

    \item \phantom{} \begin{align*}
        \begin{pmatrix}
            0 & 0 & 1 & 0 & 0 & 0 \\
            1 & 0 & 0 & 0 & 0 & 0 \\
            0 & 0 & 0 & 0 & 0 & 1 \\
            0 & 1 & 0 & 0 & 0 & 0 \\
            0 & 0 & 0 & 0 & 1 & 0 \\
            0 & 0 & 0 & 1 & 0 & 0 \\
        \end{pmatrix}
    \end{align*}

    \item \begin{align*}
        (P \cdot P^T)_{i,j} &= \sum_{k = 1}^n P_{i, k} \cdot P^T_{k, j} \\
        &= \sum_{k = 1}^n P_{i, k} \cdot P_{j, k}.
    \end{align*}
    Consider that $P_{i, k}$ and $P_{j, k}$ are both in the same column. Only one element in a column may be nonzero. Thus the only way for their product to be nonzero is if $i = j$, or
    \begin{align*}
        \sum_{k = 1}^n P_{i, k} \cdot P_{j, k} = \begin{cases}
            1 & \text{if}\,\, i = j \\
            0 & \text{otherwise} \\
        \end{cases}
    \end{align*}
    This is the definition of the identity matrix.
\end{enumerate}

\end{document}
