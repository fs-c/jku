\documentclass{article}
\usepackage[utf8]{inputenc}
\usepackage[ngerman]{babel}

% Convenience improvements
\usepackage{csquotes}
\usepackage{enumitem}
\setlist[enumerate,1]{label={\alph*)}}
\usepackage{amsmath}
\usepackage{amssymb}
\usepackage{mathtools}
\usepackage{tabularx}

% Proper tables and centering for overfull ones
\usepackage{booktabs}
\usepackage{adjustbox}

% Change page/text dimensions, the package defaults work fine
\usepackage{geometry}

\usepackage{parskip}

% Drawings
\usepackage{tikz}
\usepackage{forest}

% Adjust header and footer
\usepackage{fancyhdr}
\pagestyle{fancy}
\fancyhead[L]{Multimedia --- \textbf{Aufgabe 1}}
\fancyhead[R]{Laurenz Weixlbaumer (11804751)}
\fancyfoot[C]{}
\fancyfoot[R]{\thepage}
% Stop fancyhdr complaints
\setlength{\headheight}{12.5pt}

\newcommand{\Deltaop}{\, \Delta\, }
\newcommand{\xor}{\, \oplus\, }

\begin{document}

\paragraph{Huffmann-Codierung}

\begin{enumerate}
    \item
    
    \begin{minipage}{.49\textwidth}
        \centering
        \begin{forest}
            edge/.style={edge label={node[midway,fill=white,font=\scriptsize]{#1}}}
            [100
                [40,edge=1
                    [15,edge=1
                        [5 \textbf{F},edge=1]
                        [10 \textbf{I},edge=0]
                    ]
                    [25 \textbf{L},edge=0]
                ]
                [60,edge=0
                    [25,edge=1
                        [10,edge=1
                            [5,edge=1
                                [3 \textbf{O}, edge=1]
                                [2 \textbf{V}, edge=0]
                            ]
                            [5 \textbf{G}, edge=0]
                        ]
                        [15 \textbf{R}, edge=0]
                    ]
                    [35 \textbf{E}, edge=0]                
                ]
            ]
        \end{forest}
    \end{minipage}
    \begin{minipage}{.49\textwidth}
        \centering
        \begin{tabular}{cl}
            E & 00 \\
            L & 10 \\
            R & 010\\
            I & 110\\
            F & 111\\
            G & 0110\\
            O & 01111\\
            V & 01110
        \end{tabular}
    \end{minipage}

    \item Die entsprechende Codierung ist
    \begin{center}
        \begin{tabular}{cccccccccc}
            V&I&E&L&E&R&F&O&L&G \\
            01110&110&00&10&00&010&111&01111&10&0110 \\
        \end{tabular}
    \end{center}
    Sie benötigt 31 bits an Speicher.

    \item Der Informationsgehalt beschreibt die kleinste Anzahl von Bits die benötigt werden um ein Zeichen darzustellen (für E 2, für L 10, \ldots). Die Entropie ist die Zufälligkeit der Nachricht, die im gegebenen Fall -- natürliche Sprache -- nicht besonders hoch ist (etwa im Vergleich zu einer völlig zufällig generierten Nachricht).
\end{enumerate}

\paragraph{ASCII}

\begin{enumerate}
    \item 1001100 1100001 1110101 1110010 1100101 1101110 1111010 0100000 1010111 1100101 1101001 1111000 1101100 1100010 1100001 1110101 1101101 1100101 1110010.
    \item Es wäre sonst nicht möglich verschiedene Zeichen zu unterscheiden, ASCII ist kein Präfixcode.
    \item Etwa Umlaute, ASCII ist ein amerikanischer Standard.
\end{enumerate}

\end{document}
