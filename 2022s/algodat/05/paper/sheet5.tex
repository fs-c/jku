\documentclass{article}
\usepackage[utf8]{inputenc}
\usepackage[ngerman]{babel}

% Convenience improvements
\usepackage{csquotes}
\usepackage{enumitem}
\usepackage{amsmath}
\usepackage{amssymb}
\usepackage{mathtools}
\usepackage{tabularx}

% Proper tables and centering for overfull ones
\usepackage{booktabs}
\usepackage{adjustbox}

% Change page/text dimensions, the package defaults work fine
\usepackage{geometry}

\usepackage{parskip}

\usepackage{forest}

% Drawings
\usepackage{tikz}

% Adjust header and footer
\usepackage{fancyhdr}
\pagestyle{fancy}
\fancyhead[L]{Algodat --- \textbf{Assignment 5}}
\fancyhead[R]{Laurenz Weixlbaumer (11804751)}
\fancyfoot[C]{}
\fancyfoot[R]{\thepage}
% Stop fancyhdr complaints
\setlength{\headheight}{12.5pt}

\newcommand{\Deltaop}{\, \Delta\, }
\newcommand{\xor}{\, \oplus\, }
\newcommand{\id}{\text{id}}

\begin{document}

\begin{enumerate}[label=(\alph*)]
    \item In my case, $n_1 = 118$, $n_2 = 4$, $n_3 = 75$ and $n_4 = 1$.
    \begin{center}
        \begin{tabular}{c c c c c c c c c c c c c}\toprule
            0 & 1 & 2 & 3 & 4 & 5 & 6 & 7 & 8 & 9 & 10 & 11 & 12 \\\midrule
            107 \\
            79 & 107 \\
            79 & 107 & 118 \\
            59 & 79 & 107 & 118 \\
            4 & 59 & 79 & 107 & 118 \\
            4 & 59 & 62 & 79 & 107 & 118 \\
            4 & 23 & 59 & 62 & 79 & 107 & 118 \\
            4 & 23 & 47 & 59 & 62 & 79 & 107 & 118 \\
            4 & 23 & 47 & 59 & 62 & 75 & 79 & 107 & 118 \\
            4 & 19 & 23 & 47 & 59 & 62 & 75 & 79 & 107 & 118 \\
            4 & 19 & 23 & 24 & 47 & 59 & 62 & 75 & 79 & 107 & 118 \\
            1 & 4 & 19 & 23 & 24 & 47 & 59 & 62 & 75 & 79 & 107 & 118 \\
            1 & 4 & 6 & 19 & 23 & 24 & 47 & 59 & 62 & 75 & 79 & 107 & 118 \\\bottomrule
        \end{tabular}
    \end{center}

    \item \phantom{} \begin{center}
        \begin{tabular}{c c c c c c c c c c c l}\toprule
            0 & 1 & 2 & 3 & 4 & 5 & 6 & 7 & 8 & 9 & 10 & remarks \\\midrule
            54 & 28 & 39 & 8 & 17 & 20 & 21 & 5 & 2 & 15 & 1 & \\
            1 & 28 & 39 & 8 & 17 & 20 & 21 & 5 & 2 & 15 & & replace root with the far right node on lowest level  \\
            39 & 28 & 1 & 8 & 17 & 20 & 21 & 5 & 2 & 15 & & 1st downheap, candidates were indices 1 and 2 \\
            39 & 28 & 21 & 8 & 17 & 20 & 1 & 5 & 2 & 15 & & 2nd downheap, candidates were indices 5 and 6 \\
            &&&&&&&&&&& done, no more candidates
        \end{tabular}
    \end{center}
\end{enumerate}

\end{document}
