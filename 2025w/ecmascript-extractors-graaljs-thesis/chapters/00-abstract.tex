\cleardoubleoddpage%
\begin{abstract*}
Despite many language features encouraging declarative coding styles, ECMAScript currently lacks a mechanism to achieve symmetry between object construction and destructuring---functionality that other languages like Scala or C\# offer. The ECMAScript Extractors proposal addresses this by introducing a mechanism to execute user-defined logic during assignment and binding. This thesis presents the design and experimental implementation of this Stage 1 proposal within GraalJS, a Java-based JavaScript engine. We detail the implementation effort, in particular changes to the parser and runtime nodes. Given the proposal's early stage there is no official conformance suite; therefore, we developed a custom test suite based on functional requirements derived from the draft specification. This work demonstrates the feasibility of the feature within a production engine and lays the groundwork for future work as the proposal matures.
\end{abstract*}

\cleardoubleoddpage%
\begin{otherlanguage}{ngerman}
\begin{abstract*}
Trotz einiger Sprachfeatures, die deklarative Programmierstile fördern, fehlt ECMAScript derzeit ein Mechanismus, um Symmetrie zwischen Objektkonstruktion und Destrukturierung zu erlangen---Funktionalität, die andere Sprachen wie Scala oder C\# bieten. Das ECMAScript Extractors Proposal adressiert dies durch die Einführung eines Mechanismus zur Ausführung benutzerdefinierter Logik während der Assignment- und Binding-Phasen. Diese Arbeit präsentiert das Design und die experimentelle Implementierung dieses Stage-1-Vorschlags in GraalJS, einer Java-basierten JavaScript-Engine. Wir beschreiben die Implementierung im Detail, insbesondere die Änderungen am Parser und den Laufzeit-Nodes. Da aufgrund des frühen Proposal-Stadiums keine offizielle Konformitäts-Suite existiert, entwickelten wir eine eigene Test-Suite basierend auf funktionalen Anforderungen, die aus dem Spezifikationsentwurf abgeleitet wurden. Diese Arbeit demonstriert die Umsetzbarkeit der Funktionalität innerhalb einer Produktiv-Engine und legt den Grundstein für zukünftige Arbeiten, wenn das Proposal in Stage 2 oder 3 übergeht.
\end{abstract*}
\end{otherlanguage}
