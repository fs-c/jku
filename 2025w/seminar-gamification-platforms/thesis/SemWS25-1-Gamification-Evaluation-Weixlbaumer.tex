\documentclass[runningheads]{llncs}

\PassOptionsToPackage{hidelinks}{hyperref}

\usepackage{lmodern}

\usepackage[T1]{fontenc}
\usepackage{graphicx}
\usepackage{csquotes}
\usepackage{microtype}
\usepackage[inline]{enumitem}

\usepackage{xcolor}
\usepackage{bookmark}

\usepackage{booktabs}
\usepackage{pdflscape}
\usepackage{geometry}
\usepackage{tabularx,ragged2e}
\usepackage{multirow}
\usepackage{adjustbox}
\usepackage{makecell}
\usepackage{datetime2}

\usepackage{tikz}
\usetikzlibrary{external}

\usepackage{pgfplots}
\pgfplotsset{compat=1.18}
\usepackage{pgfplotstable}
\usepgfplotslibrary{dateplot}
\usepgfplotslibrary{groupplots}

\setcounter{tocdepth}{3}

\newcommand{\reqno}[2][black]{\tikz[baseline=-0.5ex]{\draw[#1,radius=2pt] (0,0) circle ;}}%
\newcommand{\reqna}[2][black]{\tikz[baseline=-0.5ex]{\draw[#1,radius=2pt] (0,0)[densely dotted] circle ;}}%
\newcommand{\reqpartial}[2][black]{\tikz[baseline=-0.5ex, rotate=-45]{
    \begin{scope}
        \clip(-2pt, -2pt) rectangle (2pt, 0);
        \draw[#1,radius=2pt, fill=black] (0,0) circle ;
    \end{scope}
    \draw[#1,radius=2pt] (0,0) circle ;
}}%
\newcommand{\reqyes}[2][black,fill=black]{\tikz[baseline=-0.5ex]{\draw[#1,radius=2pt] (0,0) circle ;}}%

\usepackage[style=lncs, bibstyle=mylncs, backref=true, sortcites, defernumbers=true, backend=biber]{biblatex}
\addbibresource{references.bib}
\addbibresource{frameworks.bib}

\newenvironment{xcenter}
 {\par\setbox0=\hbox\bgroup\ignorespaces}
 {\unskip\egroup\noindent\makebox[\textwidth]{\box0}\par}

% this should be loaded as late as possible
\usepackage{zref-clever}

\begin{document}

% this makes sure references are included regardless of whether they are cited or not
\nocite{*}

\title{Gamification: Evaluation of Platforms}
\author{Laurenz Weixlbaumer}
\authorrunning{L. Weixlbaumer}
\institute{JKU Linz, Austria\\\email{k11804751@students.jku.at}}
\maketitle
 
\begin{abstract}
    The application of gamification to enhance user engagement is widespread, yet the supporting software framework landscape may be difficult to navigate. Technical decision-makers lack a clear comparative basis for selecting appropriate tools. This thesis addresses this issue through a systematic survey and evaluation of the current gamification framework market. We establish a comprehensive evaluation catalogue covering gamification-specific features, technical/architectural characteristics, and project health. Using these evaluation criteria, we analyze 20 open-source frameworks (sourced from GitHub) and 5 proprietary frameworks (sourced from Capterra). Our findings reveal that while foundational mechanics are universally supported, more complex ones are frequently left up to users of the respective framework, particularly with open-source projects which typically provide \enquote{toolbox} APIs, while proprietary platforms aim for complete, out-of-the-box solutions. A particular focus is the open-source ecosystem where we find diverse tech-stacks but mixed project health; while 50\% of evaluated projects appear active, most are maintained by small teams or solo developers. This work contributes a structured overview of the available frameworks, providing a resource for technical decision-making and a starting point for future research.
\keywords{Gamification platforms \and Gamification frameworks \and Gamification evaluation \and Gamification mechanics \and Gamification features \and Software evaluation}
\end{abstract}
\clearpage

\tableofcontents
\listoffigures
\listoftables
\clearpage

\section{Introduction}

Gamification, the application of game-design elements in non-game contexts, is an established method for enhancing user engagement, motivation, and learning~\cite{DeDKN11}. A landscape of software frameworks emerged to help businesses integrate gamification mechanics into their products. These frameworks promise to reduce implementation complexity over developing equivalent functionality fully in-house. Their features range from simple point systems in combination with a linear system of progression to complex rule engines with near real-time distributed processing capabilities.

This work provides a systematic survey and evaluation of the currently available gamification frameworks, with a focus on open-source solutions. We analyze frameworks to identify what features are commonly supported, what technical architectures are prevalent, and (for open-source) what the relative project health of these tools is.

To achieve this, the paper is structured as follows: \zcref[S]{sec:evaluation-criteria} first establishes a comprehensive evaluation catalogue, detailing the specific gamification features, technical characteristics, and project health metrics to be assessed.\@ \zcref[S]{sec:frameworks-under-consideration} then details the methodology used to systematically search and filter for relevant frameworks from GitHub (open-source) and Capterra (proprietary).\@ \zcref[S]{sec:evaluation} presents the process by which the individual frameworks are evaluated and the raw results.\@ \zcref[S]{sec:discussion-and-limitations} analyzes key findings and patterns emerging from the evaluation data and discusses some limitations. Finally, \zcref[S]{sec:conclusion} summarizes the contributions of this work and suggests directions for future research.

\paragraph{AI Usage Declaration} AI tools such as LLMs were used in the process of researching and writing this thesis:
\begin{enumerate}
    \item \emph{Gemini 2.5 Pro} was used for literature search, initial drafts, proof-reading and information synthesis (in particular see \zcref{sec:evaluation-process} for notes on usage in the evaluation process)
    \item \emph{Elicit} was used for literature search
    \item \emph{Cursor} was used for inline completions (primarily for \LaTeX-specific syntax) and for generating data collection scripts (in particular the data behind \zcref{fig:commits-over-time})
\end{enumerate}

\section{Evaluation Criteria\label{sec:evaluation-criteria}}

In the following, we define the evaluation criteria that will be used to evaluate the gamification frameworks. For the sake of clarity, the features have been split into categories and subcategories: First we establish a catalogue of \emph{Gamification Features} in \zcref{sec:gamification-features}, which are the functional aspects of gamification frameworks that are relevant to the drive to engage the user. For this we will use a popular \enquote{meta-framework} as a starting point and then flesh out subcategories.

Then we consider \emph{Technical and Architectural Characteristics} in \zcref{sec:technical-and-architectural-characteristics}, which are the properties that are inherent to a framework's architecture and implementation.

Finally, we also consider characteristics related to \emph{Project Health \& Ecosystem} in \zcref{sec:project-health-and-ecosystem}---properties that are related to the stability, development and ecosystem of a project.

\subsection{Gamification Features\label{sec:gamification-features}}

Many evaluations of concrete gamification implementations use the Octalysis framework~\cite{Chou15} as a basis for their evaluation~\cite{SANCN24, EDPKM15, ChrWa21, MohaB23}. At its core, Octalysis identifies eight \enquote{core drives} of human engagement that it considers to be essential for gamification:
\begin{itemize}
    \item Epic Meaning \& Calling
    \item Development \& Accomplishment
    \item Empowerment of Creativity \& Feedback
    \item Ownership \& Possession
    \item Social Influence \& Relatedness
    \item Scarcity \& Impatience
    \item Unpredictability \& Curiosity
    \item Loss \& Avoidance
\end{itemize}
Because Octalysis does not attempt to provide a comprehensive taxonomy of gamification features, but rather considers these overarching categories, we will augment the respective drives with more specific features, inspired by the literature but also by existing solutions.

In particular, while the original grouping is arranged around factors concerned with human motivation, we will instead group features by their fundamental nature and conceptual similarity. It should be noted that other evaluations of more abstract frameworks for gamification consider e.g.~such things as economic viability~\cite{GearB13} or consideration for stakeholders~\cite{HeAWS15} as \enquote{features} of gamification frameworks, while this work focuses only on features that a concrete implementation of a gamification framework could provide or directly enable (cf.~\emph{functional} vs.~\emph{non-functional} requirements~\cite{ISOReq}, where we would consider only the former).

In the following, for each of the eight drives, alongside a brief definition we will thus list functional gamification features that are relevant to the drive. Consider that this could never be an exhaustive list, inasmuch as the space of \enquote{game-like} features is vast, but rather it should serve as a representative sample based on the literature (as noted individually) and existing solutions (see \zcref{sec:frameworks-under-consideration}).

\subsubsection{Epic Meaning \& Calling\label{sec:epic-meaning-calling}} This encompasses the drive to feel like being a part of something larger than oneself, having a meaningful impact on some greater goal~\cite{Chou15}. Providing such a \enquote{narrative} around the gamified elements lies upon the implementor, but the framework may still provide supporting tools in this regard. Stretching the original definition, we will also more broadly include other long-term incentives in this category.

Features in this category are concerned with establishing first an initial sense of purpose and direction, and then a sustainable \enquote{progression loop} that keeps the user engaged.

\paragraph{Onboarding \& Tutorials} serve as an early guide for new users, helping them to understand the gamified elements and the overall purpose of the platform~\cite{Herz14, ZichC11, APLLAC24}.

\paragraph{Narratives} provide a story or a setting for the gamified elements, helping to create a sense of immersion, and are designed to foster engagement~\cite{MRGA15}. In a framework, this might look like providing tooling to create/visualize a roadmap that a user can progress through~\cite{SaHMM17}.

\paragraph{Quests} are a structured way to earn \emph{Points} (see \zcref{sec:development-accomplishment}) after completing some given tasks~\cite{ZichC11, APLLAC24}. They are a key component of the \enquote{progression loop} and generally an abstraction over whatever the gamified experience is trying to achieve.

\subsubsection{Development \& Accomplishment\label{sec:development-accomplishment}} The drive to make visible progress towards a goal, to develop skills and to overcome challenges~\cite{Chou15}.

Frameworks support this drive by providing feedback on progression and visible representations of achievement, as such providing \emph{visible proof}\footnote{Metaphorically speaking; many actual frameworks do not provide a user interface at all and it is up to the implementor to actually make these things visible. But we consider these features supported if the framework provides those means.} of progress---both to the user and to others.

\paragraph{Points} (or \emph{XP}) are the fundamental unit of progression in many descriptions of gamification~\cite{DeDKN11, SaHMM17, XiZIA18, ZichC11, Yongw15, APLLAC24, BozHS24, HeAWS15}. They are generally combined with \emph{Levels} to structure the progression and enhance the motivational aspects~\cite{SaHMM17}.

\paragraph{Levels} increase players' competitive instinct by providing a clear hierarchy of progression~\cite{DeDKN11, Yongw15, APLLAC24, BozHS24}. They are generally based on point thresholds (see \emph{Points}).~\cite{Yongw15} identifies different \enquote{player personality archetypes} and finds that of all identified game mechanics, status-enabling mechanics like player levels are the only ones to specifically motivate all kinds.

\paragraph{Badges \& Trophies} (or \emph{Achievements}, \emph{Milestones}) are orthogonal to the points and levels system---they provide a way to reward players for extraordinary achievements outside of the regular \enquote{progression loop}~\cite{SaHMM17, ZichC11, Yongw15, APLLAC24, BozHS24}. Awarding new players comparatively easy badges early on is a common way to bolster early engagement and foster retention~\cite{CoGS19}.

\paragraph{Leaderboards} rank players based on their relative \enquote{performance}, be it \emph{Points} or some other metric~\cite{ZichC11, APLLAC24, BozHS24}. They can be effective motivators to some players~\cite{HaKS14}, however the general \enquote{motivational potential} of leaderboards is mixed~\cite{SaHMM17}.

\paragraph{Rewards} are coupled with completion of \emph{Quests}, on obtaining \emph{Badges}~\cite{Yongw15} or otherwise as a result of some desired action~\cite{XiZIA18, APLLAC24}. They are particularly relevant for customer-facing gamification systems~\cite{XiZIA18}. While some authors make a distinction between \enquote{in-game} and \enquote{real-world} rewards~\cite{APLLAC24}, we will not do so here, in particular since the majority of examined frameworks do not support this distinction.

\subsubsection{Empowerment of Creativity \& Feedback\label{sec:creativity-and-feedback}} The drive to be creative, to experiment, being able to immediately see the result of one's actions~\cite{Chou15}.

Features in this category are concerned with providing the user with a sense of agency and control over their own progress and experience.

\paragraph{Customizable Avatars} allow users to personalize their own visual representation~\cite{SaHMM17, APLLAC24, BozHS24}. It is important to delineate avatars from \emph{Profiles} (see \zcref{sec:ownership-and-possession}): While they may be presented together, the primary purpose of avatars is self-expression while profiles are primarily concerned with information display.

\paragraph{\enquote{Choose Your Path} Quests} give users multiple paths to achieve a goal, giving them a sense of agency and control~\cite{CoGS19, CoGS19a}. Note that we will, in our evaluation, not limit this to only \enquote{Quests} but more generally progress: If all progress is linear along a single path (regardless of the concept of progression), the framework is not considered to support this.

\paragraph{Notifications}~provide real-time feedback to users about their actions (and the consequences thereof) inside the platform~\cite{SaHMM17}, for example through push-notifications or pop-ups. This closes the feedback loop and reinforces the user's sense of agency~\cite{CoGS19}.

\subsubsection{Ownership \& Possession\label{sec:ownership-and-possession}} The drive to own, control and accumulate things~\cite{Chou15}.

Features in this category provide the means for the user to own e.g.~items and (virtual) currency within the bounds of the platform.

\paragraph{Collectibles} are virtual items~\cite{HeAWS15}, generally with no direct \enquote{functional value}, that users can earn, collect and display~\cite{DeDKN11} (although, for the sake of this evaluation, simply the concept of this kind of virtual item is sufficient). Their value is driven by a \enquote{collector's instinct}, an important part of this drive.

\paragraph{Virtual Currency} is any framework-internal currency (e.g.~\enquote{Gems}, \enquote{Coins}) earned as rewards for completing \emph{Quests}, obtaining \emph{Badges} or otherwise through some activity~\cite{HeAWS15}. It is generally used to purchase \emph{Collectibles} or as a means to personalize the \emph{Avatar} (see \zcref{sec:creativity-and-feedback}) but may in some usage contexts also be exchanged for real-world currency or other benefits.

\subsubsection{Social Influence \& Relatedness\label{sec:social-influence-and-relatedness}} The drive for social interaction and connection, including e.g.~companionship, competition, cooperation and communication~\cite{Chou15}. It is concerned with the fundamental human need for social connection, interaction and validation.

Framework mechanics can leverage this drive by providing opportunities for social interaction and fostering social connections.

\paragraph{Profiles} are a dedicated space representing a user's identity within the platform, generally containing information about the user's \emph{Roles}, \emph{Badges \& Trophies} and \emph{Level} (see \zcref{sec:development-accomplishment}), and other relevant details~\cite{ZichC11}. 

\paragraph{Roles} represent specific labels or permissions assigned to users, often based on their contributions, status, or seniority~\cite{HeAWS15}. Depending on usage context roles can facilitate or provide a (social) hierarchy or be used for access control and permissions management.

\paragraph{Social Network} is the \enquote{umbrella category} for all kinds of social features allowing users to form and foster connections with one another, such as the concept of \emph{Friends}, \emph{Groups} or \emph{Followers}~\cite{ZichC11, Herz14, Yongw15}. This facilitates direct social interactions like cooperation and communication on-platform, giving the chance to integrate it into the gamified experience (e.g.~by awarding rewards for cooperative interactions).

\paragraph{Widgets} are embeddable components (e.g.~iframe or script reference) that allow users to display their profile, achievements, or status outside of the primary platform (e.g., on a personal blog, forum signature, or company intranet page)~\cite{ZichC11}. This has the potential to extend the social aspects of the gamified experience beyond the primary platform.

\subsubsection{Scarcity \& Impatience\label{sec:scarcity-and-impatience}} The drive to value and prioritize things simply because they are rare or difficult to obtain, or because their availability is limited in some other way~\cite{Chou15}. This drive encompasses e.g. the well-known \enquote{fear of missing out} (FOMO) effect.

Frameworks can allow implementors to leverage this by making opportunities and rewards limited in time, quantity or otherwise.

\paragraph{Events} are rare, time-limited and generally cyclic opportunities to gain special rewards, often by participating in an exclusive activity tied to the event~\cite{Herz14, HeAWS15}.

\paragraph{Time-gates} on tasks or other activities are a constraint that requires a task to be completed within a maximum allowed time~\cite{Yongw15, DeDKN11}. This creates pressure and can make the successful completion of the task feel more rewarding.

\paragraph{Deadlines} on tasks or other activities are a constraint that requires a task to be initiated or completed by a specific point in time, which is generally useful to model real-life time constraints in the gamified context~\cite{Yongw15, DeDKN11}.

\subsubsection{Unpredictability \& Curiosity\label{sec:unpredictability-and-curiosity}} The drive to explore the unknown, being motivated by the desire to find out what happens next~\cite{Chou15}. This encompasses the \enquote{Skinner box} effect, where variable or random rewards are often more engaging than predictable ones.

Features in this category are concerned with enabling random reward behavior in the gamified experience.

\paragraph{Variable Rewards} broadly represent any kind of non-deterministic outcomes or rewards for actions, a classic example of tapping into the Skinner box effect~\cite{HeAWS15}.


\paragraph{Gambling} is a super-category for any kind of games of chance supported by frameworks, generally requiring the user to stake something of value (e.g.~\emph{Points} or \emph{Virtual currency}, see \zcref{sec:development-accomplishment} and \zcref{sec:ownership-and-possession} respectively) in order to participate~\cite{Yongw15}. This distinguishes it from \emph{Variable rewards} in that the user stakes a resource other than their own time (in order to complete the associated task).

\subsubsection{Loss \& Avoidance\label{sec:loss-and-avoidance}} The drive to avoid negative consequences as a result of one's actions or inactions~\cite{Chou15}. It is based on the psychological principle of \enquote{loss aversion}, where the pain of losing something may be a more powerful motivator than the pleasure of gaining an equivalent reward.

Features in this category are concerned with enabling this kind of \enquote{negative motivation} as part of the engagement mechanisms.

\paragraph{Streaks} are a mechanic that rewards users for performing a desired action (e.g.~\enquote{logging in}) on a consecutive basis (daily, weekly)~\cite{HuynI17, Rapp15}. They set up a loss-avoidance incentive (\enquote{sunk-cost}) that enhances user retention. However, they may also have the opposite effect: Once users lose a streak they are much more likely to abandon the activity altogether~\cite{HuynI17}.

\paragraph{Punishments} are negative consequences for failures (e.g.~for failed tasks) or inactions (e.g.~for missed deadlines or time-gates)~\cite{SaHMM17, XiZIA18}. We consider this feature to be supported if there are mechanics that motivate users through the drive to avoid a penalty.

\subsection{Technical and Architectural Characteristics\label{sec:technical-and-architectural-characteristics}}

This section evaluates the properties that are inherent to the frameworks' architecture and implementation. These characteristics may impose technical constraints on the implementor and affect the integration and development effort.

\subsubsection{Integration\label{sec:integration}}
Evaluates the process and complexity required for an implementor to get the framework operational as well as the method of logically connecting it to the parent application. The evaluated elements concern the initial setup and the architectural/infrastructure footprint of the framework, i.e.~the specific integration points and the level of coupling they require.

\paragraph{Dependency}
Is the framework a library intended to be included directly in the application's codebase (e.g., via a package manager like npm, pip, or Maven)? This approach typically means the framework code runs within the same context as the parent application, generally even within the same process. We consider this to be partially supported if e.g.~the framework provides client dependencies to interact with an API (= separate deployment) or if it's primarily meant to be hosted separately but additionally provides all necessary functionality as exports.

\paragraph{Separate Deployment}
Alternatively, we evaluate if the framework is designed to run as a standalone service (e.g., a containerized microservice). This approach offers logical decoupling (integration code may use a completely different technology stack) and scalability but introduces infrastructural overhead (separate hosting, networking, and monitoring). 

\paragraph{Platform-specific (Plugin)}
Is the framework a plugin for a \enquote{parent} system (e.g., a WordPress installation or Shopify app)? We delineate this from the \emph{Dependency} pattern by limiting this pattern to cases where there is a clear parent framework with an opinionated way of integrating plugins and requiring that the framework support drop-in integration without any additional glue code.

\subsubsection{Technology Stack Requirements\label{sec:technology-stack-requirements}}
This category details the specific software stack required to integrate the framework.

\paragraph{Programming Language}
For this point, the evaluation scheme differs based on the integration pattern (see \zcref{sec:integration}): For the case of direct integration, this is the language that the framework is built in. For the remaining cases we will consider this point to be not applicable, unless the framework is clearly designed to be used with some particular language (e.g.~PHP for a WordPress plugin, or if the documentation otherwise mentions it).

\paragraph{Persistence}
The persistence layers that are supported. This will generally be a list of supported databases (e.g.~MySQL, Postgres, Redis, \ldots) but may also be not applicable if the framework does not support any persistence at all, or very broad in case \emph{adapters} or e.g.\emph{ORM} are supported.

\paragraph{Framework or Platform}
Any hard dependencies on other frameworks or runtimes, such as Node.js, Spring Boot, or a specific web server. A WordPress plugin would for example be considered to have a hard dependency on WordPress.

\subsubsection{Internationalization (i18n) Capabilities\label{sec:internationalization-i18n-capabilities}}
Examines the framework's ability to support different languages and regions for any user-facing content~\cite{AhaPP04}. This is concerned with both out-of-the-box support for locales and the ease of adding new translations.

\paragraph{Supported Languages}
The set of out-of-the-box translations provided for any user-facing content, such as built-in badge descriptions or notification messages. We consider this to be not applicable if the framework does not have any initial user-facing strings by default (but note that \emph{Extensibility} may still be relevant in this case).

\paragraph{Extensibility}
Whether the framework supports the addition of new locales and other localization concerns like RTL text or region-specific formats. This is considered not applicable if there is a required \enquote{super-framework} (e.g.~WordPress, Laravel, \ldots) which already has an established way of handling localization.

\subsubsection{Provided Interfaces\label{sec:provided-interfaces}}
This evaluates the built-in interfaces provided for different user roles, such as end-users or administrators. The elements in this category focus on what is provided out-of-the-box.

\paragraph{User Interface}
Does the framework provide pre-built UI widgets (e.g., as web components, React components, or server-rendered templates) or does it only provide a data API, leaving all UI development to the implementor?

\paragraph{Admin Interface}
We evaluate the existence and completeness of an administrative dashboard for non-technical users. The minimum requirement is that the administrative view enables viewing and modifying user data, we consider it fully supported if it also supports manipulating the gamification mechanics themselves (e.g.~creating new badges or events, \ldots).

\subsubsection{3rd Party Integration Capabilities\label{sec:3rd-party-integration-capabilities}}
This category assesses the framework's out-of-the-box capabilities to connect with external, third-party services for user authentication and management. This focuses on common integrations for identity and authentication.

\paragraph{Social}
Considers identity providers that users might have personal accounts with, such as Google, Facebook, or Microsoft.

\paragraph{Enterprise}
Considers support for single sign-on (SSO) with identity providers that are provided as part of company infrastructure, such as SAML or LDAP.\@

\subsection{Project Health and Ecosystem\label{sec:project-health-and-ecosystem}}

This section evaluates the external attributes related to the project's development, community, and long-term viability. These factors indicate the project's stability, maturity, and the level of support an implementor can expect.

\subsubsection{Development Activity \& Stability\label{sec:development-activity-and-stability}}
This category uses quantitative measures from the project's source code repository to estimate its maintenance status, momentum, and risk of abandonment~\cite{AhaPP04, MattB99}

\paragraph{Commits over Time} are the primary indicator of development activity and can reveal trends.

\paragraph{Backers \& Project Ownership}
We identify the primary maintainers or organizations (e.g., private company, university, government) behind the project. This helps assess the \enquote{bus factor} and the project's long-term strategic interests.

\paragraph{Number of Forks} is an indicator for project usage: Users might create a fork to make minor modifications or bugfixes for upstreaming, which implies they use the framework in their own projects~\cite{YaKKN20}. We chose to use number of forks as opposed to number of stars on GitHub since stars seem more likely to be used as a form of \enquote{bookmark} or reminder, and thus less suitable as a proxy for real-world usage.

\paragraph{Number of Contributors} is similar to \emph{Backers \& Project Ownership} in that it's a metric to assess the \enquote{bus factor}. Projects with a smaller number of contributors are more likely to have more open issues and are at higher risk of abandonment~\cite{YaKKN20}.

\subsubsection{Licensing\label{sec:licensing}}
This category evaluates the legal terms under which the framework can be used, which is a critical-path check for any adoption. We are concerned with both the license type and its implications.

\paragraph{License Type}
The specific license for the framework (e.g., MIT, Apache 2.0, GPL, AGPL). Note that frameworks may be available under multiple licenses for a multitude of reasons, in which case we will consider the most restrictive one, particularly in terms of \emph{Commercial Viability}.

\paragraph{Commercial Viability}
is often the key implication of a software license, particularly the distinction between permissive licenses (e.g., MIT) which allow use in closed-source commercial projects, and copyleft licenses (e.g., AGPL) which may require the parent application to also be open-sourced. We will consider this to be fully supported if the license allows \emph{use} in closed-source commercial projects, even if it would otherwise require \emph{modifications} to be open-sourced (as in the GPLv3 for example).

\subsubsection{Community \& Support\label{sec:community-and-support}}

\paragraph{Community} Are there active forums, a Discord, mailing list, \ldots? GitHub issues and pull requests are a frequent place of discussion in open-source projects, but if they are the only channels of communication we will consider this to be only partially fulfilled.

\paragraph{Commercial} Does the framework have an entity offering commercial support or for example managed hosting in case it's a separate deployment?

\section{Frameworks under Consideration\label{sec:frameworks-under-consideration}}

This section details the frameworks evaluated in this thesis, as well as the methodology by which they were selected. Since the processes are different, we first describe the selection process for open-source frameworks in \zcref{sec:frameworks-under-consideration-open-source}. This involved a systematic search and filter process based on data from GitHub. Then we describe the selection process for proprietary frameworks in \zcref{sec:frameworks-under-consideration-proprietary}, which used a similar process based on data from the Capterra software comparison platform.

\subsection{Open-Source\label{sec:frameworks-under-consideration-open-source}}

We selected GitHub as the sole data source for open-source work due to its position as the de-facto standard for open-source software development. Its scale and central role in the open-source developer community~\cite{github25} make it the most representative source among alternatives like GitLab or SourceForge.

The collection process was divided into two stages, \emph{Searching} using an API with some initial criteria and manual \emph{Filtering} through individually inspecting the results.

\paragraph{Search} The GitHub Repository Search API\footnote{\url{https://docs.github.com/en/search-github/searching-on-github/searching-for-repositories}, accessed 11/04/2025} was used to search for repositories with the following initial criteria:
\begin{itemize}
    \item Name, description or tags must include \enquote{gamification}. This notably does not include the \enquote{readme} file, which is customary for most repositories and would also be available for search. We found that when including it, the search results were dominated by projects that were not actually frameworks, but rather tools or libraries that happened to off-handedly mention gamification as part of their broader scope.
    \item Must be public and not archived by the owner. Not considering archived repositories invites a minor contradiction, since we otherwise included obviously abandoned repositories. However we took the archival status as an explicit sign from the owner that the repository should no longer be considered for usage, as opposed to simply being (perhaps temporarily) inactive.
    \item Must have been updated in the last 10 years\footnote{Note that GitHub has somewhat counter-intuitive behavior in this regard: A repository's \enquote{pushed date} also includes pushes on non-default branches. In particular this may include branches that are created and pushed to by automated processes (e.g.~dependency update bots) long after a repository was already abandoned. Repositories that were included only due to obviously automated activity were filtered out by hand.}. In modern software engineering, dependencies that have remained unmaintained for extended periods are generally considered to be a security risk~\cite{owasp21} and should be avoided. Thus, if we wished to limit ourselves to frameworks that should still be reasonably used in practice, a tighter limit would be more appropriate. However, we also wanted to consider the historical development of the space, and thus chose the 10-year limit as a compromise.
    \item The 5-star limit was arbitrarily chosen to limit the number of results to a manageable number.
\end{itemize}

These criteria can be expressed as the following search query:
\url{https://api.github.com/search/repositories?q=gamification+is:public+archived:false+pushed:>2015+stars:>5}\footnote{Or alternatively, \url{https://github.com/search?q=gamification+is:public+archived:false+pushed:>2015+stars:>5} to use the web interface. But this opens up the possibility of cookies and other tracking interfering with the results and wasn't used.}. Executing this query yielded 191 results as of 11/04/2025.

\paragraph{Filter} The obtained repositories were then manually screened and filtered according to the following criteria:
\begin{itemize}
    \item Discarding projects that are overly domain-specific, e.g.~apps that are concerned with gamifying a particular activity and thus have the relevant tags, but are not a framework. (This filtered out 139 projects.)
    \item Discarding projects that are overly \enquote{simple} or focused. This included projects that explicitly focus only on a singular feature, e.g.~adding badges to some forum software. These may be part of a larger gamification effort, and may be extended or combined with other software to form a framework, but do not qualify as one on their own according to our definition. (This filtered out 19 projects.)
    \item Discarding projects that have no documentation, in particular when projects have links to documentation on external websites that are no longer accessible. (This filtered out 12 projects.)
    \item Choosing the more recently updated fork if applicable. (This filtered out one project.)
\end{itemize}
This process filtered out 171 projects in total.

\paragraph{Results} This leaves us with 20 open-source frameworks for further evaluation, listed in \zcref{tab:open-source-frameworks}.

\begin{table}[h]
    \centering
    \caption{Open-source frameworks under consideration, as per \zcref{sec:frameworks-under-consideration-open-source}.\label{tab:open-source-frameworks}}
    \begin{tabular}{@{} r l l @{}}
        \toprule
        Ref. & Short Name & URL (prepend \url{https://github.com/}) \\
        \midrule
        \cite{ActiDoo} & Gamification Engine & \href{https://github.com/ActiDoo/gamification-engine}{\url{ActiDoo/gamification-engine}} \\
        \cite{SieteValles} & Siete Valles & \href{https://github.com/jorgegorka/siete-valles}{\url{jorgegorka/siete-valles}} \\
        \cite{UserInfuser} & UserInfuser & \href{https://github.com/nlake44/UserInfuser}{\url{nlake44/UserInfuser}} \\
        \cite{GamificationServer} & Gamification-Server & \href{https://github.com/devleague/Gamification-Server}{\url{devleague/Gamification-Server}} \\
        \cite{DjangoGamification} & django-gamification & \href{https://github.com/mattjegan/django-gamification}{\url{mattjegan/django-gamification}} \\
        \cite{FlarumGamification} & gamification & \href{https://github.com/FriendsOfFlarum/gamification}{\url{FriendsOfFlarum/gamification}} \\
        \cite{Gioco} & gioco & \href{https://github.com/joaomdmoura/gioco/tree/1.1.1}{\url{joaomdmoura/gioco/tree/1.1.1}} \\
        \cite{HyperosloGamification} & gamification & \href{https://github.com/hyperoslo/gamification}{\url{hyperoslo/gamification}} \\
        \cite{MoodleBlockXP} & moodle-block\_xp & \href{https://github.com/FMCorz/moodle-block_xp}{\url{FMCorz/moodle-block\_xp}} \\
        \cite{ScoreJS} & score.js & \href{https://github.com/mulhoon/score.js}{\url{mulhoon/score.js}} \\
        \cite{LaravelGamify} & laravel-gamify & \href{https://github.com/qcod/laravel-gamify}{\url{qcod/laravel-gamify}} \\
        \cite{Yay} & yay & \href{https://github.com/sveneisenschmidt/yay}{\url{sveneisenschmidt/yay}} \\
        \cite{Kinben} & Kinben & \href{https://github.com/InteractiveSystemsGroup/GamificationEngine-Kinben}{\url{InteractiveSystemsGroup/GamificationEngine-Kinben}} \\
        \cite{Oasis} & oasis & \href{https://github.com/isuru89/oasis}{\url{isuru89/oasis}} \\
        \cite{LevelUp} & level-up & \href{https://github.com/cjmellor/level-up}{\url{cjmellor/level-up}} \\
        \cite{Gamify} & Gamify & \href{https://github.com/GollaBharath/Gamify}{\url{GollaBharath/Gamify}} \\
        \cite{GamifyLaravel} & gamify-laravel & \href{https://github.com/pacoorozco/gamify-laravel}{\url{pacoorozco/gamify-laravel}} \\
        \cite{Honor} & honor & \href{https://github.com/jrmyward/honor}{\url{jrmyward/honor}} \\
        \cite{PhpGamify} & php-gamify & \href{https://github.com/mediasoftpro/php-gamify}{\url{mediasoftpro/php-gamify}} \\
        \cite{AcisGamificationFramework} & Acis-Gamification & \href{https://github.com/rwth-acis/Gamification-Framework}{\url{rwth-acis/Gamification-Framework}} \\
        \bottomrule
    \end{tabular}
\end{table}

\subsection{Proprietary\label{sec:frameworks-under-consideration-proprietary}}

We have limited our evaluation to the Capterra software comparison platform, as it is a leading source for software reviews and comparisons~\cite{AlADM21}. We especially did not want to consider older analyses of the market since the industry appears to move fast: Of 10 commercial gamification solutions treated in one of the literature reviews under consideration~\cite{HeAWS15}\footnote{Published 10 years ago at the time of writing. Looking only at \enquote{Generic Gamification Platforms} and \enquote{Integrated Solutions}.}, only two were still accessible under the given URL\@.

In the following, the search and filter process is described in more detail.

\paragraph{Search} Capterra's \enquote{Gamification} category\footnote{\url{https://www.capterra.com/gamification-software/}, accessed 11/06/2025} was used to obtain a list of software products that were classified as containing gamification features. This list contained 210 software offerings in total.

\paragraph{Filter} Since the primary focus of this thesis is on open-source frameworks, we did not evaluate these products as in-depth as we did for the open-source frameworks (\zcref{sec:frameworks-under-consideration-open-source}). In particular, we chose to only consider the first 5 products in the list. Notably we elected to leave the default \enquote{Sponsored} sorting option in place, where vendors bid for placement in this particular category. We found this to be the strongest filter for vendors that were primarily concerned with providing gamification, as opposed to those who listed it as a minor feature but were still included in the category.

\paragraph{Results} This leaves us with 5 proprietary frameworks for further evaluation, listed in \zcref{tab:proprietary-frameworks}.

\begin{table}[h]
    \centering
    \caption{Proprietary frameworks under consideration, as per \zcref{sec:frameworks-under-consideration-proprietary}.\label{tab:proprietary-frameworks}}
    \begin{tabular}{@{} r l l @{}}
        \toprule
        Ref. & Short Name & URL \\\midrule
        \cite{OvosPlay} & OvosPlay & \url{https://www.capterra.com/ovosplay/} \\
        \cite{TalonOne} & Talon One & \url{https://www.capterra.com/talonone/}\\
        \cite{SoliticsGamification} & Solitics Gamification & \url{https://www.capterra.com/solitics-gamification/} \\
        \cite{Staffino} & Staffino & \url{https://www.capterra.com/staffino/} \\
        \cite{DotVu} & DotVu & \url{https://www.capterra.com/dotvu/} \\
        \bottomrule
    \end{tabular}
\end{table}

\section{Evaluation\label{sec:evaluation}}

\subsection{Evaluation Process\label{sec:evaluation-process}}

The individual evaluation was carried out using two primary methods:
\begin{itemize}
    \item For open-source frameworks, manually inspecting the repository's documentation (generally readme, repository wiki or documentation website) and spot-checking source code (in particular data models and example code) to assess individual features. For proprietary frameworks, publicly available marketing material was considered.
    \item Feeding the respective primary URL into Gemini 2.5 Pro alongside the evaluation criteria. The LLM served as a valuable \enquote{second pair of eyes}, being able to dive deeper into the details more quickly than manual (human) inspection would allow. This was particularly useful for the often large and convoluted corporate websites---companies were found to be reluctant to even provide access to their frameworks' documentation without an active account.
\end{itemize}
Note in particular that it was infeasible to manually inspect the source code of open-source frameworks in significant detail given the time constraints for this thesis. For the same reason, no attempt was made to set up any of the frameworks for hands-on evaluation. In the same vein, no attempt was made to create \enquote{demo accounts} for proprietary frameworks.

Additionally, both the Technical and Architectural Characteristics (\zcref{sec:technical-and-architectural-characteristics}) and the Project Health and Ecosystem (\zcref{sec:project-health-and-ecosystem}) were primarily established with the intent of evaluating open-source frameworks, and were found to be unsuitable for evaluating proprietary frameworks. The kind of information that would be required to evaluate these characteristics, e.g.~the internally used tech stack, is not generally available or even relevant for end-users since modification is impossible anyway. For these reasons, proprietary frameworks were only evaluated according to the primary focus of this thesis, the gamification features (\zcref{sec:gamification-features}).

For proprietary frameworks in particular it was frequently challenging to conclusively determine whether a feature was supported or not. We tended towards the most \enquote{reasonable} extrapolation of the available information in these cases---e.g.~if marketing material extensively mentioned social features we assumed that at least some form of profile support was present as well, even if no explicit mention was made.

\subsection{Evaluation Results\label{sec:evaluation-results}}

For the \enquote{raw} evaluation results collected according to \zcref{sec:evaluation-process}, see:
\begin{itemize}
    \item \zcref[S]{tab:gamification-features-eval} for the gamification features as described in \zcref{sec:gamification-features}.
    \item \zcref[S]{tab:technical-and-architectural-characteristics-eval} for the technical and architectural characteristics as described in \zcref{sec:technical-and-architectural-characteristics} (excluding proprietary frameworks). As part of this evaluation, \zcref{fig:commits-over-time} was created to visualize the development activity of these frameworks over time.
\end{itemize}
A \enquote{ranking} of gamification features by overall support is given in \zcref{tab:gamification-features-ranking}.

\begin{table}[htb]
    \centering
    \caption{Ranking of gamification features by overall support in frameworks, assigning a score of 0 for \reqno{}, 1 for \reqpartial{} and 2 for \reqyes{}, showing top and bottom 5.\label{tab:gamification-features-ranking}}
    \begin{tabular}{@{} l c @{}}
        \toprule
        Feature & Score \\\midrule
        Points & 48 \\
        Badges & 39 \\
        Notifications & 39 \\
        Levels & 34 \\
        Leaderboards & 31 \\
        \makecell[c]{\vdots} & \vdots \\
        \enquote{Choose your Path} Quests & 6 \\
        Variable Rewards & 6 \\
        Widgets & 5 \\
        Gambling & 4 \\
        Customizable Avatar & 0 \\
        \bottomrule
    \end{tabular}
\end{table}

\section{Discussion \& Limitations\label{sec:discussion-and-limitations}}

\subsection{Analysis of Findings\label{sec:analysis-of-findings}}

This section analyzes the findings from \zcref{sec:evaluation-results}, highlights some key patterns emerging from the evaluation, and finally discusses some limitations discovered throughout the study.

\subsubsection{Gamification Features}

The evaluation confirms that the fundamental gamification \enquote{PBL-triad} of \emph{Points}, \emph{Badges}, and \emph{Leaderboards}~\cite{APLLAC24} is well-supported. \emph{Points} and \emph{Badges} are nearly ubiquitous. \emph{Leaderboards} were observed less frequently; however, this may be an evaluation artifact: Most evaluated open-source frameworks are headless APIs and almost all provide point-querying mechanisms, which means an implementor can create leaderboards with minimal effort. Framework developers may thus not consider explicit support a necessary or desirable feature.

\cite{ZichC11} identify seven primary mechanics of gamification, extending the previously mentioned PBL triad: \enquote{points, levels, leaderboards, badges, challenges/quests, onboarding, and engagement loops}. The first four mechanics, along with quests, are well-supported. However, mechanics such as \enquote{onboarding} (our \emph{Tutorials}) and \enquote{engagement loops} (our \emph{Narratives} and \emph{Streaks}) are poorly represented, especially in open-source solutions. Proprietary frameworks had stronger support for these, suggesting an increased focus on providing complete, out-of-the-box solutions as opposed to the more \enquote{toolbox-style} approach of many open-source frameworks.

Several features which were commonly referenced in the literature were notably absent. No framework, open-source or proprietary, was found to support \emph{Customizable Avatars}. We take this not as an indication that this isn't an important gamification mechanic, but that frameworks consider it out-of-scope and leave it up to the implementor. Similarly, \emph{Variable Rewards} and \emph{Gambling} were found to be absent from all open-source frameworks. Given their high engagement potential~\cite{SaHMM17}, this absence also appears to be a deliberate decision to limit the scope.

Some feature correlations were (informally) observed:
\begin{itemize}
    \item \emph{Points and Levels}: Support for \emph{Levels} consistently implied support for \emph{Points}, since all frameworks used the latter as thresholds for progression.
    \item \emph{Deadlines and Streaks}: These features were frequently paired together. We take this to be due to a technical overlap: Implementing streaks necessarily requires a concept of time-based deadlines.
\end{itemize}

\subsubsection{Technical and Architectural Characteristics}

The evaluation of the 20 open-source frameworks revealed a roughly even distribution between two primary integration patterns, direct \emph{Dependencies} and separate \emph{Deployments}, with some overlap. Only two frameworks considered were \emph{Plugins}.

A diverse mix of programming languages and underlying frameworks was observed, with predictable correlations (e.g., Ruby frameworks utilizing Rails; PHP frameworks often building on Laravel).

Frameworks were roughly split between providing only a headless interface or having some form of UI, although many frameworks with a UI were limited to partial implementations providing only some fragments to be embedded by the implementor. Dedicated Admin Interfaces are rare, most Admin Interfaces found were provided by a parent framework.

\subsubsection{Project Health and Ecosystem}

Project health metrics paint a mixed picture. Only 10 of 20 projects are considered \enquote{active} (updated within the last year). The median number of contributors is 4 (mean 6.3), suggesting that most projects are maintained by small teams (or, more likely, a singular dedicated user with some sporadic or one-off contributions from others). This is also reflected in community activity: 8 projects have dormant communication channels (inactive issue trackers, no other communication channels), while only 2 maintain dedicated channels (e.g., Discord, forums).

\emph{Commercial Viability} is high, all projects use a license that allows usage in proprietary, commercial projects (with some caveats, as mentioned in \zcref{sec:licensing}). Notably, only a single framework had visible offerings for commercial support, certainly also a relevant point when companies consider a framework for adoption.

\subsection{Limitations\label{sec:discussion-and-limitations-limitations}}

The overall methodology introduces several limitations on the meaningfulness of these findings.

\begin{enumerate}
    \item The evaluation of certain features may be ambiguous. For example the high prevalence of \emph{Notification} support may reveal a weakness of the evaluation: For headless APIs, i.e. a large number of the evaluated frameworks, providing event hooks is a natural and expected architectural feature. However, it is not clear if this support is intended for, or used as, the immediate user feedback mechanic defined in the evaluation criterion.

    \item The criteria themselves may have been misaligned with the typical scope of these frameworks. The 3rd Party Integration category (Social and Enterprise logins) yielded almost no positive results. No framework supported enterprise logins, and most were found to treat user identity as an opaque ID, relying either on the implementor or a parent framework to handle the actual authentication. In retrospect, these frameworks appear to (rightfully) consider user authentication as an out-of-scope concern.

    \item The evaluation process itself was constrained in several ways. Open-source frameworks were not evaluated in as much detail as would have been possible with more time (e.g.~more detailed code analysis, setup of local test environments for evaluation, \ldots). As a result of this, projects with more thorough and easily accessible documentation (not doc-comments on functions, for example) may have been favored in the evaluation. Additionally, for proprietary frameworks, there was an uncomfortably high reliance on marketing materials (due to a lack of better alternatives), also potentially skewing the results.
\end{enumerate}

\section{Conclusion\label{sec:conclusion}}

This thesis conducted a systematic survey and evaluation of 20 open-source and 5 proprietary gamification frameworks. By establishing a detailed catalogue of evaluation criteria (\zcref{sec:evaluation-criteria}), in addition to collecting the gamification frameworks to evaluate (\zcref{sec:frameworks-under-consideration}), we assessed the current landscape to understand the prevalence of specific features, common technical architectures, and project health.

Key findings indicate that foundational gamification mechanics (\emph{Points}, \emph{Badges}, \emph{Levels}) are almost universally supported among frameworks. We also found that open-source frameworks tended to be architected as \enquote{toolboxes} for gamification, while commercial frameworks aim to provide more complete, out-of-the-box solutions. The open-source ecosystem itself is mixed. While 50\% (10 of 20) projects show recent activity, most are maintained by small teams or solo developers (median 4 contributors) and lack dedicated community and support channels. Conversely, licensing is overwhelmingly permissive, with 19 of 20 projects viable for use in closed-source commercial applications.

The primary contribution of this work is a structured overview of current open-source gamification frameworks, contrasted against a more limited set of commercial offerings. The findings are, however, subject to the limitations discussed in \zcref{sec:discussion-and-limitations-limitations}, most importantly the reliance on public documentation and marketing materials.

Future work could proceed in several directions: A more hands-on evaluation of a selection of open-source frameworks could yield more concrete insights, particularly also in regards to integration complexity and performance. Additionally, the analysis of proprietary tools could be expanded on significantly, perhaps by obtaining demo accounts to provide a more balanced comparison.

\pgfplotsset{
    tick label style={font=\small},
    label style={font=\small},
    legend style={font=\footnotesize},
} 

\begin{figure}[p]
    \centering

    \caption{Number of commits (left axis, solid) and number of total lines changed (right axis, dashed) over time for the evaluated frameworks, as per \zcref{sec:development-activity-and-stability}. Data taken from the respective repository's default branch. Note the different scales of the y-axes across graphs.\label{fig:commits-over-time}}

    % https://tex.stackexchange.com/questions/444761/pgfplots-print-groupplots-over-multiple-pages
    \centerline{\begin{tikzpicture}
        \begin{groupplot}[
            group style={
                group size=1 by 20,
                xticklabels at=edge bottom,
                x descriptions at=edge bottom,
                horizontal sep=0pt,vertical sep=3pt, 
            },
            date coordinates in=x,
            width=\textwidth,
            height=2.5cm,
            yticklabel style={font=\tiny},
            enlarge x limits=false,
            enlarge y limits=0.2,
            xticklabel=\year,
            xtick distance=365,
            xmajorgrids,
            x grid style=dashed,
            y label style={at={(-0.05,0.5)}, rotate=-90},
            % yticklabel style=black!80!black,
            % every y tick/.style={black}
        ]
            \nextgroupplot[
                ylabel={\cite{ActiDoo}}
            ]
            \addplot+[mark=none, black] table [
                col sep=comma,
                x=date,
                y=commit_count,
            ] {./data/gamification-engine_monthly.csv};

            \nextgroupplot[
                ylabel={\cite{SieteValles}}
            ]
            \addplot+[mark=none, black] table [
                col sep=comma,
                x=date,
                y=commit_count,
            ] {./data/siete-valles_monthly.csv};

            \nextgroupplot[
                ylabel={\cite{UserInfuser}}
            ]
            \addplot+[mark=none, black] table [
                col sep=comma,
                x=date,
                y=commit_count,
            ] {./data/UserInfuser_monthly.csv};

            \nextgroupplot[
                ylabel={\cite{GamificationServer}}
            ]
            \addplot+[mark=none, black] table [
                col sep=comma,
                x=date,
                y=commit_count,
            ] {./data/Gamification-Server_monthly.csv};

            \nextgroupplot[
                ylabel={\cite{DjangoGamification}}
            ]
            \addplot+[mark=none, black] table [
                col sep=comma,
                x=date,
                y=commit_count,
            ] {./data/django-gamification_monthly.csv};

            \nextgroupplot[
                ylabel={\cite{FlarumGamification}}
            ]
            \addplot+[mark=none, black] table [
                col sep=comma,
                x=date,
                y=commit_count,
            ] {./data/FlarumGamification_monthly.csv};

            \nextgroupplot[
                ylabel={\cite{Gioco}}
            ]
            \addplot+[mark=none, black] table [
                col sep=comma,
                x=date,
                y=commit_count,
            ] {./data/gioco_monthly.csv};

            \nextgroupplot[
                ylabel={\cite{HyperosloGamification}}
            ]
            \addplot+[mark=none, black] table [
                col sep=comma,
                x=date,
                y=commit_count,
            ] {./data/HyperosloGamification_monthly.csv};

            \nextgroupplot[
                ylabel={\cite{MoodleBlockXP}}
            ]
            \addplot+[mark=none, black] table [
                col sep=comma,
                x=date,
                y=commit_count,
            ] {./data/moodle-block_xp_monthly.csv};

            \nextgroupplot[
                ylabel={\cite{ScoreJS}}
            ]
            \addplot+[mark=none, black] table [
                col sep=comma,
                x=date,
                y=commit_count,
            ] {./data/score.js_monthly.csv};

            \nextgroupplot[
                ylabel={\cite{LaravelGamify}}
            ]
            \addplot+[mark=none, black] table [
                col sep=comma,
                x=date,
                y=commit_count,
            ] {./data/laravel-gamify_monthly.csv};

            \nextgroupplot[
                ylabel={\cite{Yay}}
            ]
            \addplot+[mark=none, black] table [
                col sep=comma,
                x=date,
                y=commit_count,
            ] {./data/yay_monthly.csv};

            \nextgroupplot[
                ylabel={\cite{Kinben}}
            ]
            \addplot+[mark=none, black] table [
                col sep=comma,
                x=date,
                y=commit_count,
            ] {./data/GamificationEngine-Kinben_monthly.csv};

            \nextgroupplot[
                ylabel={\cite{Oasis}}
            ]
            \addplot+[mark=none, black] table [
                col sep=comma,
                x=date,
                y=commit_count,
            ] {./data/oasis_monthly.csv};

            \nextgroupplot[
                ylabel={\cite{LevelUp}}
            ]
            \addplot+[mark=none, black] table [
                col sep=comma,
                x=date,
                y=commit_count,
            ] {./data/level-up_monthly.csv};

            \nextgroupplot[
                ylabel={\cite{Gamify}}
            ]
            \addplot+[mark=none, black] table [
                col sep=comma,
                x=date,
                y=commit_count,
            ] {./data/Gamify_monthly.csv};

            \nextgroupplot[
                ylabel={\cite{GamifyLaravel}}
            ]
            \addplot+[mark=none, black] table [
                col sep=comma,
                x=date,
                y=commit_count,
            ] {./data/gamify-laravel_monthly.csv};

            \nextgroupplot[
                ylabel={\cite{Honor}}
            ]
            \addplot+[mark=none, black] table [
                col sep=comma,
                x=date,
                y=commit_count,
            ] {./data/honor_monthly.csv};

            \nextgroupplot[
                ylabel={\cite{PhpGamify}}
            ]
            \addplot+[mark=none, black] table [
                col sep=comma,
                x=date,
                y=commit_count,
            ] {./data/php-gamify_monthly.csv};

            \nextgroupplot[
                ylabel={\cite{AcisGamificationFramework}}
            ]
            \addplot+[mark=none, black] table [
                col sep=comma,
                x=date,
                y=commit_count,
            ] {./data/Gamification-Framework_monthly.csv};
        \end{groupplot}
        
        \begin{groupplot}[
            title style={at={(1,0.5)}},
            group style={
                group size=1 by 20,
                xticklabels at=edge bottom,
                x descriptions at=edge bottom,
                y descriptions at=edge right,
                horizontal sep=0pt,vertical sep=3pt, 
            },
            date coordinates in=x,
            width=\textwidth,
            height=2.5cm,
            yticklabel style={font=\tiny},
            enlarge x limits=false,
            enlarge y limits=0.2,
            xticklabel=\year,
            xtick distance=365,
            xmajorgrids,
            x grid style=dashed,
            y label style={at={(-0.05,0.5)}, rotate=-90},
            xtick=\empty, axis line style=transparent,
            % yticklabel style=black!80!black,
            % every y tick/.style={black}
        ]
            \nextgroupplot[scaled y ticks=false]
            \addplot+[mark=none, black, dashed] table [
                col sep=comma,
                x=date,
                y=total_changes,
            ] {./data/gamification-engine_monthly.csv};

            \nextgroupplot[scaled y ticks=false]
            \addplot+[mark=none, black, dashed] table [
                col sep=comma,
                x=date,
                y=total_changes,
            ] {./data/siete-valles_monthly.csv};

            \nextgroupplot[scaled y ticks=false]
            \addplot+[mark=none, black, dashed] table [
                col sep=comma,
                x=date,
                y=total_changes,
            ] {./data/UserInfuser_monthly.csv};

            \nextgroupplot[scaled y ticks=false]
            \addplot+[mark=none, black, dashed] table [
                col sep=comma,
                x=date,
                y=total_changes,
            ] {./data/Gamification-Server_monthly.csv};

            \nextgroupplot[scaled y ticks=false]
            \addplot+[mark=none, black, dashed] table [
                col sep=comma,
                x=date,
                y=total_changes,
            ] {./data/django-gamification_monthly.csv};

            \nextgroupplot[scaled y ticks=false]
            \addplot+[mark=none, black, dashed] table [
                col sep=comma,
                x=date,
                y=total_changes,
            ] {./data/FlarumGamification_monthly.csv};

            \nextgroupplot[scaled y ticks=false]
            \addplot+[mark=none, black, dashed] table [
                col sep=comma,
                x=date,
                y=total_changes,
            ] {./data/gioco_monthly.csv};

            \nextgroupplot[scaled y ticks=false]
            \addplot+[mark=none, black, dashed] table [
                col sep=comma,
                x=date,
                y=total_changes,
            ] {./data/HyperosloGamification_monthly.csv};

            \nextgroupplot[scaled y ticks=false]
            \addplot+[mark=none, black, dashed] table [
                col sep=comma,
                x=date,
                y=total_changes,
            ] {./data/moodle-block_xp_monthly.csv};

            \nextgroupplot[scaled y ticks=false]
            \addplot+[mark=none, black, dashed] table [
                col sep=comma,
                x=date,
                y=total_changes,
            ] {./data/score.js_monthly.csv};

            \nextgroupplot[scaled y ticks=false]
            \addplot+[mark=none, black, dashed] table [
                col sep=comma,
                x=date,
                y=total_changes,
            ] {./data/laravel-gamify_monthly.csv};

            \nextgroupplot[scaled y ticks=false]
            \addplot+[mark=none, black, dashed] table [
                col sep=comma,
                x=date,
                y=total_changes,
            ] {./data/yay_monthly.csv};

            \nextgroupplot[scaled y ticks=false]
            \addplot+[mark=none, black, dashed] table [
                col sep=comma,
                x=date,
                y=total_changes,
            ] {./data/GamificationEngine-Kinben_monthly.csv};

            \nextgroupplot[scaled y ticks=false]
            \addplot+[mark=none, black, dashed] table [
                col sep=comma,
                x=date,
                y=total_changes,
            ] {./data/oasis_monthly.csv};

            \nextgroupplot[scaled y ticks=false]
            \addplot+[mark=none, black, dashed] table [
                col sep=comma,
                x=date,
                y=total_changes,
            ] {./data/level-up_monthly.csv};

            \nextgroupplot[scaled y ticks=false]
            \addplot+[mark=none, black, dashed] table [
                col sep=comma,
                x=date,
                y=total_changes,
            ] {./data/Gamify_monthly.csv};

            \nextgroupplot[scaled y ticks=false]
            \addplot+[mark=none, black, dashed] table [
                col sep=comma,
                x=date,
                y=total_changes,
            ] {./data/gamify-laravel_monthly.csv};

            \nextgroupplot[scaled y ticks=false]
            \addplot+[mark=none, black, dashed] table [
                col sep=comma,
                x=date,
                y=total_changes,
            ] {./data/honor_monthly.csv};

            \nextgroupplot[scaled y ticks=false]
            \addplot+[mark=none, black, dashed] table [
                col sep=comma,
                x=date,
                y=total_changes,
            ] {./data/php-gamify_monthly.csv};

            \nextgroupplot[scaled y ticks=false]
            \addplot+[mark=none, black, dashed] table [
                col sep=comma,
                x=date,
                y=total_changes,
            ] {./data/Gamification-Framework_monthly.csv};
        \end{groupplot}
    \end{tikzpicture}}
\end{figure}
\newgeometry{left=0.5cm, right=0.5cm, top=1cm, bottom=1cm, landscape}
\begin{landscape}
    \begin{table}[p]
        \scriptsize
    
        \renewcommand{\arraystretch}{1.3}
    
        \caption{Gamification Features (\fref{sec:gamification-features}) Evaluation Table (WIP) (\reqno{} = not supported, \reqpartial{} = partially supported, \reqyes{} = fully supported; see respective sections for details)\label{tab:gamification-features-eval}}
    
        \centerline{\begin{tabular}{@{} l cccccccccccccccccccccccc @{}} \toprule
            & \multicolumn{3}{l}{Meaning (\ref{sec:epic-meaning-calling})} & \multicolumn{5}{l}{Development (\ref{sec:development-accomplishment})} & \multicolumn{3}{l}{Creativity (\ref{sec:creativity-and-feedback})} & \multicolumn{2}{l}{Ownership (\ref{sec:ownership-and-possession})} & \multicolumn{4}{l}{Social (\ref{sec:social-influence-and-relatedness})} & \multicolumn{3}{l}{Scarcity (\ref{sec:scarcity-and-impatience})} & \multicolumn{2}{l}{Curiosity (\ref{sec:unpredictability-and-curiosity})} & \multicolumn{2}{l}{Loss (\ref{sec:loss-and-avoidance})} \\ \cmidrule(r){2-4} \cmidrule(r){5-9} \cmidrule(r){10-12} \cmidrule(r){13-14} \cmidrule(r){15-18} \cmidrule(r){19-21} \cmidrule(r){22-23}  \cmidrule{24-25}
            Ref. & Tutorials & \makecell[lt]{Narra--\\tives} & Quests & Points & Levels & Badges & \makecell[lt]{Leader--\\boards} & Rewards & Avatar & CYP & \makecell[lt]{Notifi--\\cations} & Collectibles & Virtual \$ & Profiles & Roles & Social & Widgets & Events & \makecell[lt]{Time-\\gates} & \makecell[lt]{Dead--\\lines} & Var. Rewards & Gambling & Streaks & \makecell[lt]{Punish--\\ments} \\\midrule
        
            % reference
            \cite{ActiDoo} &
            % tutorials
            \reqno{} &
            % narratives
            \reqpartial{} &
            % quests
            \reqyes{} &
            % points
            \reqyes{} &
            % levels
            \reqyes{} &
            % badges/achievements
            \reqyes{} &
            % leaderboards
            \reqyes{} &
            % rewards
            \reqpartial{} &
            % avatar
            \reqno{} &
            % CYP
            \reqno{} &
            % notifications
            \reqyes{} &
            % collectibles
            \reqno{} &
            % virtual \$
            \reqno{} &
            % profiles
            \reqno{} &
            % roles
            \reqyes{} &
            % social
            \reqpartial{} &
            % widgets
            \reqno{} &
            % events
            \reqno{} &
            % time-gates
            \reqno{} &
            % deadlines
            \reqyes{} &
            % var. rewards
            \reqno{} &
            % gambling
            \reqno{} &
            % streaks
            \reqno{} &
            % punishments
            \reqno{} \\
        
            % reference
            \cite{SieteValles} &
            % tutorials
            \reqno{} &
            % narratives
            \reqno{} &
            % quests
            \reqyes{} &
            % points
            \reqyes{} &
            % levels
            \reqpartial{} &
            % badges/achievements
            \reqyes{} &
            % leaderboards
            \reqno{} &
            % rewards
            \reqno{} &
            % avatar
            \reqno{} &
            % CYP
            \reqno{} &
            % notifications
            \reqno{} &
            % collectibles
            \reqno{} &
            % virtual \$
            \reqno{} &
            % profiles
            \reqno{} &
            % roles
            \reqno{} &
            % social
            \reqno{} &
            % widgets
            \reqno{} &
            % events
            \reqno{} &
            % time-gates
            \reqno{} &
            % deadlines
            \reqno{} &
            % var. rewards
            \reqno{} &
            % gambling
            \reqno{} &
            % streaks
            \reqno{} &
            % punishments
            \reqno{} \\
    
            % reference
            \cite{UserInfuser} &
            % tutorials
            \reqno{}&
            % narratives
            \reqno{}&
            % quests
            \reqno{}&
            % points
            \reqyes{}&
            % levels
            \reqno{}&
            % badges/achievements
            \reqyes{}&
            % leaderboards
            \reqyes{}&
            % rewards
            \reqpartial{}&
            % avatar
            \reqno{}&
            % CYP
            \reqno{}&
            % notifications
            \reqyes{}&
            % collectibles
            \reqno{}&
            % virtual \$
            \reqno{}&
            % profiles
            \reqno{}&
            % roles
            \reqno{}&
            % social
            \reqno{}&
            % widgets
            \reqyes{}&
            % events
            \reqno{}&
            % time-gates
            \reqno{}&
            % deadlines
            \reqno{}&
            % var. rewards
            \reqno{}&
            % gambling
            \reqno{}&
            % streaks
            \reqno{}&
            % punishments
            \reqno{} \\

            % reference
            \cite{GamificationServer} &
            % tutorials
            \reqno{} &
            % narratives
            \reqno{} &
            % quests
            \reqno{} &
            % points
            \reqyes{} &
            % levels
            \reqno{} &
            % badges/achievements
            \reqyes{} &
            % leaderboards
            \reqno{}&
            % rewards
            \reqpartial{} &
            % avatar
            \reqno{} &
            % CYP
            \reqno{} &
            % notifications
            \reqno{} &
            % collectibles
            \reqno{} &
            % virtual \$
            \reqno{} &
            % profiles
            \reqno{} &
            % roles
            \reqno{} &
            % social
            \reqno{} &
            % widgets
            \reqno{} &
            % events
            \reqno{} &
            % time-gates
            \reqno{} &
            % deadlines
            \reqno{} &
            % var. rewards
            \reqno{} &
            % gambling
            \reqno{} &
            % streaks
            \reqno{} &
            % punishments
            \reqno{}\\

            % reference
            \cite{DjangoGamification} &
            % tutorials
            \reqno{} &
            % narratives
            \reqno{} &
            % quests
            \reqno{} &
            % points
            \reqyes{} &
            % levels
            \reqpartial{} &
            % badges/achievements
            \reqyes{} &
            % leaderboards
            \reqno{} &
            % rewards
            \reqyes{} &
            % avatar
            \reqno{} &
            % CYP
            \reqno{} &
            % notifications
            \reqno{} &
            % collectibles
            \reqpartial{} &
            % virtual \$
            \reqno{} &
            % profiles
            \reqpartial{} &
            % roles
            \reqno{} &
            % social
            \reqno{} &
            % widgets
            \reqno{} &
            % events
            \reqno{} &
            % time-gates
            \reqno{} &
            % deadlines
            \reqno{} &
            % var. rewards
            \reqno{} &
            % gambling
            \reqno{} &
            % streaks
            \reqno{} &
            % punishments
            \reqno{}\\

            % reference
            \cite{FlarumGamification} &
            % tutorials
            \reqno{} &
            % narratives
            \reqno{} &
            % quests
            \reqno{} &
            % points
            \reqyes{} &
            % levels
            \reqyes{} &
            % badges/achievements
            \reqpartial{} &
            % leaderboards
            \reqyes{} &
            % rewards
            \reqno{} &
            % avatar
            \reqno{} &
            % CYP
            \reqno{} &
            % notifications
            \reqyes{} &
            % collectibles
            \reqno{} &
            % virtual \$
            \reqno{} &
            % profiles
            \reqyes{} &
            % roles
            \reqno{} &
            % social
            \reqyes{} &
            % widgets
            \reqno{} &
            % events
            \reqno{} &
            % time-gates
            \reqno{} &
            % deadlines
            \reqno{} &
            % var. rewards
            \reqno{} &
            % gambling
            \reqno{} &
            % streaks
            \reqno{} &
            % punishments
            \reqyes{}\\

            % reference
            \cite{Gioco} &
            % tutorials
            \reqpartial{} &
            % narratives
            \reqno{} &
            % quests
            \reqno{} &
            % points
            \reqyes{} &
            % levels
            \reqyes{} &
            % badges/achievements
            \reqyes{} &
            % leaderboards
            \reqpartial{} &
            % rewards
            \reqno{} &
            % avatar
            \reqno{} &
            % CYP
            \reqno{} &
            % notifications
            \reqno{} &
            % collectibles
            \reqno{} &
            % virtual \$
            \reqpartial{} &
            % profiles
            \reqno{} &
            % roles
            \reqpartial{} &
            % social
            \reqno{} &
            % widgets
            \reqno{} &
            % events
            \reqno{} &
            % time-gates
            \reqno{} &
            % deadlines
            \reqno{} &
            % var. rewards
            \reqno{} &
            % gambling
            \reqno{} &
            % streaks
            \reqno{} &
            % punishments
            \reqno{}\\

            % reference
            \cite{HyperosloGamification}&
            % tutorials
            \reqno{} &
            % narratives
            \reqno{} &
            % quests
            \reqyes{} &
            % points
            \reqno{}&
            % levels
            \reqno{}&
            % badges/achievements
            \reqyes{}&
            % leaderboards
            \reqno{}&
            % rewards
            \reqyes{}&
            % avatar
            \reqno{}&
            % CYP
            \reqno{}&
            % notifications
            \reqyes{}&
            % collectibles
            \reqpartial{}&
            % virtual \$
            \reqno{}&
            % profiles
            \reqno{}&
            % roles
            \reqno{}&
            % social
            \reqno{}&
            % widgets
            \reqno{}&
            % events
            \reqno{}&
            % time-gates
            \reqno{}&
            % deadlines
            \reqno{}&
            % var. rewards
            \reqno{}&
            % gambling
            \reqno{}&
            % streaks
            \reqpartial{}&
            % punishments
            \reqno{}\\

            % reference
            \cite{MoodleBlockXP}&
            % tutorials
            \reqno{} &
            % narratives
            \reqno{} &
            % quests
            \reqyes{} &
            % points
            \reqyes{} &
            % levels
            \reqyes{} &
            % badges/achievements
            \reqyes{} &
            % leaderboards
            \reqno{} &
            % rewards
            \reqyes{} &
            % avatar
            \reqno{} &
            % CYP
            \reqno{} &
            % notifications
            \reqyes{} &
            % collectibles
            \reqpartial{} &
            % virtual \$
            \reqno{}&
            % profiles
            \reqyes{}&
            % roles
            \reqno{}&
            % social
            \reqno{}&
            % widgets
            \reqno{}&
            % events
            \reqno{}&
            % time-gates
            \reqno{}&
            % deadlines
            \reqno{}&
            % var. rewards
            \reqno{}&
            % gambling
            \reqno{}&
            % streaks
            \reqno{}&
            % punishments
            \reqpartial{}\\

            % reference
            \cite{ScoreJS}&
            % tutorials
            \reqno{}&
            % narratives
            \reqpartial{}&
            % quests
            \reqno{}&
            % points
            \reqyes{}&
            % levels
            \reqyes{}&
            % badges/achievements
            \reqno{}&
            % leaderboards
            \reqno{}&
            % rewards
            \reqno{}&
            % avatar
            \reqno{}&
            % CYP
            \reqno{}&
            % notifications
            \reqyes{}&
            % collectibles
            \reqno{}&
            % virtual \$
            \reqno{}&
            % profiles
            \reqno{}&
            % roles
            \reqno{}&
            % social
            \reqno{}&
            % widgets
            \reqno{}&
            % events
            \reqpartial{}&
            % time-gates
            \reqpartial{}&
            % deadlines
            \reqno{}&
            % var. rewards
            \reqno{}&
            % gambling
            \reqno{}&
            % streaks
            \reqno{}&
            % punishments
            \reqpartial{}\\

            % reference
            \cite{LaravelGamify}&
            % tutorials
            \reqno{}&
            % narratives
            \reqno{}&
            % quests
            \reqno{}&
            % points
            \reqyes{}&
            % levels
            \reqno{}&
            % badges/achievements
            \reqno{}&
            % leaderboards
            \reqno{}&
            % rewards
            \reqno{}&
            % avatar
            \reqno{}&
            % CYP
            \reqno{}&
            % notifications
            \reqyes{}&
            % collectibles
            \reqpartial{}&
            % virtual \$
            \reqno{}&
            % profiles
            \reqno{}&
            % roles
            \reqno{}&
            % social
            \reqno{}&
            % widgets
            \reqno{}&
            % events
            \reqno{}&
            % time-gates
            \reqno{}&
            % deadlines
            \reqno{}&
            % var. rewards
            \reqno{}&
            % gambling
            \reqno{}&
            % streaks
            \reqno{}&
            % punishments
            \reqno{}\\

            % reference
            \cite{Yay}&
            % tutorials
            \reqno{}&
            % narratives
            \reqno{}&
            % quests
            \reqyes{}&
            % points
            \reqyes{}&
            % levels
            \reqyes{}&
            % badges/achievements
            \reqyes{}&
            % leaderboards
            \reqyes{}&
            % rewards
            \reqpartial{}&
            % avatar
            \reqno{}&
            % CYP
            \reqno{}&
            % notifications
            \reqyes{}&
            % collectibles
            \reqno{}&
            % virtual \$
            \reqno{}&
            % profiles
            \reqno{}&
            % roles
            \reqno{}&
            % social
            \reqno{}&
            % widgets
            \reqno{}&
            % events
            \reqno{}&
            % time-gates
            \reqno{}&
            % deadlines
            \reqno{}&
            % var. rewards
            \reqno{}&
            % gambling
            \reqno{}&
            % streaks
            \reqno{}&
            % punishments
            \reqno{}\\

            % reference
            \cite{Kinben}&
            % tutorials
            \reqno{}&
            % narratives
            \reqno{}&
            % quests
            \reqyes{}&
            % points
            \reqyes{}&
            % levels
            \reqyes{}&
            % badges/achievements
            \reqyes{}&
            % leaderboards
            \reqyes{}&
            % rewards
            \reqno{}&
            % avatar
            \reqno{}&
            % CYP
            \reqpartial{}&
            % notifications
            \reqno{}&
            % collectibles
            \reqno{}&
            % virtual \$
            \reqno{}&
            % profiles
            \reqyes{}&
            % roles
            \reqyes{}&
            % social
            \reqyes&
            % widgets
            \reqno{}&
            % events
            \reqno{}&
            % time-gates
            \reqno{}&
            % deadlines
            \reqno{}&
            % var. rewards
            \reqno{}&
            % gambling
            \reqno{}&
            % streaks
            \reqno{}&
            % punishments
            \reqno{}\\

            % reference
            \cite{Oasis}&
            % tutorials
            \reqno{}&
            % narratives
            \reqno{}&
            % quests
            \reqyes{}&
            % points
            \reqyes{}&
            % levels
            \reqyes{}&
            % badges/achievements
            \reqyes{}&
            % leaderboards
            \reqyes{}&
            % rewards
            \reqyes{}&
            % avatar
            \reqno{}&
            % CYP
            \reqno{}&
            % notifications
            \reqyes{}&
            % collectibles
            \reqno{}&
            % virtual \$
            \reqno{}&
            % profiles
            \reqno{}&
            % roles
            \reqno{}&
            % social
            \reqpartial{}&
            % widgets
            \reqno{}&
            % events
            \reqno{}&
            % time-gates
            \reqyes{}&
            % deadlines
            \reqyes{}&
            % var. rewards
            \reqno{}&
            % gambling
            \reqno{}&
            % streaks
            \reqyes{}&
            % punishments
            \reqno{}\\

            % reference
            \cite{LevelUp}&
            % tutorials
            \reqno{}&
            % narratives
            \reqno{}&
            % quests
            \reqno{}&
            % points
            \reqyes{}&
            % levels
            \reqyes{}&
            % badges/achievements
            \reqyes{}&
            % leaderboards
            \reqyes{}&
            % rewards
            \reqno{}&
            % avatar
            \reqno{}&
            % CYP
            \reqno{}&
            % notifications
            \reqyes{}&
            % collectibles
            \reqno{}&
            % virtual \$
            \reqno{}&
            % profiles
            \reqpartial{}&
            % roles
            \reqno{}&
            % social
            \reqno{}&
            % widgets
            \reqno{}&
            % events
            \reqpartial{}&
            % time-gates
            \reqno{}&
            % deadlines
            \reqno{}&
            % var. rewards
            \reqno{}&
            % gambling
            \reqno{}&
            % streaks
            \reqno{}&
            % punishments
            \reqno{}\\

            % reference
            \cite{Gamify}&
            % tutorials
            \reqno{}&
            % narratives
            \reqno{}&
            % quests
            \reqyes{}&
            % points
            \reqyes{}&
            % levels
            \reqno{}&
            % badges/achievements
            \reqno{}&
            % leaderboards
            \reqyes{}&
            % rewards
            \reqyes{}&
            % avatar
            \reqno{}&
            % CYP
            \reqno{}&
            % notifications
            \reqyes{}&
            % collectibles
            \reqno{}&
            % virtual \$
            \reqyes{}&
            % profiles
            \reqno{}&
            % roles
            \reqyes{}&
            % social
            \reqno{}&
            % widgets
            \reqno{}&
            % events
            \reqno{}&
            % time-gates
            \reqno{}&
            % deadlines
            \reqno{}&
            % var. rewards
            \reqno{}&
            % gambling
            \reqno{}&
            % streaks
            \reqno{}&
            % punishments
            \reqno{}\\

            % reference
            \cite{GamifyLaravel}&
            % tutorials
            \reqno{}&
            % narratives
            \reqno{}&
            % quests
            \reqyes{}&
            % points
            \reqyes{}&
            % levels
            \reqyes{}&
            % badges/achievements
            \reqyes{}&
            % leaderboards
            \reqno{}&
            % rewards
            \reqpartial{}&
            % avatar
            \reqno{}&
            % CYP
            \reqno{}&
            % notifications
            \reqyes{}&
            % collectibles
            \reqno{}&
            % virtual \$
            \reqno{}&
            % profiles
            \reqyes{}&
            % roles
            \reqyes{}&
            % social
            \reqpartial{}&
            % widgets
            \reqno{}&
            % events
            \reqno{}&
            % time-gates
            \reqno{}&
            % deadlines
            \reqno{}&
            % var. rewards
            \reqno{}&
            % gambling
            \reqno{}&
            % streaks
            \reqno{}&
            % punishments
            \reqno{}\\

            % reference
            \cite{Honor}&
            % tutorials
            \reqno{}&
            % narratives
            \reqno{}&
            % quests
            \reqno{}&
            % points
            \reqyes{}&
            % levels
            \reqno{}&
            % badges/achievements
            \reqno{}&
            % leaderboards
            \reqyes{}&
            % rewards
            \reqno{}&
            % avatar
            \reqno{}&
            % CYP
            \reqno{}&
            % notifications
            \reqpartial{}&
            % collectibles
            \reqno{}&
            % virtual \$
            \reqno{}&
            % profiles
            \reqno{}&
            % roles
            \reqno{}&
            % social
            \reqno{}&
            % widgets
            \reqno{}&
            % events
            \reqno{}&
            % time-gates
            \reqno{}&
            % deadlines
            \reqno{}&
            % var. rewards
            \reqno{}&
            % gambling
            \reqno{}&
            % streaks
            \reqno{}&
            % punishments
            \reqno{}\\

            % reference
            \cite{PhpGamify}&
            % tutorials
            \reqno{}&
            % narratives
            \reqno{}&
            % quests
            \reqno{}&
            % points
            \reqyes{}&
            % levels
            \reqyes{}&
            % badges/achievements
            \reqyes{}&
            % leaderboards
            \reqno{}&
            % rewards
            \reqyes{}&
            % avatar
            \reqno{}&
            % CYP
            \reqno{}&
            % notifications
            \reqyes{}&
            % collectibles
            \reqno{}&
            % virtual \$
            \reqyes{}&
            % profiles
            \reqno{}&
            % roles
            \reqno{}&
            % social
            \reqno{}&
            % widgets
            \reqno{}&
            % events
            \reqno{}&
            % time-gates
            \reqno{}&
            % deadlines
            \reqno{}&
            % var. rewards
            \reqno{}&
            % gambling
            \reqno{}&
            % streaks
            \reqno{}&
            % punishments
            \reqno{}\\

            % reference
            \cite{AcisGamificationFramework}&
            % tutorials
            \reqno{}&
            % narratives
            \reqno{}&
            % quests
            \reqyes{}&
            % points
            \reqyes{}&
            % levels
            \reqyes{}&
            % badges/achievements
            \reqyes{}&
            % leaderboards
            \reqyes{}&
            % rewards
            \reqpartial{}&
            % avatar
            \reqno{}&
            % CYP
            \reqno{}&
            % notifications
            \reqyes{}&
            % collectibles
            \reqpartial{}&
            % virtual \$
            \reqno{}&
            % profiles
            \reqyes{}&
            % roles
            \reqno{}&
            % social
            \reqpartial{}&
            % widgets
            \reqno{}&
            % events
            \reqno{}&
            % time-gates
            \reqyes{}&
            % deadlines
            \reqyes{}&
            % var. rewards
            \reqno{}&
            % gambling
            \reqno{}&
            % streaks
            \reqyes{}&
            % punishments
            \reqno{}\\
    
            % % reference
            % &
            % % tutorials
            % &
            % % narratives
            % &
            % % quests
            % &
            % % points
            % &
            % % levels
            % &
            % % badges/achievements
            % &
            % % leaderboards
            % &
            % % rewards
            % &
            % % avatar
            % &
            % % CYP
            % &
            % % notifications
            % &
            % % collectibles
            % &
            % % virtual \$
            % &
            % % profiles
            % &
            % % roles
            % &
            % % social
            % &
            % % widgets
            % &
            % % events
            % &
            % % time-gates
            % &
            % % deadlines
            % &
            % % var. rewards
            % &
            % % gambling
            % &
            % % streaks
            % &
            % % punishments
            % \\
            \bottomrule
        \end{tabular}}
    \end{table}
    \end{landscape}
\begin{landscape}
    \begin{table}[p]
        \scriptsize
    
        \renewcommand{\arraystretch}{1.3}
    
        \caption{Technical and Architectural Characteristics (\fref{sec:technical-and-architectural-characteristics}) and Project Health and Ecosystem (\fref{sec:project-health-and-ecosystem}) Evaluation Table (WIP)\label{tab:technical-and-architectural-characteristics-eval}}
    
        \centerline{\begin{tabular}{@{} l ccccccccccccccccccc @{}} \toprule
            & \multicolumn{3}{l}{Integration (\ref{sec:integration})} & \multicolumn{3}{l}{Technology Stack (\ref{sec:technology-stack-requirements})} & \multicolumn{2}{l}{i18n (\ref{sec:internationalization-i18n-capabilities})} & \multicolumn{2}{l}{Interfaces (\ref{sec:provided-interfaces})} & \multicolumn{2}{l}{3rd Party (\ref{sec:3rd-party-integration-capabilities})} & \multicolumn{3}{l}{Activity (\ref{sec:development-activity-and-stability})} & \multicolumn{2}{l}{Licensing (\ref{sec:licensing})} & \multicolumn{2}{l}{Community (\ref{sec:community-and-support})} \\ \cmidrule(r){2-4} \cmidrule(r){5-7} \cmidrule(r){8-9} \cmidrule(r){10-11} \cmidrule(r){12-13} \cmidrule(r){14-16} \cmidrule(r){17-18} \cmidrule{19-20}
            Ref. & Dependency & Deployment & Plugin & Languages & Persistence & Framework & Supported & Ext. & User & Admin & Social & Ent. & Backers & Forks & Contributors & License & \$? & Community & \$ Support \\ \midrule
    
            % reference
            \cite{ActiDoo} &
            % dependency
            \reqno{} &
            % deployment
            \reqyes{} &
            % plugin
            \reqno{} &
            % programming languages
            Python &
            % persistence
            Postgres &
            % framework
            \reqno{} &
            % supported languages
            EN &
            % extensible i18n
            \reqyes{} &
            % user
            \reqno{} &
            % admin
            \reqno{} &
            % social
            \reqno{} &
            % enterprise
            \reqno{} &
            % backers
            Company &
            % forks
            111 &
            % contributors
            9 &
            % license
            MIT &
            % license commercially viable?
            \reqyes{} &
            % community
            \reqpartial{} &
            % commercial support offered?
            \reqpartial{} \\
    
            % reference
            \cite{SieteValles} &
            % dependency
            \reqyes{} &
            % deployment
            \reqno{} &
            % plugin
            \reqno{} &
            % programming languages
            Ruby &
            % persistence
            Active Record &
            % framework
            Rails &
            % supported languages
            \reqna{} &
            % extensible i18n
            \reqna{} &
            % user
            \reqpartial{} &
            % admin
            \reqpartial{} &
            % social
            \reqno{} &
            % enterprise
            \reqno{} &
            % backers
            Private &
            % forks
            4 &
            % contributors
            1 &
            % license
            MIT &
            % license commercially viable?
            \reqyes{} &
            % community
            \reqpartial{} &
            % commercial support offered?
            \reqno{} \\
    
            % reference
            \cite{UserInfuser} &
            % dependency
            \reqpartial{} &
            % deployment
            \reqyes{} &
            % plugin
            \reqno{} &
            % programming languages
            Python &
            % persistence
            G. Datastore &
            % framework
            G. App Engine &
            % supported languages
            EN &
            % extensible i18n
            \reqno{} &
            % user
            \reqyes{} &
            % admin
            \reqna{} &
            % social
            \reqno{} &
            % enterprise
            \reqno{} &
            % backers
            Abandoned &
            % forks
            52 &
            % contributors
            3 &
            % license
            Unclear &
            % license commercially viable?
            \reqpartial{} &
            % community
            \reqno{} &
            % commercial support offered?
            \reqno{} \\

            % reference
            \cite{GamificationServer} &
            % dependency
            \reqpartial{} &
            % deployment
            \reqyes{} &
            % plugin
            \reqno{} &
            % programming languages
            JavaScript &
            % persistence
            MongoDB &
            % framework
            Express &
            % supported languages
            \reqna{} &
            % extensible i18n
            \reqna{} &
            % user
            \reqno{} &
            % admin
            \reqno{} &
            % social
            \reqno{} &
            % enterprise
            \reqno{} &
            % backers
            Company &
            % forks
            4 &
            % contributors
            1 &
            % license
            MIT &
            % license commercially viable?
            \reqyes{} &
            % community
            \reqno{} &
            % commercial support offered?
            \reqno{} \\

            % reference
            \cite{DjangoGamification}&
            % dependency
            \reqyes{} &
            % deployment
            \reqno{} &
            % plugin
            \reqno{} &
            % programming languages
            Python &
            % persistence
            Django ORM &
            % framework
            Django &
            % supported languages
            \reqna{} &
            % extensible i18n
            \reqna{} &
            % user
            \reqno{} &
            % admin
            \reqna{} &
            % social
            \reqno{} &
            % enterprise
            \reqno{} &
            % backers
            Private&
            % forks
            29 &
            % contributors
            5 &
            % license
            BSD3 &
            % license commercially viable?
            \reqyes{} &
            % community
            \reqpartial{} &
            % commercial support offered?
            \reqno{} \\
            
            % reference
            \cite{FlarumGamification} &
            % dependency
            \reqno{} &
            % deployment
            \reqno{} &
            % plugin
            \reqyes{} &
            % programming languages
            PHP/TS &
            % persistence
            Eloquent ORM &
            % framework
            Flarum &
            % supported languages
            \reqna{} &
            % extensible i18n
            \reqyes{} &
            % user
            \reqyes{} &
            % admin
            \reqyes{} &
            % social
            \reqna{} &
            % enterprise
            \reqna{} &
            % backers
            Nonprofit &
            % forks
            16 &
            % contributors
            11 &
            % license
            MIT &
            % license commercially viable?
            \reqyes{} &
            % community
            \reqyes{} &
            % commercial support offered?
            \reqpartial{}\\

            % reference
            \cite{Gioco}&
            % dependency
            \reqyes{} &
            % deployment
            \reqno{} &
            % plugin
            \reqno{} &
            % programming languages
            Ruby &
            % persistence
            Active Record &
            % framework
            Rails &
            % supported languages
            \reqno{} &
            % extensible i18n
            \reqno{} &
            % user
            \reqno{} &
            % admin
            \reqpartial{} &
            % social
            \reqno{} &
            % enterprise
            \reqno{} &
            % backers
            Private &
            % forks
            35 &
            % contributors
            11 &
            % license
            MIT &
            % license commercially viable?
            \reqyes{} &
            % community
            \reqpartial{} &
            % commercial support offered?
            \reqno{} \\

            % reference
            \cite{HyperosloGamification} &
            % dependency
            \reqyes{} &
            % deployment
            \reqno{} &
            % plugin
            \reqno{} &
            % programming languages
            Ruby &
            % persistence
            Active Record &
            % framework
            Rails &
            % supported languages
            EN, NO &
            % extensible i18n
            \reqyes{} &
            % user
            \reqyes{} &
            % admin
            \reqpartial{} &
            % social
            \reqno{} &
            % enterprise
            \reqno{} &
            % backers
            Company &
            % forks
            12&
            % contributors
            6&
            % license
            MIT &
            % license commercially viable?
            \reqyes{} &
            % community
            \reqno{} &
            % commercial support offered?
            \reqno{} \\

            % reference
            \cite{MoodleBlockXP} &
            % dependency
            \reqno{} &
            % deployment
            \reqno{} &
            % plugin
            \reqyes{} &
            % programming languages
            PHP&
            % persistence
            Moodle &
            % framework
            Moodle &
            % supported languages
            EN &
            % extensible i18n
            \reqyes{} &
            % user
            \reqyes{} &
            % admin
            \reqyes{} &
            % social
            \reqno{} &
            % enterprise
            \reqno{} &
            % backers
            Private &
            % forks
            48 &
            % contributors
            11 &
            % license
            GPLv3 &
            % license commercially viable?
            \reqyes{} &
            % community
            \reqyes{}&
            % commercial support offered?
            \reqyes{}\\

            % reference
            \cite{ScoreJS}&
            % dependency
            \reqyes{} &
            % deployment
            \reqno{} &
            % plugin
            \reqno{} &
            % programming languages
            JavaScript &
            % persistence
            Browser &
            % framework
            \reqno{} &
            % supported languages
            EN &
            % extensible i18n
            \reqpartial{} &
            % user
            \reqpartial{}&
            % admin
            \reqno{}&
            % social
            \reqno{}&
            % enterprise
            \reqno{}&
            % backers
            Private &
            % forks
            24&
            % contributors
            2&
            % license
            MIT&
            % license commercially viable?
            \reqyes{}&
            % community
            \reqpartial{}&
            % commercial support offered?
            \reqno{}\\

            % reference
            \cite{LaravelGamify} &
            % dependency
            \reqyes{} &
            % deployment
            \reqno{} &
            % plugin
            \reqno{} &
            % programming languages
            PHP&
            % persistence
            Eloquent ORM &
            % framework
            Laravel&
            % supported languages
            \reqna{}&
            % extensible i18n
            \reqna{}&
            % user
            \reqno{}&
            % admin
            \reqno{}&
            % social
            \reqna{}&
            % enterprise
            \reqna{}&
            % backers
            Private&
            % forks
            71&
            % contributors
            7&
            % license
            MIT&
            % license commercially viable?
            \reqyes{}&
            % community
            \reqpartial{}&
            % commercial support offered?
            \reqno{}\\

            % reference
            \cite{Yay}&
            % dependency
            \reqno{}&
            % deployment
            \reqyes{}&
            % plugin
            \reqno{}&
            % programming languages
            PHP&
            % persistence
            Doctrine ORM&
            % framework
            Symfony 4&
            % supported languages
            \reqna{}&
            % extensible i18n
            \reqna{}&
            % user
            \reqno{}&
            % admin
            \reqno{}&
            % social
            \reqno{}&
            % enterprise
            \reqno{}&
            % backers
            Private&
            % forks
            7&
            % contributors
            2&
            % license
            Apache 2.0&
            % license commercially viable?
            \reqyes{}&
            % community
            \reqpartial{}&
            % commercial support offered?
            \reqno{}\\

            % reference
            \cite{Kinben}&
            % dependency
            \reqno{}&
            % deployment
            \reqyes{}&
            % plugin
            \reqno{}&
            % programming languages
            Java&
            % persistence
            JPA&
            % framework
            Wildfly 8.2&
            % supported languages
            \reqna{}&
            % extensible i18n
            \reqna{}&
            % user
            \reqno{}&
            % admin
            \reqno{}&
            % social
            \reqno{}&
            % enterprise
            \reqno{}&
            % backers
            University&
            % forks
            5&
            % contributors
            4&
            % license
            LGPLv3&
            % license commercially viable?
            \reqyes{}&
            % community
            \reqno{}&
            % commercial support offered?
            \reqno{}\\

            % reference
            \cite{Oasis}&
            % dependency
            \reqyes{}&
            % deployment
            \reqyes{}&
            % plugin
            \reqno{}&
            % programming languages
            Java&
            % persistence
            Redis&
            % framework
            Spring Boot&
            % supported languages
            \reqna{}&
            % extensible i18n
            \reqna{}&
            % user
            \reqno{}&
            % admin
            \reqpartial{}&
            % social
            \reqno{}&
            % enterprise
            \reqno{}&
            % backers
            Private&
            % forks
            8&
            % contributors
            1&
            % license
            Apache 2.0&
            % license commercially viable?
            \reqyes{}&
            % community
            \reqpartial{}&
            % commercial support offered?
            \reqno{}\\

            % reference
            \cite{LevelUp}&
            % dependency
            \reqyes{}&
            % deployment
            \reqno{}&
            % plugin
            \reqno{}&
            % programming languages
            PHP&
            % persistence
            Eloquent ORM&
            % framework
            Laravel&
            % supported languages
            \reqna{} &
            % extensible i18n
            \reqno{} &
            % user
            \reqno{}&
            % admin
            \reqno{}&
            % social
            \reqno{}&
            % enterprise
            \reqno{}&
            % backers
            Private&
            % forks
            50&
            % contributors
            12&
            % license
            MIT&
            % license commercially viable?
            \reqyes{}&
            % community
            \reqpartial{}&
            % commercial support offered?
            \reqno{}\\

            % reference
            \cite{Gamify}&
            % dependency
            \reqno{}&
            % deployment
            \reqyes{}&
            % plugin
            \reqno{}&
            % programming languages
            JS&
            % persistence
            MongoDB&
            % framework
            Express&
            % supported languages
            \reqna{}&
            % extensible i18n
            \reqno{}&
            % user
            \reqyes{}&
            % admin
            \reqyes{}&
            % social
            \reqyes{}&
            % enterprise
            \reqno{}&
            % backers
            Private&
            % forks
            43&
            % contributors
            30&
            % license
            MIT&
            % license commercially viable?
            \reqyes{}&
            % community
            \reqpartial{}&
            % commercial support offered?
            \reqno{}\\

            % reference
            \cite{GamifyLaravel}&
            % dependency
            \reqyes{}&
            % deployment
            \reqno{}&
            % plugin
            \reqno{}&
            % programming languages
            PHP&
            % persistence
            Eloquent ORM&
            % framework
            Laravel&
            % supported languages
            EN&
            % extensible i18n
            \reqyes{}&
            % user
            \reqyes{}&
            % admin
            \reqyes{}&
            % social
            \reqyes{}&
            % enterprise
            \reqno{}&
            % backers
            Private&
            % forks
            4&
            % contributors
            3&
            % license
            MIT&
            % license commercially viable?
            \reqyes{}&
            % community
            \reqno{}&
            % commercial support offered?
            \reqno{}\\

            % reference
            \cite{Honor}&
            % dependency
            \reqyes{}&
            % deployment
            \reqno{}&
            % plugin
            \reqno{}&
            % programming languages
            Ruby&
            % persistence
            Active Record&
            % framework
            Rails&
            % supported languages
            \reqna{}&
            % extensible i18n
            \reqna{}&
            % user
            \reqno{}&
            % admin
            \reqno{}&
            % social
            \reqno{}&
            % enterprise
            \reqno{}&
            % backers
            Private&
            % forks
            9&
            % contributors
            2&
            % license
            MIT&
            % license commercially viable?
            \reqyes{}&
            % community
            \reqno{}&
            % commercial support offered?
            \reqno{}\\

            % reference
            \cite{PhpGamify}&
            % dependency
            \reqyes{}&
            % deployment
            \reqno{}&
            % plugin
            \reqno{}&
            % programming languages
            PHP&
            % persistence
            *SQL&
            % framework
            \reqno{}&
            % supported languages
            \reqna{}&
            % extensible i18n
            \reqno{}&
            % user
            \reqpartial{}&
            % admin
            \reqno{}&
            % social
            \reqno{}&
            % enterprise
            \reqno{}&
            % backers
            Company&
            % forks
            4&
            % contributors
            1&
            % license
            MIT&
            % license commercially viable?
            \reqyes{}&
            % community
            \reqno{}&
            % commercial support offered?
            \reqno{}\\

            % reference
            \cite{AcisGamificationFramework}&
            % dependency
            \reqno{}&
            % deployment
            \reqyes{}&
            % plugin
            \reqno{}&
            % programming languages
            Java&
            % persistence
            JPA&
            % framework
            \reqno{}&
            % supported languages
            \reqna{}&
            % extensible i18n
            \reqno{}&
            % user
            \reqno{}&
            % admin
            \reqno{}&
            % social
            \reqno{}&
            % enterprise
            \reqno{}&
            % backers
            University&
            % forks
            2&
            % contributors
            4&
            % license
            Apache 2.0&
            % license commercially viable?
            \reqyes{}&
            % community
            \reqno{}&
            % commercial support offered?
            \reqno{}\\

            % % reference
            % &
            % % dependency
            % &
            % % deployment
            % &
            % % plugin
            % &
            % % programming languages
            % &
            % % persistence
            % &
            % % framework
            % &
            % % supported languages
            % &
            % % extensible i18n
            % &
            % % user
            % &
            % % admin
            % &
            % % social
            % &
            % % enterprise
            % &
            % % backers
            % &
            % % forks
            % &
            % % contributors
            % &
            % % license
            % &
            % % license commercially viable?
            % &
            % % community
            % &
            % % commercial support offered?
            % \\
            \bottomrule
        \end{tabular}}
    \end{table}
\end{landscape}
\restoregeometry%

\appendix

\section{Literature Notes}

The following are supplementary notes and summaries of key references for this thesis.

\paragraph{\cite{APLLAC24}} While the primary output of this paper (a review of gamification in the field of mobility and transport) doesn't immediately seem relevant, its introductory section goes into comparatively great detail and is very well researched---it has served well as a means to find more relevant high-quality literature. Also notably one goal was the evaluation of the effectiveness of gamification where \enquote{the results were rather mixed between positive and neutral results, with even some negative results}.

\paragraph{\cite{BozHS24}} Proposes \enquote{criteria to consider when selecting [\ldots] platforms for gamification} which it bases on the MDA framework~\cite{HunLZ04}, originally developed for game design. The primary output of this paper is their categorisation of gamification criteria, which is of great interest for the purpose of this thesis.

\paragraph{\cite{Chou15}} Provides a detailed description of the Octalysis framework, which appears to be one of the more popular frameworks for gamification. This is the most complete reference for Octalysis, but a more accessible summary is given e.g.~in~\cite{MohaB23}.

\paragraph{\cite{GearB13}} An example of a more abstract framework for gamification, in particular one which explicitly considers economic factors.

\paragraph{\cite{HaKS14}} Examines 24 (empirical, peer-reviewed) studies, establishing a framework to examine the effectiveness of gamification. As such it focuses on somewhat different aspects than e.g.~\cite{Herz14}, but it is particularly valuable for its overview of the field.

\paragraph{\cite{HeAWS15}} Describes gamification as a software development process and lays out requirements for a successful introduction of gamification. Existing gamification solutions are analyzed with regards to these requirements. Hits on many of the same notes as other work from the main author, e.g.~\cite{Herz14} but focuses on requirements relevant in the implementation/provisioning phase.

\paragraph{\cite{Herz14}} Examines 37 studies to identify gamification requirements, but notes that much work has been done since then and refers to~\cite{HaKS14} for a more recent review.

\paragraph{\cite{MRGA15}} Identifies 10 \enquote{framework features} and judges 18 papers describing gamification frameworks or generic gamification process descriptions accordingly. Their individual reasoning is not always given, but in any case the feature groups are what is most interesting for the purpose of this thesis.

\paragraph{\cite{MRGA17}} Follows a similar methodology to~\cite{MRGA15} (same authors) but considers more papers and has a stronger focus on education as opposed to more general business applications. Given their narrow scope, the actual frameworks under review here are likely not very interesting for future reference, but the feature classification remains valuable.

\paragraph{\cite{SANCN24}} Evaluates a \enquote{gamified} app based on features of the Octalysis framework~\cite{Chou15}. Since a singular instance of a tailor-made gamification implementation is evaluated, it is not exactly applicable to the evaluation of gamification platforms in general. But it nevertheless serves as an example of how a concrete feature-based evaluation can look like. (As such, similar to~\cite{EDPKM15, ChrWa21}.)

\printbibliography[
    notkeyword=framework
]

\newrefcontext[labelprefix=F, sorting=none]
\printbibliography[
    keyword=framework,
    title={Frameworks},
]

\end{document}
