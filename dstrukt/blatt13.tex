\documentclass{article}
\usepackage[utf8]{inputenc}
\usepackage[ngerman]{babel}

% Convenience improvements
\usepackage{csquotes}
\usepackage{enumitem}
\setlist[enumerate,1]{label={\alph*)}}
\usepackage{amsmath}
\usepackage{amssymb}
\usepackage{mathtools}

% Proper tables and centering for overfull ones
\usepackage{booktabs}
\usepackage{adjustbox}

% Change page/text dimensions, the package defaults work fine
\usepackage{geometry}

\usepackage{parskip}

% Drawings
\usepackage{tikz}

% Adjust header and footer
\usepackage{fancyhdr}
\pagestyle{fancy}
\fancyhead[L]{Diskrete Strukturen --- \textbf{Übungsblatt 13}}
\fancyhead[R]{Laurenz Weixlbaumer (11804751)}
\fancyfoot[C]{}
\fancyfoot[R]{\thepage}
% Stop fancyhdr complaints
\setlength{\headheight}{12.5pt}

\newcommand{\R}{\mathbb{R}\ \\\ \{0\}}

\newcommand{\cmod}{\text{mod}}

\newcommand{\inline}{}

\begin{document}

\paragraph{Aufgabe 1.}

Induktionsanfang: Für $n = 0$ gilt
\begin{align*}
    \sum_{k = 0}^{0} (-1)^k {{m + 1}\choose{k}}
    &=
    (-1)^0 {{m}\choose{0}}
    \\
    (-1)^0 {{m + 1}\choose{0}}
    &=
    (-1)^0 {{m}\choose{0}}
    \\
    1 &= 1
\end{align*}
nachdem für beliebiges $x \in \mathbb{N}$ gilt, dass $x^0 = 1$ und ${{x}\choose{0}} = 1$.

Induktionsvorraussetzung: Angenommen $n \in \mathbb{N}$ ist so, dass
\begin{equation*}
    \sum_{k = 0}^{n} (-1)^k {{m + 1}\choose{k}} = (-1)^n {{m}\choose{n}}
\end{equation*}
gilt.

Induktionsschritt: Zu zeigen ist, dass dann auch
\begin{gather}
    \label{eq:induction-step}
    \begin{aligned}
        \sum_{k = 0}^{n + 1} (-1)^k {{m + 1}\choose{k}} &= (-1)^{n + 1} {{m}\choose{n + 1}} \\
        \left( \sum_{k = 0}^{n} (-1)^k {{m + 1}\choose{k}} \right) + (-1)^{n + 1} {{m + 1}\choose{n + 1}}
        &=
        (-1)^{n + 1} {{m}\choose{n + 1}} \\
        (-1)^n {{m}\choose{n}} + (-1)^{n + 1} {{m + 1}\choose{n + 1}} &= (-1)^{n + 1} {{m}\choose{n + 1}}
    \end{aligned}
\end{gather}
gilt. Wenn $n$ gerade dann $(-1)^n = 1$, andernfalls $(-1)^n = -1$. Angenommen $n$ gerade (also $n + 1$ ungerade), dann gilt
\begin{align*}
    {{m}\choose{n}} - {{m + 1}\choose{n + 1}} &= -{{m}\choose{n + 1}} \\
    -{{m + 1}\choose{n + 1}} &= -{{m}\choose{n + 1}} - {{m}\choose{n}} \\
    {{m + 1}\choose{n + 1}} &= {{m}\choose{n}} + {{m}\choose{n + 1}}.
\end{align*}
Angenommen $n$ ungerade (also $n + 1$ gerade), dann gilt
\begin{align*}
    -{{m}\choose{n}} + {{m + 1}\choose{n + 1}} &= {{m}\choose{n + 1}} \\
    {{m + 1}\choose{n + 1}} &= {{m}\choose{n}} + {{m}\choose{n + 1}}.
\end{align*}
Gemäß der Rekurrenz des Pascal-Dreiecks ${{m + 1}\choose{n + 1}} = {{m}\choose{n}} + {{m}\choose{n + 1}}$ ist nun \eqref{eq:induction-step} bewiesen.

% Induktionsschritt: Zu zeigen ist, dass dann auch
% \begin{align*}
%     \sum_{k = 0}^{n + 1} (-1)^k \begin{pmatrix}
%         m + 1\\
%         k
%     \end{pmatrix}
%     &=
%     (-1)^{n + 1} \begin{pmatrix}
%         m\\
%         n + 1
%     \end{pmatrix}
%     \\
%     \left(
%     \sum_{k = 0}^{n} (-1)^k \begin{pmatrix}
%         m + 1\\
%         k
%     \end{pmatrix}
%     \right)
%     +
%     (-1)^{n + 1} \begin{pmatrix}
%         m + 1\\
%         n + 1
%     \end{pmatrix}
%     &=
%     (-1)^{n + 1} \begin{pmatrix}
%         m\\
%         n + 1
%     \end{pmatrix}
% \end{align*}
% gilt.

% Induktionsanfang: Zu zeigen ist, dass für $m = 0$ gilt, dass
% \begin{equation}\label{eq:induction-start}
%     \sum^{n}_{k = 0} (-1)^{k} {{1}\choose{k}} = (-1)^{n} {{0}\choose{n}}.
% \end{equation}
% Für $k \in \mathbb{N}$ mit $k \leq 1$ gilt ${{1}\choose{k}} = 1$, für $k > 1$ gilt ${{1}\choose{k}} = 0$ (Skriptum, S. 77). Daraus ergibt sich
% \begin{align*}
%     \sum^{0}_{k = 0} (-1)^{k} {{1}\choose{k}} = (-1)^{0} {{1}\choose{0}} &= 1 \\
%     \sum^{1}_{k = 0} (-1)^{k} {{1}\choose{k}} = 1 + (-1)^{1} {{1}\choose{1}} &= 0 \\
%     \sum^{n}_{k = 0} (-1)^{k} {{1}\choose{k}} = 1 + (-1) + 0 + \cdots + 0 &= 0.
% \end{align*}
% Für $n \in \mathbb{N}$ mit $n = 0$ gilt $(-1)^{n} {{0}\choose{n}} = 1$, für $n > 0$ gilt $(-1)^{n} {{0}\choose{n}} = 0$ (ebd.), damit ist \eqref{eq:induction-start} bewiesen.

% Induktionsvorraussetzung: Angenommen $m \in \mathbb{N}$ ist so, dass
% \begin{equation*}
%     \sum^{n}_{k = 0} (-1)^{k} {{m + 1}\choose{k}} = (-1)^{n} {{m}\choose{n}}.
% \end{equation*}
% Induktionsschluss: Zu zeigen ist, dass dann auch
% \begin{equation*}
%     \sum^{n}_{k = 0} (-1)^{k} {{m + 2}\choose{k}} = (-1)^{n} {{m + 1}\choose{n}}
% \end{equation*}
% gilt.

% Es gilt $(-1)^{n} {{0}\choose{n}} = 0$ nachdem ${{0}\choose{n}} = 0$ für $n \in \mathbb{N}$ mit $n > 0$ (Skriptum, S. 77).

% Per Definition gilt ${{1}\choose{k}} = 1$ für $k \in \mathbb{N}$ mit $k \leq 1$ (ebd.) und ${{1}\choose{k}} = 0$ für $k > 1$ nachdem die Menge $\{1\}$ anschaulicherweise keine Teilmengen mit $> 1$ Elementen hat\footnote{${{n}\choose{k}}$ modelliert die Anzahl der Teilmengen von $\{1, \ldots, n\}$ mit genau $k$ Elementen.}. Damit gilt
% \begin{align*}
%     \sum^{0}_{k = 0} (-1)^{k} {{1}\choose{k}} = (-1)^{0} {{1}\choose{0}} &= 0 \\
%     \sum^{1}_{k = 0} (-1)^{k} {{1}\choose{k}} = 0 + (-1)^{1} {{1}\choose{0}} &= 0 \\
% \end{align*}

% \begin{align*}
%     \sum^{n}_{k = 0} (-1)^{k} \begin{pmatrix}
%         1 \\
%         k
%     \end{pmatrix}
%     &=
%     (-1)^{n} \begin{pmatrix}
%         0 \\
%         n
%     \end{pmatrix}
%     \\
%     \sum^{n}_{k = 0} (-1)^{k} \begin{pmatrix}
%         1 \\
%         k
%     \end{pmatrix}
%     &=
%     0
% \end{align*}

\paragraph{Aufgabe 2}

Induktionsanfang: Für $n = 0$ gilt
\begin{equation*}
    F_0 = (2F_1 - F_0)F_0
\end{equation*}

\paragraph{Aufgabe 3}

\end{document}

m + 1 = m + m
n + 1   n   n + 1

i = 1   0 = 0   0 = 1
0       i       0

(1, 0) = 1
(1, 1) = 1
(1, 2) = 0
(1, x) = 0 für x > 1

(xy)(xz) = x(yz)
