\documentclass{article}
\usepackage[utf8]{inputenc}
\usepackage[ngerman]{babel}

% Convenience improvements
\usepackage{csquotes}
\usepackage{enumitem}
\setlist[enumerate,1]{label={\alph*)}}
\usepackage{amsmath}
\usepackage{amssymb}
\usepackage{mathtools}

% Proper tables and centering for overfull ones
\usepackage{booktabs}
\usepackage{adjustbox}

% Change page/text dimensions, the package defaults work fine
\usepackage{geometry}

\usepackage{parskip}

% Drawings
\usepackage{tikz}

% Adjust header and footer
\usepackage{fancyhdr}
\pagestyle{fancy}
\fancyhead[L]{Diskrete Strukturen --- \textbf{Übungsblatt 12}}
\fancyhead[R]{Laurenz Weixlbaumer (11804751)}
\fancyfoot[C]{}
\fancyfoot[R]{\thepage}
% Stop fancyhdr complaints
\setlength{\headheight}{12.5pt}

\newcommand{\R}{\mathbb{R}\ \\\ \{0\}}

\newcommand{\cmod}{\text{mod}}

\usepackage{seqsplit}

\begin{document}

\paragraph{Aufgabe 1.}

Es gilt eine injektive Funktion $f$ von der Menge aller Partitionen von $\{ 1, \ldots, n \}$ in die Menge $S_n$ (also die Menge aller Permutationen von $\{ 1, \ldots, n \}$) zu konstruieren.
\begin{equation*}
    f(\{ p_1, p_2, \ldots p_j \}) = 
\end{equation*}

% Dafür hilfreich ist eine Funktion $z : \mathcal{P}(\mathbb{N}) \rightarrow S_n$, welche eine Menge von ganzen Zahlen in einen Zyklus überführt.
% \begin{equation*}
%     z(\{ 1, 2, \ldots, n \}) = (1, 2, \ldots, n)
% \end{equation*}

% Damit kann nun
% \begin{equation*}
%     f(\{ p_1, p_2, \ldots p_j \}) = z(p_1) \circ z(p_2) \circ \cdots \circ z(p_j)
% \end{equation*}
% formuliert werden.

\end{document}

partitionen von {1, 2, 3} (5)
{ {1}, {2}, {3} }, { {1, 2, 3} }, { {1, 2}, {3} }, { {1}, {2, 3} }, { {2}, {1, 3} }

permutationen von S_3, zyklennotation (6)
(1)(2)(3), (1)(2 3), (1 2)(3), (1 2 3), (1 3 2), (1 3)(2)

injektiv: nicht alles muss getroffen werden, aber niemals doppelt


permutationen von S_4
{1, 2, 3, 4}, {1, 2, 4, 3}, {1, 3, 2, 4}, {1, 3, 4, 2}, {1, 4, 2, 3}, {1, 4, 3, 2}, 
{2, 1, 3, 4}, {2, 1, 4, 3}, {2, 3, 1, 4}, {2, 3, 4, 1}, {2, 4, 1, 3}, {2, 4, 3, 1}, {3, 1, 2, 4}, 
{3, 1, 4, 2}, {3, 2, 1, 4}, {3, 2, 4, 1}, {3, 4, 1, 2}, {3, 4, 2, 1}, {4, 1, 2, 3}, {4, 1, 3, 2},  
{4, 2, 1, 3}, {4, 2, 3, 1}, {4, 3, 1, 2}, {4, 3, 2, 1}

z : nimmt permutation, gibt menge der disjunkten zyklen zurück

(1)(2)(3)(4), (1)(2)(3 4), (1)(2 3)(4), (1)(2 3 4), (1)(2 4 3), (1)(2 4)(3)
{1}{2}{3}{4}



