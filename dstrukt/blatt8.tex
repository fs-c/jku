\documentclass{article}
\usepackage[utf8]{inputenc}
\usepackage[ngerman]{babel}

% Convenience improvements
\usepackage{csquotes}
\usepackage{enumitem}
\setlist[enumerate,1]{label={\alph*)}}
\usepackage{amsmath}
\usepackage{amssymb}
\usepackage{mathtools}

% Proper tables and centering for overfull ones
\usepackage{booktabs}
\usepackage{adjustbox}

% Change page/text dimensions, the package defaults work fine
\usepackage{geometry}

\usepackage{parskip}

% Drawings
\usepackage{tikz}

% Adjust header and footer
\usepackage{fancyhdr}
\pagestyle{fancy}
\fancyhead[L]{Diskrete Strukturen --- \textbf{Übungsblatt 8}}
\fancyhead[R]{Laurenz Weixlbaumer (11804751)}
\fancyfoot[C]{}
\fancyfoot[R]{\thepage}
% Stop fancyhdr complaints
\setlength{\headheight}{12.5pt}

\newcommand{\R}{\mathbb{R}\ \\\ \{0\}}

\newcommand{\frectangle}{\tikz[scale=0.7, baseline]{\draw[fill] (0, 0) rectangle (1em, 1em)}}
\newcommand{\fcircle}   {\tikz[scale=0.7, baseline]{\draw[fill] (0.5em, 0.5em) circle [radius=0.5em]}}
\newcommand{\ftriangle} {\tikz[scale=0.7, baseline]{\draw[fill] (0, 0) -- (0.5em, 1em) -- (1em, 0) -- cycle}}

\begin{document}

\paragraph{Aufgabe 1}

Wir wissen
\begin{align*}
    \forall,\ u_1, v_1 \in U_1 &: u_1 \circ v_1 \in U_1 \land u_1^{-1} \in U_1 \\
    \forall,\ u_2, v_2 \in U_2 &: u_2 \circ v_2 \in U_2 \land u_2^{-1} \in U_2.
\end{align*}

\begin{enumerate}
    \item Zu zeigen ist, dass $U_1 \cap U_2 \neq \emptyset$ und $\forall\, u, v \in U_1 \cap U_2 : u \circ v \in U_1 \cap U_2 \land u^{-1} \in U_1 \cap U_2$.
    
    Ist $e$ das Neutralelement einer Gruppe $G$ so gilt für alle Untergruppen $U$ von $G$, dass $e \in U$. Demzufolge gilt $e \in U_1$ und $e \in U_2$ und weiters $U_1 \cap U_2 \neq \emptyset$.

    Wenn $u, v \in U_1 \cap U_2$ dann gilt $u, v \in U_1$ und $u, v \in U_2$. Nachdem $U_1$ und $U_2$ Untergruppen sind gilt weiters $u \circ v \in U_1$ und $u \circ v \in U_2$. Daraus folgt $u \circ v \in U_1 \cap U_2$.

    Analog dazu, wenn $u \in U_1 \cap U_2$ dann gilt $u \in U_1$ und $u \in U_2$. Weiters gilt $u^{-1} \in U_1$ und $u^{-1} \in U_2$. Daraus folgt $u^{-1} \in U_1 \cap U_2$.

    Demzufolge ist $U_1 \cap U_2$ eine Untergruppe von $G$.

    \item Zu zeigen ist, dass $U_1 \cup U_2 \neq \emptyset$ und $\forall\, u, v \in U_1 \cup U_2 : u \circ v \in U_1 \cup U_2 \land u^{-1} \in U_1 \cup U_2$.

    Nachdem $U_1$ und $U_2$ Untergruppen sind muss gelten, dass $U_1 \neq \emptyset$ und $U_2 \neq \emptyset$. Demzufolge gilt $U_1 \cup U_2 \neq \emptyset$.
        
    Man wähle ein $u_1 \in U_1 \backslash U_2$ und ein $u_2 \in U_2 \backslash U_1$. Es gilt $u_1, u_2 \in U_1 \cup U_2$. Unter der Annahme, dass $U_1 \cup U_2$ eine Untergruppe ist, muss nun gelten, dass
    \begin{align*}
        u_1 \circ u_2 \in U_1 \cup U_2 &\Rightarrow \begin{cases}
            u_1 \circ u_2 \in U_1 & \text{oder} \\
            u_1 \circ u_2 \in U_2
        \end{cases}
    \end{align*}
    und weiters, nachdem es in einer Untergruppe $U$ für jedes $v \in U$ ein $v^{-1} \in U$ gibt,
    \begin{align*}
        u_1 \circ u_2 \in U_1 \Rightarrow u_1 \circ u_1^{-1} \circ u_2 \in U_1 &\Rightarrow u_2 \in U_1 \\
        u_1 \circ u_2 \in U_2 \Rightarrow u_2 \circ u_2^{-1} \circ u_1 \in U_2 &\Rightarrow u_1 \in U_2.
    \end{align*}
    Beide \enquote{Implikationsmöglichkeiten} von $u_1 \circ u_2 \in U_1 \cup U_1$ führen also zu einem Widerspruch nachdem gilt, dass $u_1 \notin U_2$ und $u_2 \notin U_1$. Demnach ist $U_1 \cup U_2$ nicht zwingend eine Untergruppe von $G$.

    (Tatsächlich wurde gezeigt, dass $U_1 \cup U_2$ nur dann eine Untergruppe von $G$ ist, wenn entweder für alle $u_1 \in U_1$ gilt, dass $u_1 \in U_2$, oder für alle $u_2 \in U_2$ gilt, dass $u_2 \in U_1$ --- also wenn $U_1 \subseteq U_2$ oder $U_2 \subseteq U_1$.)
\end{enumerate}

\paragraph{Aufgabe 2}

\begin{enumerate}
    \item $\langle (1\ 2)(3\ 4) \rangle = \{ \text{\emph{id}, (1\ 2)(3\ 4) } \}$
    
    \item $\langle (1\ 2), (3\ 4) \rangle = \{ \text{\emph{id}}, (1\ 2), (3\ 4), (1\ 2)(3\ 4) \}$
    
    \item $\langle (1\ 2\ 3\ 4) \rangle = \{ \text{\emph{id}}, (1\ 2\ 3\ 4), (4\ 3\ 2\ 1), (1\ 3)(2\ 4) \}$
    
    \item $\langle (1\ 2\ 3\ 4), (1\ 3) \rangle = \{ \text{\emph{id}}, (1\ 2\ 3\ 4), (1\ 3), (4\ 3\ 2\ 1), (1\ 3)(2\ 4), (1\ 4)(2\ 3), (1\ 2)(3\ 4), (2\ 4) \}$
\end{enumerate}

\paragraph{Aufgabe 3}

\begin{enumerate}
    \item $\langle (1\ 2\ 3), (4\ 5) \rangle$
    \item $\langle (1\ 2\ 3\ 4\ 5) \rangle$
    \item $\langle (2\ 4)(3\ 5) \rangle$
\end{enumerate}

\end{document}
