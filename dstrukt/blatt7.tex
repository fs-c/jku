\documentclass{article}
\usepackage[utf8]{inputenc}
\usepackage[ngerman]{babel}

% Convenience improvements
\usepackage{csquotes}
\usepackage{enumitem}
\setlist[enumerate,1]{label={\alph*)}}
\usepackage{amsmath}
\usepackage{amssymb}
\usepackage{mathtools}

% Proper tables and centering for overfull ones
\usepackage{booktabs}
\usepackage{adjustbox}

% Change page/text dimensions, the defaults work fine
\usepackage{geometry}

\usepackage{parskip}

% Drawings
\usepackage{tikz}

% Adjust header and footer
\usepackage{fancyhdr}
\pagestyle{fancy}
\fancyhead[L]{Diskrete Strukturen --- \textbf{Übungsblatt 7}}
\fancyhead[R]{Laurenz Weixlbaumer (11804751)}
\fancyfoot[C]{}
\fancyfoot[R]{\thepage}
% Stop fancyhdr complaints
\setlength{\headheight}{12.5pt}

\newcommand{\R}{\mathbb{R}\ \\\ \{0\}}

\newcommand{\frectangle}{\tikz[scale=0.7, baseline]{\draw[fill] (0, 0) rectangle (1em, 1em)}}
\newcommand{\fcircle}   {\tikz[scale=0.7, baseline]{\draw[fill] (0.5em, 0.5em) circle [radius=0.5em]}}
\newcommand{\ftriangle} {\tikz[scale=0.7, baseline]{\draw[fill] (0, 0) -- (0.5em, 1em) -- (1em, 0) -- cycle}}

\begin{document}

\paragraph{Aufgabe 1}

\begin{enumerate}
    \item $(1\ 5\ 6\ 4\ 8\ 9)(2\ 7\ 2)$
    \item $(1\ 9\ 2\ 6\ 3\ 7\ 8\ 4)$
    \item Sei $\pi = (c_1\ c_2\ \ldots\ c_k)$ ein beliebiger Zyklus. Es gilt $\pi(c_i) = c_{i + 1}$ und $\pi^{-1}(c_{i + 1}) = c_i$, also $\pi^{-1} = (c_k\ c_{k-1}\ \ldots\ c_1)$. Konkret gilt nun
    \begin{align*}
        \pi &= (1\ 3\ 7\ 2\ 5\ 4)\\
        \pi^{-1} &= (4\ 5\ 2\ 7\ 3\ 1)\\
        \pi^{-2} &= (1\ 3\ 7\ 2\ 5\ 4)\\
        \pi^{-3} &= (4\ 5\ 2\ 7\ 3\ 1)\\
    \end{align*}

    \item Es gilt
    \begin{align*}
        (1\ 3\ 7\ 2\ 5\ 4)^2 &= (1\ 7\ 5)(3\ 2\ 4)\\
        (1\ 3\ 7\ 2\ 5\ 4)^3 &= (1\ 5\ 7)(2\ 3\ 4)\\
        (1\ 3\ 7\ 2\ 5\ 4)^4 &= (1\ 7\ 5)(3\ 2\ 4)\\
        (1\ 3\ 7\ 2\ 5\ 4)^5 &= (1\ 5\ 7)(2\ 3\ 4)\\
        &\cdots\\
        (1\ 3\ 7\ 2\ 5\ 4)^{1000} &= (1\ 7\ 5)(3\ 2\ 4).\\
    \end{align*}
\end{enumerate}

\paragraph{Aufgabe 2}

\begin{enumerate}
    \item Man wähle aus $S_2$ die Permutation $\begin{psmallmatrix}1 & 2\\2 & 1\end{psmallmatrix}$. Diese Permutation besteht nur aus dem Zyklus $(1 2)$, müsste aber gemäß der gegebenen Aussage aus mindestens zwei Zyklen bestehen.

    \item Eine Permutation $\pi \in S_n$ sei in disjunkte Zyklen zerlegt. Die Summe der Längen dieser Zyklen ist $n$, nachdem sie disjunkt sind. Die kleinste Länge eines Zyklus ist 1. Somit kann $\pi$ in höchstens $n$ disjunkte Zyklen zerlegt werden. \label{itm:2b}
    
    \item Man wähle aus $S_2$ die Permutation $\pi = \begin{psmallmatrix}1 & 2\\2 & 1\end{psmallmatrix}$ mit dem Zyklus $(1 2)$. Die Permutation $\pi^2 = \begin{psmallmatrix}1 & 2\\1 & 2\end{psmallmatrix}$ besteht aus den disjunkten Zyklen $(1)(2)$. Somit besteht $\pi$ aus weniger Zyklen als $\pi^2$.

    \item Es gelte $\pi \in S_n$. Es gibt nun disjunkte Zyklen $c_1, c_2, \ldots, c_k$ derart, dass $\pi = c_1c_2 \ldots c_k$ und $\pi^2 = c_1^2c_2^2 \ldots c_k^2$. Ist $c$ ein Zyklus so besteht $c^2$ aus mindestens einem Zyklus. Somit sind in $\pi^2$ mindestens so viele Zyklen wie in $\pi$ enthalten, in anderen Worten besteht $\pi$ aus höchstens so vielen disjunkten Zyklen wie $\pi^2$.
\end{enumerate}

\paragraph{Aufgabe 3}

\begin{enumerate}[label=\arabic*.]
    \item Zu zeigen ist $\forall\, x, y, z \in G : (x * y) * z = x * (y * z)$. (Assoziativität.)

    Seien $(a_1, a_2), (b_1, b_2), (c_1, c_2) \in G$ beliebig. Es gilt
    \begin{align*}
        ((a_1, a_2) * (b_1, b_2)) * (c_1, c_2) &=\\
        (a_1b_1 - a_2b_2, a_1b_2 + a_2b_1) * (c_1, c_2) &=\\
        ((a_1b_1 - a_2b_2)c_1 - (a_1b_2 + a_2b_1)c_2, (a_1b_1 - a_2b_2)c_2 + (a_1b_2 + a_2b_1)c_1) &=\\
        ((a_1b_1c_1 - a_2b_2c_1) - (a_1b_2c_2 + a_2b_1c_2), (a_1b_1c_2 - a_2b_2c_2) + (a_1b_2c_1 + a_2b_1c_1)) &= \\
        (a_1b_1c_1 - a_2b_2c_1 - a_1b_2c_2 - a_2b_1c_2, a_1b_1c_2 - a_2b_2c_2 + a_1b_2c_1 + a_2b_1c_1) &\phantom{=}
    \end{align*}
    und
    \begin{align*}
        (a_1, a_2) * ((b_1, b_2) * (c_1, c_2)) &=\\
        (a_1, a_2) * (b_1c_1 - b_2c_2, b_1c_2 + b_2c_1) &=\\
        (a_1(b_1c_1 - b_2c_2) - a_2(b_1c_2 + b_2c_1), a_1(b_1c_2 + b_2c_1) + a_2(b_1c_1 - b_2c_2)) &=\\
        ((a_1b_1c_1 - a_1b_2c_2) - (a_2b_1c_2 + a_2b_2c_1), (a_1b_1c_2 + a_1b_2c_1) + (a_2b_1c_1 - a_2b_2c_2)) &=\\
        (a_1b_1c_1 - a_1b_2c_2 - a_2b_1c_2 - a_2b_2c_1, a_1b_1c_2 + a_1b_2c_1 + a_2b_1c_1 - a_2b_2c_2) &\phantom{=}
    \end{align*}
    und somit
    \begin{equation*}
        ((a_1, a_2) * (b_1, b_2)) * (c_1, c_2) = (a_1, a_2) * ((b_1, b_2) * (c_1, c_2)).
    \end{equation*}

    \item Zu zeigen ist $\exists\, e \in G : \forall\, x \in G : x * e = e *x = x$. (Neutralelement.)

    Für $e = (1, 0)$ und beliebige $(a, b) \in G$ gilt
    \begin{alignat*}{2}
        (a, b) * (1, 0) &= (1, 0) * (a, b) &= (a, b)\\
        (a - 0, 0 + b) &= (a - 0, 0 + b) &= (a, b).
    \end{alignat*}

    \item Zu zeigen ist $\forall\, x \in G : \exists\, y \in G : x * y = y * x = e$, wobei $e \in G$ das Neutralelement ist. (Invertierbarkeit.)

    Für beliebige $(a, b) \in G$ gebe es ein $c \in G$ derart, dass
    \begin{equation*}
        (a, b) * c = c * (a, b) = (1, 0).
    \end{equation*}
    Man wähle
    \begin{equation*}
        c = \left(\frac{a}{a^2 + b^2}, -\frac{b}{a^2 + b^2}\right),
    \end{equation*}
    nun gilt
    \begin{align*}
        (a, b) * \left(\frac{a}{a^2 + b^2}, -\frac{b}{a^2 + b^2}\right) = \left(\frac{a}{a^2 + b^2}, -\frac{b}{a^2 + b^2}\right) * (a, b) &= (1, 0)\\
        \left(\frac{a^2}{a^2+b^2} + \frac{b^2}{a^2 + b^2}, \frac{ab}{a^2+b^2} - \frac{ab}{a^2+b^2}\right) &= (1, 0)\\
        \left(\frac{a^2 + b^2}{a^2+b^2}, \frac{ab - ab}{a^2+b^2}\right) &= (1, 0)\\
        (1, 0) &= (1, 0).\\
    \end{align*}
\end{enumerate}

\end{document}

(1 3 7 2 5 4)

1 2 3 4 5 6 7
3 5 7 1 4 6 2

-1
(4 5 2 7 3 1)

1 2 3 4 5 6 7
4 7 1 5 2 6 3

-2
(1 3 7 2 5 4)
1 2 3 4 5 6 7
3 5 7 1 4 6 2
