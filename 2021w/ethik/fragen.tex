\documentclass{article}
\usepackage[utf8]{inputenc}
\usepackage[ngerman]{babel}

% Convenience improvements
\usepackage{csquotes}
\usepackage{enumitem}
\usepackage{amsmath}
\usepackage{amssymb}

% Proper tables and centering for overfull ones
\usepackage{booktabs}
\usepackage{adjustbox}

% Change page/text dimensions, the defaults work fine
\usepackage{geometry}

\usepackage{parskip}

% Drawings
\usepackage{tikz}

% Adjust header and footer
\usepackage{fancyhdr}
\pagestyle{fancy}
\fancyhead[L]{Ethik --- \textbf{Abgabe 3}}
\fancyhead[R]{Laurenz Weixlbaumer (11804751)}
\fancyfoot[C]{}
\fancyfoot[R]{\thepage}

% Specific to this particular exercise
\renewcommand{\thetable}{\alph{table}}
\usepackage{caption}
\DeclareCaptionLabelFormat{custom}{Truth Table (#2):}
\captionsetup{labelformat={custom}, labelsep=space}

\newcommand{\R}{\mathbb{R}\ \\\ \{0\}}

\newcommand{\frectangle}{\tikz[scale=0.7, baseline]{\draw[fill] (0, 0) rectangle (1em, 1em)}}
\newcommand{\fcircle}   {\tikz[scale=0.7, baseline]{\draw[fill] (0.5em, 0.5em) circle [radius=0.5em]}}
\newcommand{\ftriangle} {\tikz[scale=0.7, baseline]{\draw[fill] (0, 0) -- (0.5em, 1em) -- (1em, 0) -- cycle}}

\begin{document}
    \begin{enumerate}
        \item Was zählt zu den äußerlich wahrnehmbaren Unterschieden zwischen Menschen?
        \begin{enumerate}[label=(\alph*)]
            \item Geschlecht
            \item Alter
            \item körperliche Behinderungen
            \item Haarfarbe
            \item Religion
            \item Drogenkonsum
            \item Erziehung
            \item geistige Behinderungen
            \item sexuelle Orientierung
        \end{enumerate}
        
        \item Welche der folgenden Aussagen über Stereotype sind zutreffend?
        
        Steretype sind\ldots
        
        \begin{enumerate}[label=(\alph*)]
            \item Vorurteile.
            \item keine Vorurteile.
            \item ausschließlich unbewusst.
            \item gefühlscolorierte Vorstellungen.
            \item netrale Vorstellungen.
        \end{enumerate}
        
        \item Beschrieben Sie den Begriff \enquote{Unconscious Bias}.
        
        Der Begriff setzt sich aus dem Englischen Wörtern \emph{unconscious} (unbewusst) und \emph{bias} (Vorurteil) zusammen, bedeutet also eine unbewusste Voreingenommenheit. Er beschreibt eine unbewusste kognitive Verzerrung die zu einer fehlerbehafteten Wahrnehmung, Erinnerung und Beurteilung führt.
        
        \item Welche der folgenden Aussagen über Diskrimierung sind zutreffend?
        
        Diskriminierung ist\ldots
        
        \begin{enumerate}[label=(\alph*)]
            \item ausschließlich negativ.
            \item ausschließlich positiv.
            \item eine Verhaltensreaktion auf stereotype Bewertungen.
            \item das Resultat einer Verhaltensreaktion auf stereotype Bewertungen.
        \end{enumerate}
    \end{enumerate}
\end{document}
