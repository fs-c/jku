\documentclass{article}
\usepackage[utf8]{inputenc}
\usepackage[ngerman]{babel}

% Convenience improvements
\usepackage{csquotes}
\usepackage{enumitem}
\setlist[enumerate,1]{label={\alph*)}}
\usepackage{amsmath}
\usepackage{amssymb}
\usepackage{mathtools}

% Proper tables and centering for overfull ones
\usepackage{booktabs}
\usepackage{adjustbox}

% Change page/text dimensions, the package defaults work fine
\usepackage{geometry}

\usepackage{parskip}

% Drawings
\usepackage{tikz}

% Adjust header and footer
\usepackage{fancyhdr}
\pagestyle{fancy}
\fancyhead[L]{Logic --- \textbf{FOBL}}
\fancyhead[R]{Laurenz Weixlbaumer (11804751)}
\fancyfoot[C]{}
\fancyfoot[R]{\thepage}
% Stop fancyhdr complaints
\setlength{\headheight}{12.5pt}

\newcommand{\R}{\mathbb{R}\ \\\ \{0\}}

\newcommand{\cmod}{\text{mod}}

\newcommand{\asymmetric}{\textit{asymmetric}\,}
\newcommand{\antisymmetric}{\textit{antisymmetric}\,}
\newcommand{\irreflexive}{\textit{irreflexive}\,}

\begin{document}

Binary relations $R$ operating on a set $A$ are said to be asymmetric, antisymmetric or irreflexive according to the following definitions for arbitrary but fixed $R$ and $A$.
\begin{align}
    \label{eq:asymmetric} \asymmetric(R, A) &\Longleftrightarrow \forall\, x, y \in A : R(x, y) \Rightarrow \neg R(y, x) \\
    \label{eq:antisymmetric} \antisymmetric(R, A) &\Longleftrightarrow \forall\, x, y \in A : (R(x, y) \land R(y, x)) \Rightarrow (x = y) \\
    \label{eq:irreflexive} \irreflexive(R, A) &\Longleftrightarrow \forall\, x \in A : \neg R(x, x)
\end{align}

We show that, again for arbitrary but fixed $R$ and $A$, the implication
\begin{equation*}
    \asymmetric(R, A) \Rightarrow (\antisymmetric(R, A) \land \irreflexive(R, A))
\end{equation*}
holds by assuming that $\asymmetric(R, A)$ is true and showing that $\antisymmetric(R, A)$ and $\irreflexive(R, A)$ then hold.

It is elementary that, for arbitrary but fixed $x, y \in A$ the statement $R(x, y) \land R(y, x)$ in $\eqref{eq:antisymmetric}$ is a contradiction. $R(x, y)$ implies that $\neg R(y, x)$, therefore $R(x, y) \land R(y, x)$ can never be true for fixed $x, y$ under the assumption of $\eqref{eq:asymmetric}$. Since the left-hand side of the implication in \eqref{eq:antisymmetric} is always false the statement will always be true. We have thus shown that $\antisymmetric(R, A)$ holds.

If $R(x, x)$ were true for arbitrary but fixed $x \in A$ it would lead to $R(x, x) \Rightarrow \neg R(x, x)$ being false. Since that is a contradiction (we know that \eqref{eq:asymmetric} holds for our $R$ and $A$) we conclude that $\neg R(x, x)$. We have thus shown that $\irreflexive(R, A)$ holds.

\end{document}
