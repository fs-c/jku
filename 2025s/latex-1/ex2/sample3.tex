\documentclass[a4paper, 12pt]{article}

\usepackage[utf8]{inputenc}
\usepackage[naustrian]{babel}
\usepackage[T1]{fontenc}

\usepackage[a4paper, margin=2cm, includeheadfoot]{geometry}

\usepackage{amsmath}
\usepackage{amssymb}

\usepackage{parskip}

\usepackage{fancyhdr}
\fancyhf{} % reset header/footer
\pagestyle{fancy}

\lhead{\leftmark}
\rhead{Seite \thepage}
\lfoot{\footnotesize{\LaTeX-\"Ubung 2, erstellt am \today}}
\rfoot{\footnotesize{Laurenz Weixlbaumer (k11804751)}}

\renewcommand{\headrulewidth}{1.5pt}
\renewcommand{\footrulewidth}{1.5pt}

\begin{document}

\section{Matrixmultiplikation}

\subsection{Berechnung Matrixmultiplikation}

\begin{equation*}
    \begin{bmatrix}
        1 & 8 & 5 & -3 \\
        2 & -1 & 2 & 9 \\
        4 & -2 & 7 & 2 \\
        1 & -2 & 7 & -9
    \end{bmatrix}
    \cdot
    \begin{bmatrix}
        20 \\
        6 \\
        27 \\
        5
    \end{bmatrix}
    =
    \begin{bmatrix}
        188 \\
        133 \\
        267 \\
        152
    \end{bmatrix}
\end{equation*}

\subsection{Berechnung der transponierten Matrix}

\begin{equation*}
    \begin{bmatrix}
        188 \\ 133 \\ 267 \\ 152
    \end{bmatrix}^T
    =
    \begin{bmatrix}
        188 & 133 & 267 & 152
    \end{bmatrix}
\end{equation*}

\section{Fourierreihen}

Sei $(b_k)_{k \in \mathbb{Z}}$ ein Orthonormalsystem in $V$, sei $f \in V$, und seien $c_k = \langle f, b_k \rangle$, $k \in \mathbb{Z}$, die Fourierkoeffizienten von $f$ bezüglich des Orthogonalsystems $(b_k)_{k \in \mathbb{Z}}$. Dann gilt

\begin{equation}
    \sum_{k = -\infty}^{\infty} |c_k|^2 \leq \|f\|^2_2 \qquad \text{(Besselsche Ungleichung).}
\end{equation}

Genau dann konvergiert die Fourierreihe von $f$ im quadratischen Mittel gegen $f$, d. h. es gilt

\begin{equation*}
    \lim_{n \to \infty} \left\| f - \sum_{k = -n}^{n} c_k b_k \right\| = 0,
\end{equation*}

wenn

\begin{equation}\label{eq:parseval}
    \sum_{k = -\infty}^{\infty} |c_k|^2 = \|f\|^2_2 \qquad \text{(Parsevalsche Vollst\"andigkeitsrelation)}
\end{equation}

gilt. Das Orthonormalsystem $(b_k)_{k \in \mathbb{Z}}$ heißt vollst\"andig, wenn \eqref{eq:parseval} f\"ur jedes $f \in V$ gilt.

\end{document}